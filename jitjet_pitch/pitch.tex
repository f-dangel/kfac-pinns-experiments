\documentclass[12pt,usepdftitle=false,aspectratio=169]{beamer}

\makeatletter
\def\input@path{
  % {../..} % repository root
  % {../../latex-utilities} % LaTeX utilities
  % {../../repos/einconv-paper/tex/paper} % TN drawing commands
  % {../2023_12_01_Perimeter_Institute_45min} % PI talk slides
}
\makeatother
\graphicspath{
  % {../../} % Logo
}


% ===================================================================
% BEAMERTHEME
% ===================================================================
% requires repository root in LaTeX path (look-up in beamertheme/ dir)
\usetheme{/vector_institute/vector_institute}
% disable logo
% \noLogo

% ===================================================================
% REFERENCES
% ===================================================================
\usepackage{natbib}

% ===================================================================
% FIGURES
% ===================================================================
\usetikzlibrary{matrix}
\usepackage{animate}

% ===================================================================
% MATH
% ===================================================================
\usepackage{amsmath,amssymb}
\input{../paper/preamble/goodfellow.tex}

% ===================================================================
% META-DATA
% ===================================================================
% \title{%
% \texttt{jitjet}: Efficient Laplacians in JAX by Compiling Taylor-mode AD
% }
\title{%
  Kronecker-factored Approximate Curvature for Physics-Informed Neural Networks
}
\author{Felix Dangel*, Johannes M\"uller*, Marius Zeinhofer*}
\date{NeurIPS 2024}

\begin{document}

\makeTitleSlide

\input{slides/tldr.tex}

\begin{frame}
  \frametitle{Conceptual Overview}
  \begin{itemize}
  \item Neural network ansatz $u_{\vtheta} = f^{(L)} \circ f^{(L-1)} \circ
    \ldots \circ f^{(1)}$
  \item Evaluating $u_{\vtheta}(\vx)$
    \begin{align*}
      \vz^{(0)} \mapsto \vz^{(1)}\mapsto \vz^{(2)} \mapsto \dots \mapsto \vz^{(L)} = u_{\vtheta}
    \end{align*}
    with vector-valued $\vz^{(l)} = f^{(l)}(\vz^{(l-1)})$ and $\vz^{(0)} = x$

  \item Evaluating $\gL u_{\vtheta}(\vx)$
    \begin{align*}
      \mZ^{(0)} \mapsto \mZ^{(1)} \mapsto \mZ^{(2)} \mapsto \ldots \mapsto \mZ^{(L)} \mapsto \gL u_{\vtheta}
    \end{align*}
    with matrix-valued $\mZ^{(l)}$ and dependencies determined by Taylor-mode

    If layer $l$ is a linear layer with weight $\mW$, then
    \begin{align*}
      \vz^{(l)} &= \mW \vz^{(l-1)}
      \\
      \mZ^{(l)} &= \mW \mZ^{(l-1)}
    \end{align*}

  \end{itemize}

\end{frame}

\begin{frame}[fragile,t]
  \frametitle{Taylor-mode Automatic Differentiation (Scalar Case)}
  \begin{figure}[t]
    \centering
    \hspace*{-3ex}
    \begin{tikzpicture}
      \matrix (magic)
      [matrix of nodes,
      ampersand replacement=\&,
      nodes={
        anchor=center,
        inner sep=2pt,
      }]
      {
        \only<2->{\textcolor{orange}{\textbf{Taylor}}}
        \&$f^{(0)} = \operatorname{id}$
        \& $f^{(1)} \circ f^{(0)}$
        \& $\dots$
        \& $f^{(L)} \circ \ldots \circ f^{(0)}$
        \\
        \only<1>{%
          \textcolor{blue}{\textbf{Function}}%
          \&%
          \includegraphics[scale=0.23]{/Users/fdangel/Documents/Postdoc/kfac-pinns/kfac_pinns_exp/exp48_visualization_taylor_mode/f_0.pdf}%
          \& \includegraphics[scale=0.23]{/Users/fdangel/Documents/Postdoc/kfac-pinns/kfac_pinns_exp/exp48_visualization_taylor_mode/f_1.pdf}%
          \& \includegraphics[scale=0.23]{/Users/fdangel/Documents/Postdoc/kfac-pinns/kfac_pinns_exp/exp48_visualization_taylor_mode/f_2.pdf}%
          \& \includegraphics[scale=0.23]{/Users/fdangel/Documents/Postdoc/kfac-pinns/kfac_pinns_exp/exp48_visualization_taylor_mode/f_3.pdf}%
        }
        \\
        \only<2>{
          \textcolor{orange}{\textbf{0\textsuperscript{th}-order}}
          \& \includegraphics[scale=0.23]{/Users/fdangel/Documents/Postdoc/kfac-pinns/kfac_pinns_exp/exp48_visualization_taylor_mode/f_0_taylor_0.pdf}
          \& \includegraphics[scale=0.23]{/Users/fdangel/Documents/Postdoc/kfac-pinns/kfac_pinns_exp/exp48_visualization_taylor_mode/f_1_taylor_0.pdf}
          \& \includegraphics[scale=0.23]{/Users/fdangel/Documents/Postdoc/kfac-pinns/kfac_pinns_exp/exp48_visualization_taylor_mode/f_2_taylor_0.pdf}
          \& \includegraphics[scale=0.23]{/Users/fdangel/Documents/Postdoc/kfac-pinns/kfac_pinns_exp/exp48_visualization_taylor_mode/f_3_taylor_0.pdf}
        }
        \only<3>{
          \textcolor{orange}{\textbf{1\textsuperscript{st}-order}}
          \& \includegraphics[scale=0.23]{/Users/fdangel/Documents/Postdoc/kfac-pinns/kfac_pinns_exp/exp48_visualization_taylor_mode/f_0_taylor_1.pdf}
          \& \includegraphics[scale=0.23]{/Users/fdangel/Documents/Postdoc/kfac-pinns/kfac_pinns_exp/exp48_visualization_taylor_mode/f_1_taylor_1.pdf}
          \& \includegraphics[scale=0.23]{/Users/fdangel/Documents/Postdoc/kfac-pinns/kfac_pinns_exp/exp48_visualization_taylor_mode/f_2_taylor_1.pdf}
          \& \includegraphics[scale=0.23]{/Users/fdangel/Documents/Postdoc/kfac-pinns/kfac_pinns_exp/exp48_visualization_taylor_mode/f_3_taylor_1.pdf}
        }
        \only<4>{
          \textcolor{orange}{\textbf{2\textsuperscript{nd}-order}}
          \& \includegraphics[scale=0.23]{/Users/fdangel/Documents/Postdoc/kfac-pinns/kfac_pinns_exp/exp48_visualization_taylor_mode/f_0_taylor_2.pdf}
          \& \includegraphics[scale=0.23]{/Users/fdangel/Documents/Postdoc/kfac-pinns/kfac_pinns_exp/exp48_visualization_taylor_mode/f_1_taylor_2.pdf}
          \& \includegraphics[scale=0.23]{/Users/fdangel/Documents/Postdoc/kfac-pinns/kfac_pinns_exp/exp48_visualization_taylor_mode/f_2_taylor_2.pdf}
          \& \includegraphics[scale=0.23]{/Users/fdangel/Documents/Postdoc/kfac-pinns/kfac_pinns_exp/exp48_visualization_taylor_mode/f_3_taylor_2.pdf}
        }
        \\
        \only<2->{
          \& $f^{(0)}(x) = x$
          \& $f^{(1)}(f^{(0)}(x))$
          \&$\dots$
          \&$f^{(L)}(f^{(L-1)}(x))$
        }
        \\
        \only<3->{
          \&$\partial_x f^{(0)}(x) = 1$
          \&$\partial_x f^{(1)}(x)$
          \&$\dots$
          \&$\partial_x f^{(L)}(x)$
        }
        \\
        \only<4->{
          \&$\partial^2_x f^{(0)}(x) = 0$
          \&$\partial^2_x f^{(1)}(x)$
          \&$\dots$
          \&$\partial^2_x f^{(L)}(x)$
        }
        \\
      };
    \end{tikzpicture}
  \end{figure}
\end{frame}
%%% Local Variables:
%%% mode: LaTeX
%%% TeX-master: "../pitch"
%%% End:

\begin{frame}[fragile,t]
  \frametitle{Taylor-mode Automatic Differentiation (Vector Case)}

  \begin{figure}[t]
    \centering
    \hspace*{-3ex}
    \begin{tikzpicture}
      \matrix (magic)
      [matrix of nodes,
      ampersand replacement=\&,
      nodes={
        anchor=center,
        inner sep=2pt,
      }]
      {
        \textcolor{orange}{\textbf{Taylor}}
        \&$f^{(0)} = \operatorname{id}$
        \& $f^{(1)} \circ f^{(0)}$
        \& $\dots$
        \& $f^{(L)} \circ \ldots \circ f^{(0)}$
        \\
        \textcolor{orange}{\textbf{2\textsuperscript{nd}-order}}
        \& \includegraphics[scale=0.23]{/Users/fdangel/Documents/Postdoc/kfac-pinns/kfac_pinns_exp/exp48_visualization_taylor_mode/f_0_taylor_2.pdf}
        \& \includegraphics[scale=0.23]{/Users/fdangel/Documents/Postdoc/kfac-pinns/kfac_pinns_exp/exp48_visualization_taylor_mode/f_1_taylor_2.pdf}
        \& \includegraphics[scale=0.23]{/Users/fdangel/Documents/Postdoc/kfac-pinns/kfac_pinns_exp/exp48_visualization_taylor_mode/f_2_taylor_2.pdf}
        \& \includegraphics[scale=0.23]{/Users/fdangel/Documents/Postdoc/kfac-pinns/kfac_pinns_exp/exp48_visualization_taylor_mode/f_3_taylor_2.pdf}
        \\
        \& $f^{(0)}(\vx) = \vx$
        \& $\left\{ f_i^{(1)}(f^{(0)}(\vx)) \right\}_i$
        \&$\dots$
        \&$\left\{ f_i^{(L)}(f^{(L-1)}(\vx)) \right\}_i$
        \\
        \only<2->{
          \&$\left\{\partial_{\vv} f_i^{(0)}(\vx) = v_{i}\right\}_i$
          \&$\left\{\partial_{\vv} f_i^{(1)}(\vx) \right\}_i$
          \&$\dots$
          \&$\left\{\partial_{\vv} f_i^{(L)}(\vx) \right\}_i$
        }
        \\
        \only<3->{
          \&$\left\{\partial^2_{\vv} f_i^{(0)}(\vx) = 0 \right\}_i$
          \&$\left\{\partial^2_{\vv} f_i^{(1)}(\vx) \right\}_i$
          \&$\dots$
          \&$\left\{\partial^2_{\vv} f_i^{(L)}(\vx) \right\}_i$
        }
        \\
      };
    \end{tikzpicture}
  \end{figure}
\end{frame}
%%% Local Variables:
%%% mode: LaTeX
%%% TeX-master: "../pitch"
%%% End:

\begin{frame}
  \frametitle{Taylor-mode for Linear Layers}

  \uncover<4->{
    \vspace{3ex}
    \ribbon[\paperwidth][black][VectorPink]{
      \centering
      \textbf{Linear layers propagate Taylor coefficients by applying their weight.}
      \uncover<5->{
        \\[2ex]
        \textbf{Computing any differential operator yields a net with linear weight sharing layers.}
      }}
  }

  \vspace{1.5ex}

  Let layer $l$ be a linear layer with weight $\mW$ (no bias for simplicity):
  \uncover<5->{
    \\
    We can stack the processed vectors into a matrix:
  }
  \only<1-4>{
    \begin{align*}
      \vz^{(l)}
      &= \mW\vz^{(l-1)}
      &\text{(function value)}
      \\
      \uncover<2->{
      \partial_{\vv}z_i^{(l)}
      &= \mW \partial_{\vv}\vz_i^{(l-1)}
      &\text{(directional slope)}
        }
      \\
      \uncover<3->{
      \partial^2_{\vv}z_i^{(l)} &= \mW \partial_{\vv}^2\vz_i^{(l-1)}
      &\text{(directional curvature)}
        }
      \\
      \uncover<4->{
      &\vdots
      \\
      \partial^K_{\vv}z_i^{(l)} &= \mW \partial_{\vv}^K\vz_i^{(l-1)}
                                  }
    \end{align*}
  }
  \only<5->{
    \begin{align*}
      \mZ^{(l)}
      &= \mW \mZ^{(l-1)}
        \intertext{with}
        \mZ^{(l)}
      &=
        \begin{pmatrix}
          \vz^{(l-1)} & \partial_{\vv}z_1^{(l-1)} & \partial_{\vv}z_2^{(l-1)} & \dots & \partial_{\vv}^2 z_1^{(l-1)} & \partial_{\vv}^2 z_d^{(l-1)} & \dots
        \end{pmatrix}
    \end{align*}
  }
\end{frame}
%%% Local Variables:
%%% mode: LaTeX
%%% TeX-master: "../pitch"
%%% End:

\begin{frame}[fragile,t]
  \frametitle{From Taylor-mode to Forward Laplacian {\scriptsize \citep{li2023forward}}}

  \only<3->{\vspace{-2ex}}
  \emph{Example:} Laplacian \quad $\Delta f_i^{(L)}(\vx) = \sum_j \frac{\partial^2 f_i^{(L)}(\vx)}{\partial x_j^2} = \sum_j \partial^2_{\ve_j} f_i^{(L)}(\vx)$
  \vspace{-1ex}

  \begin{figure}[t]
    \centering
    \hspace*{-3ex}
    \begin{tikzpicture}
      \matrix (magic)
      [matrix of nodes,
      ampersand replacement=\&,
      nodes={
        anchor=center,
        inner sep=2pt,
      }]
      {
        \includegraphics[scale=0.23]{../kfac_pinns_exp/exp48_visualization_taylor_mode/figures/f_0_taylor_2.pdf}
        \& \includegraphics[scale=0.23]{../kfac_pinns_exp/exp48_visualization_taylor_mode/figures/f_1_taylor_2.pdf}
        \& \includegraphics[scale=0.23]{../kfac_pinns_exp/exp48_visualization_taylor_mode/figures/f_2_taylor_2.pdf}
        \& \includegraphics[scale=0.23]{../kfac_pinns_exp/exp48_visualization_taylor_mode/figures/f_3_taylor_2.pdf}
        \\
        $f^{(0)}(\vx) = \vx$
        \& $\left\{ f_i^{(1)}(f^{(0)}(\vx)) \right\}_i$
        \&$\dots$
        \&$\left\{ f_i^{(L)}(f^{(L-1)}(\vx)) \right\}_i$
        \\
        $\left\{\partial_{\only<1>{\vv}\only<2->{\ve_j}} f_i^{(0)}(\vx) = v_{i}\right\}_{i\only<2->{,j}}$
        \&$\left\{\partial_{\only<1>{\vv}\only<2->{\ve_j}} f_i^{(1)}(\vx) \right\}_{i\only<2->{,j}}$
        \&$\dots$
        \&$\left\{\partial_{\only<1>{\vv}\only<2->{\ve_j}} f_i^{(L)}(\vx) \right\}_{i\only<2->{,j}}$
        \\
        $\left\{\partial^2_{\only<1>{\vv}\only<2->{\ve_j}} f_i^{(0)}(\vx) = 0 \right\}_{i\only<2->{,j}}$
        \&$\left\{\partial^2_{\only<1>{\vv}\only<2->{\ve_j}} f_i^{(1)}(\vx) \right\}_{i\only<2->{,j}}$
        \&$\dots$
        \&$\left\{\partial^2_{\only<1>{\vv}\only<2->{\ve_j}} f_i^{(L)}(\vx) \right\}_{i\only<2->{,j}}$
        \&\node [align=center] {\textcolor{orange}{\textbf{Na\"ive:}}\\ \textcolor{orange}{$\sum_j \partial^2_{\ve_j} f_i^{(L)}(\vx)$}};
        \\
        \only<3->{
          $\left\{\Delta f_i^{(0)}(\vx) = 0\right\}_i$
          \& $\left\{\Delta f_i^{(1)}(\vx) \right\}_i$
          \&$\dots$
          \& $\left\{\Delta f_i^{(2)}(\vx) \right\}_i$
          \& \node [align=center] {\textcolor{orange}{\textbf{Forward}}\\ \textcolor{orange}{\textbf{Laplacian}}};
        }
        \\
      };
    \end{tikzpicture}
  \end{figure}
\end{frame}
%%% Local Variables:
%%% mode: LaTeX
%%% TeX-master: "../pitch"
%%% End:


\begin{frame}[allowframebreaks]
  \frametitle{References}

  {\footnotesize
    \bibliographystyle{plainnat}
    \bibliography{../paper/references}
  }
\end{frame}


\end{document}

%%% Local Variables:
%%% mode: latex
%%% TeX-master: t
%%% End:

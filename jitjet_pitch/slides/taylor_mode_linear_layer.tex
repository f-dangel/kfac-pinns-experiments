\begin{frame}
  \frametitle{Taylor-mode for Linear Layers}

  \uncover<4->{
    \vspace{3ex}
    \ribbon[\paperwidth][black][VectorPink]{
      \centering
      \textbf{Linear layers propagate Taylor coefficients by applying their weight.}
      \uncover<5->{
        \\[2ex]
        \textbf{Computing any differential operator yields a net with linear weight sharing layers.}
      }}
  }

  \vspace{1.5ex}

  Let layer $l$ be a linear layer with weight $\mW$ (no bias for simplicity):
  \uncover<5->{
    \\
    We can stack the processed vectors into a matrix:
  }
  \only<1-4>{
    \begin{align*}
      \vz^{(l)}
      &= \mW\vz^{(l-1)}
      &\text{(function value)}
      \\
      \uncover<2->{
      \partial_{\vv}z_i^{(l)}
      &= \mW \partial_{\vv}\vz_i^{(l-1)}
      &\text{(directional slope)}
        }
      \\
      \uncover<3->{
      \partial^2_{\vv}z_i^{(l)} &= \mW \partial_{\vv}^2\vz_i^{(l-1)}
      &\text{(directional curvature)}
        }
      \\
      \uncover<4->{
      &\vdots
      \\
      \partial^K_{\vv}z_i^{(l)} &= \mW \partial_{\vv}^K\vz_i^{(l-1)}
                                  }
    \end{align*}
  }
  \only<5->{
    \begin{align*}
      \mZ^{(l)}
      &= \mW \mZ^{(l-1)}
        \intertext{with}
        \mZ^{(l)}
      &=
        \begin{pmatrix}
          \vz^{(l-1)} & \partial_{\vv}z_1^{(l-1)} & \partial_{\vv}z_2^{(l-1)} & \dots & \partial_{\vv}^2 z_1^{(l-1)} & \partial_{\vv}^2 z_d^{(l-1)} & \dots
        \end{pmatrix}
    \end{align*}
  }
\end{frame}
%%% Local Variables:
%%% mode: LaTeX
%%% TeX-master: "../pitch"
%%% End:

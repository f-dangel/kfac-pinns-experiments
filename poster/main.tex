\documentclass[a0paper]{tikzposter}
% Force a 16 by 9 ratio, 33.1 is same height as a0
\geometry{paperwidth=58.844in,paperheight=33.1in}

\usepackage{postertheme/vector_institute/vector_institute}
\usepackage{amsmath}

% \useblockstyle{Envelope}
% \usebackgroundstyle{Empty}
\usetitlestyle{Filled}

\title{
  \fontsize{115}{60}\selectfont
  \bf Kronecker-Factored Approximate Curvature
  for Physics-Informed Neural Networks}
\author{\Huge
  Felix Dangel*, Johannes M\"uller*, Marius Zeinhofer*
}
\institute{
  \LARGE
  Vector Institute (Canada), RWTH Aachen University (Germany), Simula Research Laboratory (Norway)
}


\begin{document}
% ==============================================================================
% HEADER & FOOTER

\backgroundgradient % Adds the background features
\maketitle
\headerlogo % Adds Vector logo to header
\posterfooter{Poster footer with additional information} % Footer

\hspace{-35cm}
\begin{columns}
  \centering
  \begin{column}{1.85}
    \centering \ribbon{\centering\fontsize{105}{80}\selectfont\textcolor{white}{\bf We generalize KFAC to PINN losses (which contain differential operators) \\
        using Taylor-mode autodiff and the concept of linear weight sharing layers.}}
  \end{column}
\end{columns}
\hspace{2cm}

\begin{columns}
  \column{0.9}
  \block{Ingredient 1: PINNs}{
    \begin{minipage}{0.63\linewidth}
    Explain what PINNs are.

    Explain that they are hard to train with first-order methods.
    Second-order methods work, but do not scale well to larger nets.
    \end{minipage}
    \hfill
    \begin{minipage}{0.33\linewidth}
      \centering
      Figure showing a loss curve

      \includegraphics[width=\linewidth]{example-image-a}
    \end{minipage}

    \begin{minipage}{0.3\linewidth}
      Picture showing the solution at the beginning
      \centering
    \end{minipage}
    \hfill
    \begin{minipage}{0.3\linewidth}
      \centering
      Picture showing the known solution
    \end{minipage}
    \hfill
    \begin{minipage}{0.3\linewidth}
      \centering
      Picture showing the learned solution
    \end{minipage}
  }

  \column{0.9}
  \block{Ingredient 3: Taylor mode}{
  }
\end{columns}

\begin{columns}
  \column{0.9}
  \block{Ingredient 2: KFAC}{
    Explain what Kronecker-factored approximate curvature is.

    Say that it is a scale-able curvature approximation that is cheap to invert.
    Say that it has recently been generalized to linear layers with weight sharing.
    \begin{equation*}
      a^2 + b^2 = c^2
    \end{equation*}
  }
  \column{0.9}
  \block{Putting everything together + practical evaluation: KFAC for PINNs}{
    \begin{minipage}[t]{0.3\linewidth}
      \centering
      \includegraphics[width=\linewidth]{example-image-a}
      Poisson equation
    \end{minipage}
    \hfill
    \begin{minipage}[t]{0.3\linewidth}
      \centering
      \includegraphics[width=\linewidth]{example-image-a}
      Heat equation
    \end{minipage}
    \hfill
    \begin{minipage}[t]{0.3\linewidth}
      \centering
      \includegraphics[width=\linewidth]{example-image-a}
      log-Fokker-Planck equation
    \end{minipage}
  }
\end{columns}

\end{document}
%%% Local Variables:
%%% mode: LaTeX
%%% TeX-master: t
%%% End:

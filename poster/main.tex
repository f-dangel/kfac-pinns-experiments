\documentclass[a0paper]{tikzposter}
% Force a 16 by 9 ratio, 33.1 is same height as a0
\geometry{paperwidth=58.844in,paperheight=33.1in}

\makeatletter
\def\input@path{
  {../} % repository root
  {../latex-utilities} % LaTeX utilities
}
\makeatother

\usepackage{postertheme/vector_institute/vector_institute}

% \usepackage{amsmath}
\usepackage{amsmath,amssymb,mathtools,bm}
\input{paper/preamble/goodfellow}
% math in sans serif
\usepackage{sfmath}

\usetikzlibrary{arrows.meta}

% \useblockstyle{Envelope}
% \usebackgroundstyle{Empty}
\usetitlestyle{Filled}

\title{
  \fontsize{115}{60}\selectfont
  \bf Kronecker-Factored Approximate Curvature
  for Physics-Informed Neural Networks}
\author{\Huge
  Felix Dangel*, Johannes M\"uller*, Marius Zeinhofer*
}
\institute{
  \LARGE
  Vector Institute (Canada), RWTH Aachen University (Germany), ETH Z\"urich (Switzerland)
}


\begin{document}
% ==============================================================================
% HEADER & FOOTER

\backgroundgradient % Adds the background features
\maketitle
\headerlogo % Adds Vector logo to header
\posterfooter{
  % Poster footer with additional information
} % Footer

\hspace{-35cm}
\begin{columns}
  \centering
  \begin{column}{1.85}
    \centering \ribbon{
      \begin{minipage}[b]{0.9\linewidth}
        \centering\fontsize{105}{80}\selectfont\textcolor{white}{\bf
          We develop KFAC for loss functions with differential operators.
          \\[0.5ex]
          For training PINNs, our optimizer consistently outperforms SGD/Adam.
        }
      \end{minipage}
      \begin{minipage}[t]{0.05\linewidth}
        \centering
        \includegraphics[width=\linewidth]{figures/qrcode.png}
      \end{minipage}
    }
  \end{column}
\end{columns}
\hspace{2cm}

\begin{columns}
  \column{0.9}
  \block{Background: What Are PINNs?}{
    \begin{Large}
      \begin{center}
        \textcolor{VectorBlue}{\textbf{Main idea: Train a neural network to satisfy a PDE $\to$ loss contains the PDE's differential operator.}}
      \end{center}
    \end{Large}

    \vspace{1ex}

    \begin{minipage}[t]{0.4\linewidth}
      \begin{itemize}
      \item \textbf{Goal:} Learn PDE solution $u(\vx)$
        \begin{align*}
          \gL u(\vx) &= f(\vx) \qquad  \vx \in \Omega
          \\
          u(\vx) &= g(\vx) \qquad  \vx \in \partial\Omega
        \end{align*}

      \item \textbf{Example:} 2d Poisson equation
        \begin{align*}
          - \gL = \frac{\partial^2}{\partial x_1^2} + \frac{\partial^2}{\partial x_2^2}
        \end{align*}
      \end{itemize}

    \end{minipage}
    \hfill
    \begin{minipage}[t]{0.6\linewidth}
      \begin{itemize}
      \item \textbf{Data:} Sample $\vx_n \sim \Omega$, $\vx_n^{\text{b}} \sim \partial \Omega$

      \item \textbf{Ansatz:} Neural net $u_{\vtheta}(\vx)$
        \begin{align*}
          L(\vtheta)
          =
          \underbrace{
          \frac{1}{2N_{\Omega}} \sum_{n=1}^{N_{\Omega}}
          \left(
          \gL u_{\vtheta}(\vx_n) - f(\vx_n)
          \right)^2
          }_{\substack{
          L_{\Omega}(\vtheta)
          \\[1ex] \textbf{\large satisfy interior condition}
          }}
          +
          \underbrace{
          \frac{1}{2N_{\partial\Omega}} \sum_{n=1}^{N_{\partial\Omega}}
          \left(
          u_{\vtheta}(\vx_n^{\text{b}}) - g(\vx_n^{\text{b}})
          \right)^2
          }_{\substack{
          L_{\partial\Omega}(\vtheta)
          \\[1ex] \textbf{\large satisfy boundary condition}
          }}
        \end{align*}
      \end{itemize}
    \end{minipage}

    \vspace{1ex}

    \begin{minipage}{0.33\linewidth}
      \centering
      % [trim={left bottom right top},clip]
      \includegraphics[scale=2,trim={0.9cm 0.8cm 6.7cm 1cm}, clip]{../kfac_pinns_exp/exp42_visualize_solutions/visualize_solution/SGD/poisson_2d_sin_product_mlp-tanh-64_SGD_step0000000.pdf}

      \textbf{Untrained}
    \end{minipage}
    \hfill
    \begin{minipage}{0.33\linewidth}
      \centering
      % [trim={left bottom right top},clip]
      \includegraphics[scale=2,trim={0.9cm 0.8cm 6.7cm 1cm}, clip]{../kfac_pinns_exp/exp42_visualize_solutions/visualize_solution/SGD/poisson_2d_sin_product_mlp-tanh-64_SGD_step0003299.pdf}

      \textbf{Trained}
    \end{minipage}
    \hfill
    \begin{minipage}{0.33\linewidth}
      \centering
      % [trim={left bottom right top},clip]
      \includegraphics[scale=2,trim={7.3cm 0.8cm 0.1cm 1cm}, clip]{../kfac_pinns_exp/exp42_visualize_solutions/visualize_solution/SGD/poisson_2d_sin_product_mlp-tanh-64_SGD_step0000000.pdf}

      \textbf{True solution}
    \end{minipage}
  }

  \column{0.9}
  \block{Contribution: How Do We Derive KFAC for PINN Losses?}{
    \begin{center}
      \begin{Large}
        \textcolor{VectorBlue}{\textbf{We show that computing differential operators with Taylor-mode autodiff yields networks with linear weight sharing layers}
          \\[0.5ex]
          $\mathbf{\to}$ \textbf{This allows us to apply the existing definition of KFAC for linear weight sharing layers.}}
      \end{Large}
    \end{center}

    \begin{minipage}{0.3\linewidth}

      \vspace{2ex}

      \begin{itemize}
      \item \textbf{Goal:} Approximate Gauss-Newton matrix
        \begin{align*}
          \mG(\mW) = \mG_{\Omega}(\mW) + \mG_{\partial \Omega}(\mW)
        \end{align*}
        for each linear layer with weight $\mW$
        \begin{align*}
          \mG_{\Omega}(\mW)
          &\approx
            \mA_{\Omega} \otimes \mB_{\Omega}
          \\
          \mG_{\partial\Omega}(\mW)
          &\approx
            \mA_{\partial\Omega} \otimes \mB_{\partial\Omega}
        \end{align*}

        \vspace{0.7ex}

      \item \textbf{Contribution:} Make computation of $L(\vtheta)$ explicit
        \begin{itemize}
        \item Linear layer in boundary loss
          \begin{align*}
            \vz \mapsto \mW \vz \qquad \textbf{($\vz$ is a vector)}
          \end{align*}
        \item Linear layer in interior loss
          \begin{align*}
            \mZ \mapsto \mW \mZ \qquad \textbf{($\mZ$ is a matrix)}
          \end{align*}
        \end{itemize}
      \end{itemize}

      \vspace{2ex}

      \textbf{The computation reduces to a neural net with linear weight sharing layers.
        We can just apply the existing KFAC definition from Eschenhagen (NeurIPS 2023).}
    \end{minipage}
    \hfill
    \begin{minipage}{0.68\linewidth}
      \centering
      \hspace{-13.25ex}\input{../paper/figures/computation_graph_styles.tex}
\colorlet{maincolor}{VectorPink}
\colorlet{secondcolor}{VectorOrange}
\colorlet{thirdcolor}{VectorTeal}
\colorlet{fourthcolor}{white}
\colorlet{fifthcolor}{red}
\begin{tikzpicture}
  % arrange nodes in a matrix
  \matrix [%
  row sep=2.5ex,%
  column sep=3.5ex,%
  ampersand replacement=\&,% in order to put this inside of a scalebox
  ]{%
    % neural network parameters
    \&
    \node [paramNode] (param-1) {$\mW^{(1)}$};
    \&
    \node [dotsNode] (param-2) {$\dots$};
    \&
    \node [paramNode] (param-3) {$\mW^{(i-1)}$};
    \&
    \node [paramNode] (param-4) {$\mW^{(i)}$};
    \&
    \node [dotsNode] (param-5) {$\dots$};
    \&
    \node [paramNode] (param-6) {$\mW^{(L)}$};
    \\
    % forward pass
    \node [inputNode] (forward-0) {$\vx$};
    \&
    \node [forwardNode] (forward-1) {$\vz^{(1)}$};
    \&
    \node [dotsNode] (forward-2) {$\dots$};
    \&
    \node [forwardNode] (forward-3) {$\vz^{(i-1)}$};
    \&
    \node [forwardNode] (forward-4) {$\vz^{(i)}$};
    \&
    \node [dotsNode] (forward-5) {$\dots$};
    \&
    \node [forwardNode] (forward-6) {$u(\vx)$};
    \\
  };
  % dependency arrows
  \foreach \i in {1,...,6} {
    \draw [-Latex, line width=4pt, VectorBlue] (param-\i) to (forward-\i);
  }
  \foreach \i in {0,...,5} {
    \pgfmathsetmacro{\j}{int(\i+1)}
    \draw [-Latex, line width=3pt] (forward-\i) to (forward-\j);
  }
\end{tikzpicture}
%%% Local Variables:
%%% mode: latex
%%% TeX-master: "../main"
%%% End:


      \textbf{Boundary loss compute graph}

      \vspace{2ex}

      \input{../paper/figures/computation_graph_styles.tex}
\colorlet{maincolor}{VectorPink}
\colorlet{secondcolor}{VectorOrange}
\colorlet{thirdcolor}{VectorTeal}
\colorlet{fourthcolor}{white}
\colorlet{fifthcolor}{red}
\begin{tikzpicture}
  % arrange nodes in a matrix
  \matrix [%
  row sep=2.5ex,%
  column sep=3.5ex,%
  ampersand replacement=\&,% in order to put this inside of a scalebox
  ]{%
    % neural network parameters
    \&
    \node [paramNode] (param-1) {$\mW^{(1)}$};
    \&
    \node [dotsNode] (param-2) {$\dots$};
    \&
    \node [paramNode] (param-3) {$\mW^{(i-1)}$};
    \&
    \node [paramNode] (param-4) {$\mW^{(i)}$};
    \&
    \node [dotsNode] (param-5) {$\dots$};
    \&
    \node [paramNode] (param-6) {$\mW^{(L)}$};
    \\
    % forward pass
    \node [inputNode] (forward-0) {$\vx$};
    \&
    \node [forwardNode] (forward-1) {$\vz^{(1)}$};
    \&
    \node [dotsNode] (forward-2) {$\dots$};
    \&
    \node [forwardNode] (forward-3) {$\vz^{(i-1)}$};
    \&
    \node [forwardNode] (forward-4) {$\vz^{(i)}$};
    \&
    \node [dotsNode] (forward-5) {$\dots$};
    \&
    \node [forwardNode] (forward-6) {$u(\vx)$};
    \\
    % gradients
    \node [inputNode] (gradient-0) {$\mI_d$};
    \&
    \node [gradientNode] (gradient-1) {$\partial_{\vx}\vz^{(1)}$};
    \&
    \node [dotsNode] (gradient-2) {$\dots$};
    \&
    \node [gradientNode] (gradient-3) {$\partial_{\vx}\vz^{(i-1)}$};
    \&
    \node [gradientNode] (gradient-4) {$\partial_{\vx}\vz^{(i)}$};
    \&
    \node [dotsNode] (gradient-5) {$\dots$};
    \&
    \node [gradientNode] (gradient-6) {$\partial_{\vx}u(\vx)$};
    \\
    % Hessians
    \node [inputNode] (hessian-0) {$\vzero_{d\times d}$};
    \&
    \node [hessianNode] (hessian-1) {$\partial^2_{\vx} \vz^{(1)}$};
    \&
    \node [dotsNode] (hessian-2) {$\dots$};
    \&
    \node [hessianNode] (hessian-3) {$\partial^2_{\vx}\vz^{(i-1)}$};
    \&
    \node [hessianNode] (hessian-4) {$\partial^2_{\vx}\vz^{(i)}$};
    \&
    \node [dotsNode] (hessian-5) {$\dots$};
    \&
    \node [hessianNode] (hessian-6) {$\partial^2_{\vx}u(\vx)$};
    \&
    \node [hessianNode] (hessian-7) {$\gL u(\vx)$};
    \\
  };
  % dependency arrows
  \foreach \i in {1,...,6} {
    \draw [-Latex, line width=4pt, VectorBlue] (param-\i) to (forward-\i);
  }
  \foreach \i in {0,...,5} {
    \pgfmathsetmacro{\j}{int(\i+1)}
    \draw [-Latex, line width=3pt] (forward-\i) to (forward-\j);
    \draw [-Latex, line width=3pt] (gradient-\i) to (gradient-\j);
    \draw [-Latex, line width=3pt] (hessian-\i) to (hessian-\j);
  }
  \foreach \i in {1,...,6} {
    \draw [-Latex, line width=4pt, VectorBlue, out=215, in=135] (param-\i) to (gradient-\i);
    \draw [-Latex, line width=4pt, VectorBlue, out=215, in=135] (param-\i) to (hessian-\i);
  }
  \draw [-Latex, line width=3pt] (hessian-6) to (hessian-7);
  \draw [-Latex, line width=3pt, out=315, in=135] (gradient-6) to (hessian-7);
  \draw [-Latex, line width=3pt, out=315, in=90] (forward-6) to (hessian-7);
\end{tikzpicture}
%%% Local Variables:
%%% mode: latex
%%% TeX-master: "../main"
%%% End:


      \textbf{Interior loss compute graph} (simplified)

    \end{minipage}
  }
\end{columns}

\begin{columns}
  \column{0.9}
  \block{Motivation: PINNs Are Hard to Train, Second-order Methods Can Help}{
    \begin{center}
      \begin{Large}
        \textcolor{VectorBlue}{\textbf{Natural gradient methods beat first-order methods on small problems\dots
            but do not scale well to larger nets $\mathbf{\to}$ our KFAC scales.}}
      \end{Large}
    \end{center}

    \hfill
    \begin{minipage}{0.08\linewidth}
      \textbf{Small net}
      \\
      \textbf{($\mathbf{D = 257}$)}
    \end{minipage}
    \begin{minipage}{0.38\linewidth}
      \centering
      % [trim={left bottom right top},clip]
      \includegraphics[width=0.85\linewidth, trim={0 0 0 0.3cm},clip]{../presentation/figures/poisson2d-02.pdf}
    \end{minipage}
    \hfill
    \begin{minipage}{0.38\linewidth}
      \centering
      % [trim={left bottom right top},clip]
      \includegraphics[width=0.85\linewidth, trim={0 0.08cm 0 0.3cm},clip]{figures/poisson2d-medium.pdf}
    \end{minipage}
    \begin{minipage}{0.08\linewidth}
      \textbf{Medium net}
      \\
      \textbf{($\mathbf{D = 9873}$)}
    \end{minipage}
    \hfill
  }
  \column{0.9}
  \block{Evaluation: Our KFAC Optimizer Outperforms First-order Methods and Scales Well}{
    \begin{minipage}[t]{0.31\linewidth}
      \centering
      \textbf{9+1d log-Fokker-Planck equation, $\mathbf{D \approx 10^5}$}

      \vspace{1ex}

      \includegraphics[width=\linewidth]{../kfac_pinns_exp/exp43_log_fokker_planck9d_isotropic_gaussian_random/l2_error_over_time.pdf}
    \end{minipage}
    \hfill
    \begin{minipage}[t]{0.28\linewidth}
      \centering
      \textbf{4+1d Heat equation, $\mathbf{D \approx 10^5}$}

      % [trim={left bottom right top},clip]
      \includegraphics[trim={9.5cm 0.5cm 0 0.3cm},clip, width=0.93\linewidth]{../kfac_pinns_exp/exp33_poisson_bayes_groupplot/l2_error_over_time.pdf}
    \end{minipage}
    \hfill
    \begin{minipage}[t]{0.28\linewidth}
      \centering
      \textbf{100d Poisson equation, $\mathbf{D\approx 10^6}$}

      % [trim={left bottom right top},clip]
      \includegraphics[trim={9.5cm 0.5cm 0 0.3cm},clip, width=0.9\linewidth]{../kfac_pinns_exp/exp30_heat4d_groupplot/l2_error_over_time.pdf}
    \end{minipage}
    \hfill

    \vspace{-0.5ex}
  }
\end{columns}

\end{document}
%%% Local Variables:
%%% mode: LaTeX
%%% TeX-master: t
%%% End:

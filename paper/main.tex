\documentclass{article}

% if you need to pass options to natbib, use, e.g.:
% \PassOptionsToPackage{numbers, compress}{natbib}
% before loading neurips_2023

\usepackage[preprint]{neurips_2023}

% to compile a preprint version, e.g., for submission to arXiv, add add the
% [preprint] option:
% \usepackage[preprint]{neurips_2023}

% to compile a camera-ready version, add the [final] option, e.g.:
% \usepackage[final]{neurips_2023}

% to avoid loading the natbib package, add option nonatbib:
% \usepackage[nonatbib]{neurips_2023}

\input{preamble/custom_early.tex}
\input{preamble/neurips_2023.tex}
\input{preamble/goodfellow.tex} % follow DL notation from the Goodfellow book
% ===================================================================
% MATH
% ===================================================================
\usepackage{nicefrac} % fractions that fit into inline text

% ===================================================================
% REFERENCES
% ===================================================================
\usepackage{cleveref} % automatically adds type of reference, MUST BE LOADED AFTER AMSMATH

%%% Local Variables:
%%% mode: latex
%%% TeX-master: "../main"
%%% End:

\newcommand{\papertitle}{%
  Kronecker-Factored Approximate Curvature for Physics-Informed Neural Networks
}
\title{\papertitle}

% The \author macro works with any number of authors. There are two commands
% used to separate the names and addresses of multiple authors: \And and \AND.
%
% Using \And between authors leaves it to LaTeX to determine where to break the
% lines. Using \AND forces a line break at that point. So, if LaTeX puts 3 of 4
% authors names on the first line, and the last on the second line, try using
% \AND instead of \And before the third author name.

\author{%
  Felix Dangel\thanks{Equal contribution}\\
  Vector Institute \\
  Toronto \\ Canada \\
  \texttt{fdangel@vectorinstitute.ai} \\
  \And
  Johannes M\"uller$^*$\\
  Chair of Mathematics of Information Processing \\
  RWTH Aachen University \\
  Aachen, Germany \\
  \texttt{mueller@mathc.rwth-aachen.de} \\
  \And
  Marius Zeinhofer$^*$\\
  Seminar for Applied Mathematics, ETH Z\"urich, \\
  Department of Nuclear Medicine, University Hospital Freiburg\\
  \texttt{marius.zeinhofer@uniklinik-freiburg.de}
}
%%% Local Variables:
%%% mode: latex
%%% TeX-master: "../main"
%%% End:


\begin{document}

\maketitle

\begin{abstract}
  PINNs are hard to train with first-order methods.
  To train PINNs efficiently, we need to take into account the geometry implied by the PDE operator.
  Existing methods that consider this geometry compute and invert the full Gramian.
  However, these ENGD-based methods do not scale well to architectures with many parameters due to the quadratic memory and cubic time complexity of storing and inverting the Gramian.
  The challenge to develop approximations to the Gramian is that it requires taking the parameter derivative of the PDE operator, which itself contains higher-order derivative.
  Here, we propose a Kronecker-factored approximation for the Gramian, which scales more favourably than existing approaches in terms of both time and memory, while showing similar performance downstream.
\end{abstract}


\section{Introduction}

PINNs are difficult to optimize.
\begin{itemize}
    \item PINNs receive ever growing amount of attention
    \item their failure to produce high accuracy solution when trained with variants of GD like Adam is well documented
    \item Quasi-Newton methods like L-BFGS yield improved but still not very high accuracy
    \item Other suggestions: reweighting of the loss, specialized sampling strategies, greedy training, reformulation as saddle point problem
    \item recently, a variant of NG based on the geometry of the specific energy / PDE was proposed; yields greatly improved accuracy over direct gradient-based optimizers and enjoys the nice property that it can be shown to mimic Newton's method in function space; for PINNs it can be seen as Gau\ss-Newton method in the space of residuals, for other problems as a generalized GN?
    \item whereas, this method was shown to be able to produce highly accurate approximations of the solution of the PDE it comes with a considerable iteration cost as it involves the solution of a linear system of the size of the number of parameters. Hence, this is only feasible for networks of small to moderate size when done naively.
    \item we build on the idea of Kronecker-factored approximations known as KFAC proposed in the context of supervised learning to provide an efficient implementation of energy natural gradients; 
    however, PDE terms appear in the Gramian, so existing implementations can not be used off the shelve 
\end{itemize}

\paragraph{Contribution:} \toodoo{Formulate our goal.}

\begin{itemize}
    \item we develop a Kronecker-factored approximation of the Gramian matrix appearing as a preconditioner in the energy natural gradient method; we call it KFAG 
    \item we provide an efficient implementation of KFAG; we experimentally show that it provides a good approximation of the true Gramian 
    \item We demonstrate that KFAG can be used to efficiently train networks of considerable size in a PINN style setting 
\end{itemize}

\paragraph{Related work:}
\begin{itemize}
\item OG KFAC papers: \cite{martens2015optimizing}, \cite{martens2018kroneckerfactored}, double check similarities to RNNs
\item KFAC for Rayleigh quotients:
\item PINNs: recent preconditioning papers
\end{itemize}

%%% Local Variables:
%%% mode: latex
%%% TeX-master: "../main"
%%% End:


\section{Background}

For simplicity, we present our approach for multi-layer perceptrons (MLPs) consisting of fully-connected and element-wise activation layers.
However, the generality of Taylor-mode automatic differentiation and KFAC for linear layers with weight sharing allow our KFAC to be applied to such layers (e.g.\,fully-connected, convolution, attention) in arbitrary neural network architectures.

\paragraph{Flattening \& Derivatives}
We vectorize matrices using the \emph{first-index-varies-fastest} convention, i.e.\,column-stacking (row index varies first, column index varies second) and denote the corresponding flattening operation by $\flatten$.
This allows to reduce derivatives of matrix- or tensor-valued objects back to the vector case by flattening a functions input and output domain before differentiation.
The Jacobian of a vector-to-vector function $\va \mapsto \vb(\va)$ has entries $[\jac_{\va}\vb]_{i,j} = \nicefrac{\partial \evb_i}{\partial \eva_j}$.
For a matrix-to-matrix function $\mA \mapsto \mB(\mA)$, the Jacobian is $\jac_{\mA} \mB = \jac_{\flatten \mA }\flatten\mB$.
A useful property of $\flatten$ is $\flatten(\mA\mX\mB) = (\mB^\top\otimes \mA)\flatten{\mX}$ for matrices $\mA, \mX, \mB$ which implies $\jac_\mX(\mA\mX\mB) = \mB^\top\otimes \mA$.

\paragraph{Sequential neural nets} Consider a \emph{sequential neural network} $u_{\vtheta} = f_{\vtheta^{(L)}} \circ f_{\vtheta^{(L-1)}} \circ \ldots \circ f_{\vtheta^{(1)}} $ of depth $L\in\mathbb N$. It consists of layers $f_{\vtheta^{(l)}}\colon \sR^{h^{(l-1)}}\to\sR^{h^{(l)}}$, $\vz^{(l-1)}\mapsto \vz^{(l)} = f_{\vtheta^{(l)}}(\vz^{(l-1)})$ with trainable parameters $\vtheta^{(l)} \in \sR^{p^{(l)}}$ that transform an input $\vz^{(0)} \coloneqq \vx \in \mathbb R^{d \coloneqq h^{(0)}}$ into a prediction $u_\vtheta(\vx) = \vz^{(L)} \in \sR^{h^{(L)}}$ via intermediate representations $\vz^{(l)} \in \sR^{h^{(l)}}$.
In the context of PINNs, we use networks with scalar outputs ($h^{(L)}=1$) and denote the concatenation of all parameters by $\vtheta = (\vtheta^{(1)\top}, \dots, \vtheta^{(L)\top})^{\top} \in \sR^P$.
A common choice is to alternate fully-connected and activation layers.
Linear layers map $\vz^{(l-1)} \mapsto \vz^{(l)} = \mW^{(l)} \vz^{(l-1)}$ using a weight matrix $\mW^{(l)} = \flatten^{-1}\vtheta^{(l)}  \in \sR^{h^{(l)} \times h^{(l-1)}}$ (bias terms can be added as an additional column and by appending a $1$ to the layer input).
Activation layers map $\vz^{(l-1)}\mapsto \vz^{(l)} \sigma(\vz^{(l-1)})$ element-wise for a (typically smooth) $\sigma\colon\mathbb R\to\mathbb R$.

\subsection{Energy Natural Gradients for Physics-Informed Neural Networks}
Let us consider a domain $\Omega\subseteq\mathbb R^d$ and the partial differential equation
\begin{align*}\tag{PE}\label{eq:PE}
  -\mathcal{L} u & = f \quad \text{in }\Omega \\
  u & = g \quad \text{on }\partial\Omega
\end{align*}
with square-integrable right hand side $f\in L^2(\Omega)$ and twice continuously differentiable boundary condition $g\in C^2(\Omega)\cap C(\overline{\Omega})$.
$\mathcal{L}$ denotes a differential operator, e.g.\,the Laplacian $\mathcal{L} u = \Delta u = \sum_{i=1}^d \partial_{\evx_i}^2 u$.
We parametrize $u$ using a neural network and train its parameters $\vtheta$ to minimize the loss
\begin{align}\label{eq:pinn-loss}
  L(\vtheta)
  &=
    \underbrace{\frac{1}{2N_\Omega} \sum_{n=1}^{N_\Omega} (\mathcal{L} u_\vtheta(\vx_i) + f(\vx_n))^2}_{\eqqcolon L_\Omega(\vtheta)} + \underbrace{\frac{1}{2N_{\partial\Omega}}\sum_{n=1}^{N_{\partial\Omega}} ( u_\vtheta(\vx^\text{b}_n) - g(\vx^\text{b}_n))^2}_{\eqqcolon L_{\partial\Omega}(\vtheta)},
\end{align}
with points $\{\vx_n \in \Omega \}_{n=1}^{N_\Omega}$ from the domain's interior, and points $\{\vx^\text{b}_n \in \partial\Omega \}_{n=1}^{N_{\partial\Omega}}$ on its boundary.\footnote{The second regression loss can also include other constraints like measurement data.}

%%% Local Variables:
%%% mode: latex
%%% TeX-master: "../main"
%%% End:

\subsection{Energy natural gradients}

%\begin{itemize}
%    \item recently, energy NGs have been proposed
%    \item one can show that they mimic Newtons method in function space
%    \item yield very good accuracy
%    \item
%\end{itemize}

Natural gradients have been introduced by~\citet{amari1998natural} and have shown great success in reinforcement learning, and other problems...
The general idea is to replace the vanilla GD update rule by a preconditioned version
    \[ \theta_{k+1} = \theta_k - \eta_k G(\theta_k)^{-1} \nabla L(\theta_k), \]
where $G(\theta)\in\mathbb R^{p\times p}$, $G(\theta)_{ij} \coloneqq g_{u_\theta}(\partial_{\theta_i} u_\theta, \partial_{\theta_j} u_\theta)$ is a matrix capturing the function space geometry of the problem and its parametrization.
Classically, $G$ is chosen as the Fisher information matrix,
%\begin{equation}
%    F_I(\theta)_{ij} = \sum_{x} \frac{\partial_{\theta_i}p_\theta(x)\partial_{\theta_j}p_\theta(x)}{p_\theta(x)} = \sum_{x} \partial_{\theta_i} \log p_\theta(x) \partial_{\theta_j} \log p_\theta(x),
%\end{equation}
in which case the Riemannian metric $g$ is given by the Fisher-Rao metric~\cite[text]{keylist}.
For a supervised learning problem with training data $(x_1, y_1), \dots, (x_N, y_N)$ the (empirical) Fisher-information matrix commonly used, has entries the entries 
\begin{equation}
  F(\theta)_{ij} = \sum_{n=1}^N \partial_{\theta_i} u_\theta(x_n)\partial_{\theta_j} u_\theta(x_n),
\end{equation}
see~\cite{amari2000natural,martens2020new}. 

In the PINN setting however, the models $u_\theta$ are functions rather than probability measures %$p_\theta$
and the loss involves PDE terms.
%In order to adjust the definition of
To capture the geometric properties of this specific problem we consider the following Fisher / Gramian matrix %to this problemThe energy natural gradient is for this example to use the Fischer/Gramian of the form
\begin{equation}
  F(\theta) = F_\Omega(\theta) + F_{\partial\Omega}(\theta) = \frac1{{N_\Omega}} \sum_{k=1}^{N_\Omega} \partial_{\theta_i} \Delta u_\theta(x_k) \partial_{\theta_j} \Delta u_\theta(x_k) + \frac1{{N_{\partial\Omega}}} \sum_{k=1}^{N_{\partial\Omega}} \partial_{\theta_i} u_\theta(x_k^b) \partial_{\theta_j} u_\theta (x_k^b).
\end{equation}
%where
%\begin{equation}\label{eq:FisherInterior}
%  F_\Omega(\theta)_{ij} = \frac1{{N_\Omega}} \sum_{k=1}^{N_\Omega} \partial_{\theta_i} \Delta u_\theta(x_k) \partial_{\theta_j} \Delta u_\theta(x_k)
  % = \frac1{{N_\Omega}} \sum_{i=1}^{N_\Omega} (\partial_{\theta_i} f_\theta) (\partial_{\theta_j} f_\theta ),
%\end{equation}
% where $f_\theta = \Delta u_\theta$.
%and
%\begin{equation}
%  F_{\partial\Omega}(\theta)_{ij} = \frac1{{N_{\partial\Omega}}} \sum_{k=1}^{N_{\partial\Omega}} \partial_{\theta_i} u_\theta(x_k^b) \partial_{\theta_j} u_\theta (x_k^b).
  % = \frac1{{N_\Omega}} \sum_{i=1}^{N_\Omega} (\partial_{\theta_i} f_\theta) (\partial_{\theta_j} f_\theta ),
%\end{equation}
It can be shown that the energy natural gradient method mimics Newton's method up to a projection onto the tangent space of the model and a discretization error that vanishes quadratically in the step size~\cite{muller2023achieving, }
Further, the Gramian matrix admits an interpretation as the Gauß-Newton matrix of the residual function, see~\cite{} and Appendix....

%%% Local Variables:
%%% mode: latex
%%% TeX-master: "../main"
%%% End:

\subsection{Kronecker-factored Approximate Curvature (KFAC)}

%\toodoo{F.D.  Get to the point more quickly. Make notation  consistent.}

We review the idea of Kronecker-factored Approximate Curvature (KFAC) which was introduced by~\citet{?, martens2015optimizing, ?} as an approximation of the per-layer Fisher information by a Kronecker product to speed up the computation of the natural gradient direction. 
%We review the general principle here. 
%This approach was introduced in the context of maximum likelihood estimation with a neural network and we reivew its basic principles here. 
%Later, we will expand this to Gramian matrices with PDE terms. 
\toodoo{We could also follow \cite{eschenhagen2023kroneckerfactored} for the notation} 


\paragraph{Block-diagonal approximation}
In the first step the Fisher-information matrix $\mF$ can be approximated by a block diagonal matrix with blocks corresponding to the Fisher-information matrices of the individual layers of the network, i.e., $\mF(\vtheta) \approx \operatorname{diag}(\mF(\vtheta^{(1)}), \dots, \mF(\vtheta^{(L)}))$. 
Note that a block diagonal linear system can be solved by solving the subsystems corresponding to the individual blocks, hence reducing the computational complexity. 

\paragraph{Kronecker-factored approximation of the blocks}
We now consider the individual blocks $\mF(\vtheta^{(l)})$, for which we examine $\jac_{\vtheta^{(l)}} u_{\vtheta}(x_n)$ for a fixed data point. 
The parameters $\vtheta^{(l)} = \mW^{(l)}$ of the $l$-th layer appears in the computational graph by $\vz^{(l)} = \mW^{(l)}\vz^{(l-1)}$ and note that by the vec-trick, we have $\jac_{\vtheta^{(l)}} \vz^{(l)} = \vz^{(l-1)} \otimes I$. 
By the chain rule, we have
\begin{align}
    \jac_{\vtheta^{(l)}} u_{\vtheta}(\vx_n) & = \jac_{\vtheta^{(l)}} \vz_n^{(l)} \jac_{\vz^{(l)}}  \vz^{(L)}_n = %(\vz^{(l-1)}_n\otimes I) \nabla_{\vz^{(l)}}  \vz^{(L)}_n = 
    \vz^{(l-1)}_n\otimes  \jac_{\vz^{(l)}}  \vz^{(L)}_n. 
\end{align}
%and hence 
%\begin{align}
%    \nabla_{\vtheta^{(l)}} u_{\vtheta}(x_n)\nabla_{\vtheta^{(l)}} u_{\vtheta}(x_n)^\top = (\vz^{(l-1)}_n\otimes \vz^{(l-1)}_n) \nabla_{\vz^{(l)}}  \vz^{(L)}_n\nabla_{\vz^{(l)}}  \vz^{(L)}_n^\top.  
%\end{align}
Summing over the data points and using  $\sum_n \mA_n \otimes \mB_n \approx (\sum_n \mA_n) \otimes (\sum_n \mB_n)$ we obtain 
\begin{equation}
    \mF(\vtheta^{(l)}) %= \sum_{n=1}^N \nabla_{\vtheta^{(l)}} u_{\vtheta}(x_n)\nabla_{\vtheta^{(l)}} u_{\vtheta}(x_n)^\top 
    \approx \left(\sum_{n=1}^N \vz^{(l-1)}_n {\vz^{(l-1)}_n}^\top \right)\otimes \left(\sum_{n=1}^N\nabla_{\vz^{(l)}}  \vu_n\nabla_{\vz^{(l)}}  \vu_n^\top\right),
\end{equation}
see~\citep{eschenhagen2023kroneckerfactored}. 
Solving a Kronecker-factored linear system is only as expensive as solving two systems of the sizes of the two factors therefore greatly reducing the computational cost. \todo{correct?} 
%the inverse of a Kronecker-factored matrix is given by the Kronecker product of the individual inverses hence reducing the computational complexity of the corresponding linear system. 

\paragraph{Weight sharing}
\todo{add}

\clearpage

Assume we have drawn a data set $\smash{\sD = \left\{ (\vx_n, \vy_n) \right\}_{n=1}^N}$ with $\smash{(\vx_n, \vy_n) \stackrel{\text{i.i.d}}{\sim} p_{\text{data}}(\vx, \vy)}$.
We want to approximate the data-generating process through $p_{\vtheta}(\vx, \vy)$ by modelling a likelihood $p_{\vtheta}(\vy \mid \vx)$ for the labels with a neural network, that is we use $p_{\vtheta}(\vx, \vy) = p_{\text{data}}(\vx) p_{\vtheta}(\vy | \vx)$ and maximize $KL(p_{\text{data}} || p_{\vtheta})$.
Since $p_{\text{data}}$ is not accessible, one replaces $p_{\text{data}}(\vx)$ and $p_{\text{data}}(\vy \mid \vx)$ with their empirical distributions implied by $\sD$.
This yields the objective \cite[see][Section 4]{martens2020new}
\begin{align*}
  \frac{1}{N} \sum_{n=1}^N -\log p_{\vtheta}(\vy_n \mid \vx_n)
\end{align*}
which corresponds to the empirical risk $\frac{1}{N} \sum_{n=1}^N \ell(\vx_n, \vy_n, \vtheta)$ with a negative log-likelihood loss function, such as square or softmax cross-entropy loss.
The likelihood modelled by the neural network is of the form $p_{\vtheta}(\vy_n
\mid \vx_n) = r(\vy_n \mid f_{\vtheta}(\vx))$. The Fisher of our modelled
probability $p_{\vtheta}(\vx, \vy)$ is
\begin{align*}
  \mF(\vtheta)
  &=
    \E_{(\vx, \vy) \sim p_{\vtheta}(\vx,\vy)}
    \left[
    \grad{\vtheta} \log p_{\vtheta}(\vx, \vy)
    (\grad{\vtheta} \log p_{\vtheta}(\vx, \vy))^{\top}
    \right]
  \\
  &=
    \E_{p_{\text{data}}(\vx)}
    \E_{p_{\vtheta}(\vy \mid \vx)}
    \underbrace{
    \left[
    \grad{\vtheta} \log p_{\vtheta}(\vy \mid \vx)
    (\grad{\vtheta} \log p_{\vtheta}(\vy \mid \vx))^{\top}
    \right]
    }_{\coloneqq \mF_{\vy \mid \vx}(\vtheta)}\,.
  \\
  &\approx
    \frac{1}{N} \sum_{n=1}^N
    \E_{p_{\vtheta}(\vy \mid \vx_n)}
    \left[
    \grad{\vtheta} \log p_{\vtheta}(\vy \mid \vx_n)
    (\grad{\vtheta} \log p_{\vtheta}(\vy \mid \vx_n))^{\top}
    \right]
\end{align*}

\paragraph{One datum, no weight sharing:}
Let's start with maximum likelihood estimation with a single data point $(\vx, \vy)$.
Consider a linear layer inside a neural network which maps some vector-valued hidden feature of $\vx$, $\va \in \sR^{D_{\text{in}}}$ to a vector-valued output $\vz \in \sR^{D_{\text{out}}}$ via $\vz = \mW \va$.
$\vz$ is then further processed and used to compute the negative log-likelihood loss $\ell(\vx, \vy, \mW) = - \log p(\vy \mid \vx, \mW)$.
For this single-usage layer, the weigh matrix's Fisher is exactly Kroneckerfactored, $\mF(\mW) = \va \va^{\top} \otimes \E_{\hat{\vy} \sim p(\vy \mid \vx, \mW)}\left[ \vg \vg^{\top} \right]$ where $\vg = \grad{\vz} \ell(\vx, \hat{\vy}, \mW)$.
By applying the chain rule at the layer's output, the Kronecker structure emerges from the output-parameter Jacobian $\jac_{\mW}\vz = \va^{\top} \otimes \mI$.
In practise, we will use one sample from the model's likelihood to estimate the expectation, $\mF(\mW) \approx \vz \vz^{\top} \otimes \vg \vg^{\top}$.

% explain how batch axes are treated
\paragraph{Multiple data, no weight sharing} In the presence of multiple data points, the sum over per-datum Kronecker products is further approximated as a Kronecker product of sums over data points:
\begin{align*}
  \mF(\mW)
  &=
    \frac{1}{N}
    \sum_{n=1}^N
    \va_n \va_n^{\top} \otimes \E_{\hat{\vy}_n \sim p(\vy_n \mid \vx_n, \mW)}\left[ \vg_n \vg_n^{\top} \right]
  \\
  &\approx
    \left(
    \frac{1}{N}
    \sum_{n=1}^N
    \va_n \va_n^{\top}
    \right)
    \otimes
    \left(
    \sum_{n=1}^N
    \E_{\hat{\vy}_n \sim p(\vy_n \mid \vx_n, \mW)}\left[ \vg_n \vg_n^{\top} \right]
    \right)
  \\
  &\approx
    \left(
    \frac{1}{N}
    \sum_{n=1}^N
    \va_n \va_n^{\top}
    \right)
    \otimes
    \left(
    \sum_{n=1}^N
    \vg_n \vg_n^{\top}
    \right)
\end{align*}

% expand approximation treats the shared axis like a batch axis
\paragraph{One datum, weight sharing} Now consider a layer whose weight is applied onto \emph{multiple} vectors.
This concept is known as weight sharing.
This could be a linear layer with matrix-valued inputs like in attention, a convolution layer whose kernel is shared between patches of the input, or weights that are used multiple times throughout the computation graph (e.g.\, weight tying).
This means the layer will not process a single vector $\va$, but a sequence of vectors $\left\{ \va_1, \dots, \va_S \right\}$ where $S$ denotes weight sharing number.
We can column-stack these vectors into a matrix $\mA \in \sR^{D_{\text{in}}\times S}$, likewise for the linear layer's outputs $\vz = \mW \mA \in \sR^{D_{\text{out}}\times S}$ and activation gradients $\mG \in \sR^{D_{\text{out}} \times S}$.
The output-weight Jacobian of a weight-sharing layer is $\jac_{\mW} \mZ = \mA^{\top} \otimes \mI$ \cite[see e.g.][]{dangel2020modular} and the Fisher does not simplify into a Kronecker product without further approximations.
As described in \citet{eschenhagen2023kroneckerfactored}, there are two possible Kronecker approximations for this setup.
We will focus on the \emph{expand} approximation, which yields the Kronecker approximation for convolutional layers proposed by~\citet{grosse2016kroneckerfactored}.
It treats the shared axis like a batch axis,
$\mF(\mW) \approx \nicefrac{1}{S} \sum_{s=1}^S \va_s \va_s^{\top} \otimes \sum_{s=1}^S \vg_s \vg_s^{\top}$ where $\vg_s = \grad{\vz_s} \ell(\vx, \hat{\vy}, \mW)$.
We can express this in matrix notation as $\mF(\mW) \approx \nicefrac{1}{S} \mA \mA^{\top} \otimes \mG \mG^{\top}$.


%%% Local Variables:
%%% mode: latex
%%% TeX-master: "../main"
%%% End:


\section{Kronecker-factored Approximate Gramians for PINNs}

We restrict ourselves to feed-forward sequential NNs where only the parameters of the linear layers are trainable.
To derive a Kronecker-factored approximation of the Gramian, we first describe how a layer's parameter enters the Laplacian's computation in~\Cref{sec:laplacian-computation-graph}.
This allows for expressing the exact Gramian as a sum over Kronecker-structured terms stemming from the parameter's direct children in the compute graph, see~\Cref{sec:kronecker-structure-gramian}.

\paragraph{Jacobian \& Hessian} The flattening notation allows to reduce derivatives of matrix/tensor-valued objects back to the matrix case.
Consider the Jacobian $\jac_{\va}\vb$ of a vector $\vb$ w.r.t.\,a vector $\va$.
It collects all partial derivatives as $[\mJ_{\va}\vb]_{i,j} = \nicefrac{\partial [\vb]_i}{\partial [\va]_j}$.
For the Jacobian $\jac_{\mA}\mB$ of a matrix $\mB$ w.r.t.\,a matrix $\mA$, we simply have $\jac_{\mA} \mB = \jac_{\flatten( \mA )}\flatten(\mB)$.
Likewise, the Hessian $\gradsquared{\va}b$ of a scalar $b$ w.r.t.\,a vector $\va$ collects the second-order partial derivatives according to $[\gradsquared{\va}b]_{i,j} = \nicefrac{\partial^2 b}{\partial [\va]_i \partial [\va]_j}$.
For the Hessian $\gradsquared{\mA} b$ of a scalar $b$ w.r.t.\,a matrix $\mA$, we simply have $\gradsquared{\mA} b = \gradsquared{\flatten(\mA)}b$.
We also have $\grad{\mA} b = \grad{\flatten(\mA)} b$ for the gradient of a scalar w.r.t.\,a matrix.

\subsection{Hessian Backpropagation}\label{sec:laplacian-computation-graph}
Here we derive the Laplacian of a feed-forward neural network with scalar output, that is $\Delta u_{\vtheta} \coloneqq \Tr(\gradsquared{\vx} u_{\theta})$.
The goal is to make the dependence of the Laplacian w.r.t.\,a weight $\mW$ in one layer of the network explicit.
Then we can write down the Jacobian $\jac_{\mW}(\Delta u_{\vtheta})$ which is required for the Fisher used by energy NGD.

We first lay out the notation for feedforward neural networks, then use the ideas of Hessian backpropagation \citep[HBP,][]{dangel2020modular} to derive a recursion for the Hessian $\gradsquared{\vx}u_{\vtheta}$.
The Laplacian follows by taking the trace of the latter.
Finally, we express the Laplacians gradient w.r.t.
a single layer's weight $\mW$, i.e.\,$\nicefrac{\partial \Delta u_{\vtheta}}{\partial \mW}$, in terms of $\mW$'s children in the compute graph.

% \subsection{Feed-forward Neural Networks}
\paragraph{Layer notation} Consider a sequential neural network $u_{\vtheta}$ with depth $L$ that consists of layers $f^{(i)}_{\vtheta^{(i)}}$ with trainable parameters $\vtheta^{(i)} \in \sR^{d^{(i)}}, i=1,\dots, L,$ that transform an input $\vx \in \sR^M$ into a prediction $u_{\vtheta}(\vx)\in \sR^C$ via intermediate representations $\vz^{(i)} \in \sR^{h^{(i)}}, i= 0, \dots, L$,
\begin{align}
  \begin{split}
    u_{\vtheta}
    &=
      f^{(L)}_{\vtheta^{(L)}} \circ f^{(L-1)}_{\vtheta^{(L-1)}} \circ \ldots \circ f^{(1)}_{\vtheta^{(1)}}
    \\
    f^{(i)}_{\vtheta^{(i)}}\colon \sR^{h^{(i-1)}}
    &\to
      \sR^{h^{(i)}}\,,
    \\
    \vz^{(i-1)}
    &\mapsto
      \vz^{(i)} = f^{(i)}_{\vtheta^{(i)}}(\vz^{(i-1)})
  \end{split}
\end{align}
where $\vz^{(0)} \coloneqq \vx$, $\vz^{(L)} \coloneqq \vu$, and $\vtheta = ({\vtheta^{(1)}}^{\top}, \dots, {\vtheta^{(L)}}^{\top})^{\top}$ is the concatenation of parameters over layers.
A parameter might be empty, e.g.\,if the layer is an activation layer.



% \subsection{Derivatives}

\paragraph{Flattening} Above, we assumed all quantities ($\vz^{(i)}, \vtheta^{(i)}$) to be vectors.
In case of tensor-valued quantities, we can first flatten them into vectors to reduce to the vector case.
Our index convention to vectorize will be first-varies-fastest, which means column-stacking for a matrix (row index varies first, column index varies second).
We denote the flattening operation by $\flatten(\cdot)$.
Very useful is the so called \emph{vec-trick} stating that
\begin{equation}\label{eq:vecTrick}
  \flatten(AXB) = (B^\top\otimes A)\flatten{X}
\end{equation}
for matrices $A, X, B$. In particular, this shows that $B^\top\otimes A$ is the  matrix representing the linear mapping $X\mapsto AXB$ and hence $J_X(AXB) = B^\top\otimes A$.

\paragraph{Jacobian \& Hessian} The flattening notation allows to reduce derivatives of matrix/tensor-valued objects back to the matrix case.
Consider the Jacobian $\jac_{\va}\vb$ of a vector $\vb$ w.r.t.\,a vector $\va$.
It collects all partial derivatives as $[\mJ_{\va}\vb]_{i,j} = \nicefrac{\partial [\vb]_i}{\partial [\va]_j}$.
For the Jacobian $\jac_{\mA}\mB$ of a matrix $\mB$ w.r.t.\,a matrix $\mA$, we simply have $\jac_{\mA} \mB = \jac_{\flatten( \mA )}\flatten(\mB)$.
Likewise, the Hessian $\gradsquared{\va}b$ of a scalar $b$ w.r.t.\,a vector $\va$ collects the second-order partial derivatives according to $[\gradsquared{\va}b]_{i,j} = \nicefrac{\partial^2 b}{\partial [\va]_i \partial [\va]_j}$.
For the Hessian $\gradsquared{\mA} b$ of a scalar $b$ w.r.t.\,a matrix $\mA$, we simply have $\gradsquared{\mA} b = \gradsquared{\flatten(\mA)}b$.
We also have $\grad{\mA} b = \grad{\flatten(\mA)} b$ for the gradient of a scalar w.r.t.\,a matrix.

\subsection{Computing the input Hessian via backpropagation}

Gradient backpropagation describes a recursive procedure to compute gradients by backpropagating a signal via vector-Jacobian products (VJPs).
A similar procedure can be derived to compute Hessians w.r.t.\,nodes in a graph ($\vz^{(i)}$ or $\vtheta^{(i)}$).
We call this recursive procedure Hessian backpropagation~\citep{dangel2020modular}.

In the following, we set $C = 1$, that is the neural network produces a scalar $u$.

\paragraph{Gradient backpropagation} As a warm-up, let's recall how to compute the gradient $\grad{\vtheta}u =
(\grad{\vtheta^{(1)}} \dots \grad{\vtheta^{(L)}})$. We start by setting $\grad{u}
u = 1$, then backpropagate the error via VJPs,
\begin{align}\label{eq:gradient-backpropagation}
  \begin{split}
    \grad{\vz^{(i-1)}}u
    &=
      \left( \jac_{\vz^{(i-1)}} \vz^{(i)} \right)^{\top} \grad{\vz^{(i)}}u\,,
    \\
    \grad{\vtheta^{(i)}}u
    &=
      \left( \jac_{\vtheta^{(i)}} \vz^{(i)} \right)^{\top} \grad{\vz^{(i)}}u\,
  \end{split}
\end{align}
for $i = L, \dots, 1$, and initialization $\grad{\vz^{(L)}}u = \grad{u}u = 1$ of the recursion.
This yields the gradients of $u$ w.r.t.\,all intermediate representations and parameters.

\paragraph{Hessian backpropagation}
Recall the Hessian chain rule
\begin{equation*}
  % D^2(f\circ g)(x) = D^2f(g(x))(\nabla g(x), \nabla g(x)) +
  \nabla^2 (f\circ g) = (J g)^\top\cdot \nabla^2 f(g) \cdot Jg + \sum_{k} (\nabla f)_k \cdot \nabla^2 g_k,
\end{equation*}
where $g_i$ dentotes the individual components of $g$, see~\cite{skorski2019chain}.
The recursion for computing Hessians of $u$
w.r.t.\,intermediate representations and parameters starts by initializing the
recursion with $\gradsquared{\vz^{(L)}}u = \gradsquared{u} u = 0$, then
backpropagates according to
\begin{align}\label{eq:hessian-backpropagation}
  \begin{split}
    \gradsquared{\vz^{(i-1)}}u
    &=
      \left( \jac_{\vz^{(i-1)}} \vz^{(i)} \right)^{\top}
      \gradsquared{\vz^{(i)}}u
      \left( \jac_{\vz^{(i-1)}} \vz^{(i)} \right)
      +
      \sum_{k=1}^{h^{(i)}}
      \left(
      \gradsquared{\vz^{(i-1)}} [\vz^{(i)}]_k
      \right)
      [\grad{\vz^{(i)}} u]_k\,,
    \\
    \gradsquared{\vtheta^{(i)}}u
    &=
      \left( \jac_{\vtheta^{(i)}} \vz^{(i)} \right)^{\top}
      \gradsquared{\vz^{(i)}}u
      \left( \jac_{\vtheta^{(i)}} \vz^{(i)} \right)
      +
      \sum_{k=1}^{h^{(i)}}
      \left(
      \gradsquared{\vtheta^{(i)}} [\vz^{(i)}]_k
      \right)
      [\grad{\vz^{(i)}} u]_k
  \end{split}
\end{align}
for $i = L, \dots, 1$.
The first term takes the incoming Hessian (w.r.t.\,a layer's output) and sandwiches it between the layer's Jacobian.
It can be seen as backpropagating curvature from downstream layers.
The second term adds in curvature introduced by the current layer.
It is only non-zero if the layer is non-linear.
For linear layers, convolutional layers, and ReLU layers, it is zero.

\begin{figure}[t]
  \centering
  \includegraphics[width=0.6\linewidth]{figures/HBP_graph.png}
  \caption{Dependencies during gradient and Hessian backpropagation when computing the Hessian $\gradsquared{\vx}u_{\vtheta}$.}\label{fig:hbp-dependencies}
  \label{fig:hbp-dependencies}
\end{figure}

Following the procedure of \Cref{eq:hessian-backpropagation} yields the
per-layer parameter and feature Hessians $\gradsquared{\vz^{(i)}}u,
\gradsquared{\vtheta^{(i)}}u$. In \Cref{fig:hbp-dependencies} we depict the dependencies of
intermediate gradients and Hessians for computing $\gradsquared{\vx}u_{\vtheta}$:
\begin{itemize}
\item $\grad{\vz^{(i-1)}}u$ depends on $\grad{\vz^{(i)}}u$ due to the recursion in \Cref{eq:gradient-backpropagation}, and on $\vz^{(i-1)}, \vtheta^{(i)}$ due to the Jacobian $\mJ_{\vz^{(i-1)}}\vz^{(i)}$ in the gradient backpropagation \Cref{eq:gradient-backpropagation}.

\item $\gradsquared{\vz^{(i-1)}}u$ depends on $\gradsquared{\vz^{(i)}}u$ and $\grad{\vz^{(i)}} u$ due to the recursion in \Cref{eq:hessian-backpropagation}, and on $\vz^{(i-1)}, \vtheta^{(i)}$ due to the Jacobian $\mJ_{\vz^{(i-1)}}\vz^{(i)}$ and Hessian $\gradsquared{\vz^{(i-1)}}[\vz^{(i)}]_k$ in the Hessian backpropagation \Cref{eq:gradient-backpropagation}.
\end{itemize}

% \subsection{Laplacian}

The Laplacian $\Delta u_{\vtheta}$ follows by taking the trace of
$\gradsquared{\vx}u_{\vtheta}$ from above, and is hence recursively defined.

\begin{figure}[t]
  \centering
  \includegraphics[width=0.6\linewidth]{figures/HBP_graph_weight.png}
  \caption{Direct children of the weight matrix of a single layer for the computation graph of the Laplacian.}\label{fig:laplacian-graph-weight}
\end{figure}

\paragraph{Hessian backpropagation through nonlinear layers}
We mostly consider nonlinear layers without trainable parameters and consist of a componentwise nonlinearity $z\mapsto \sigma(z)$ for some $\sigma\colon\mathbb R\to\mathbb R$ and we assume $\sigma$ to be twice continuously differentiable (I guess one ask for weaker things).
The Jacobian of such a nonlinear layer is given by
\begin{equation}
  \jac_{\vz^{(i-1)}}\vz^{(i)} = \operatorname{diag}(\sigma'(\vz^{(i-1)}))
\end{equation}
and the %Hessian acts as
% \begin{equation*}
%   (\nabla^2_{\vz^{(i-1)}\vz^{(i)})(u,v) = u\odot v\odot \sigma''(z). %u^\top \operatorname{diag}(\sigma''(\vz_{i-1}))v = (v\top)
% \end{equation*}
Hessian terms are given by
\begin{equation}
  \nabla^2_{\vz^{(i-1)}}[\vz^{(i)}]_k = \sigma''(\vz^{(i-1)}_k) e_k  e_k^\top.
\end{equation}
\toodoo{Someone should definitely double check. I could not find a mistake -- MZ}
With these two identities we can backpropogate the input Hessian through the non-trainable nonlinear layers and obtain
\begin{equation}
  \gradsquared{\vz^{(i-1)}}u
  =
  \left( \operatorname{diag}(\sigma'(\vz^{(i-1)})) \right)^{\top}
  \gradsquared{\vz^{(i)}}u
  \left( \operatorname{diag}(\sigma'(\vz^{(i-1)})) \right)
  +
  \sum_{k=1}^{h^{(i)}}
  \operatorname{diag}(\sigma''(\vz^{(i-1)_k})
  e_k e_k^\top
  [\grad{\vz^{(i)}} u]_k\,,
\end{equation}

\paragraph{Hessian backpropagation through a linear layer} To de-clutter the dependency graph of \Cref{fig:hbp-dependencies}, we will now consider the dependency of $\Delta u_{\vtheta}$ w.r.t.\,the weight of a single layer.
We assume this layer $i$ to be a linear layer with parameters $\mW^{(i)}$ such that $\vtheta^{(i)} = \flatten(\mW^{(i)})$,
\begin{align}
  \vz^{(i)} = \mW^{(i)} \vz^{(i-1)}\,.
\end{align}
For this layer, the second terms in \Cref{eq:hessian-backpropagation} disappears because the local Hessians are zero, that is $\gradsquared{\vz^{(i-1)}}[\vz^{(i)}]_k = \vzero$ and $\gradsquared{\mW^{(i)}}[\vz^{(i)}]_k = \vzero$.
Also, the Jacobians are $\jac_{\mW^{(i)}}\vz^{(i)} = {\vz^{(i-1)}}^{\top} \otimes \mI$ and $\jac_{\vz^{(i-1)}}\vz^{(i)} = \mW^{(i)}$ and hence only depends on one of the two layer inputs.
This simplifies the computation graph.
\Cref{fig:laplacian-graph-weight} shows the dependencies of $\mW^{(i)}$ on the
Laplacian, highlighting its three direct children,
\begin{align}\label{eq:spatialDerivatives}
  \begin{split}
    \vz^{(i)}
    &=
      \mW^{(i)} \vz^{(i-1)}
    \\
    \grad{\vz^{(i-1)}}u
    &=
      {\mW^{(i)}}^{\top}
      \left(
      \grad{\vz^{(i)}}u
      \right)
    \\
    \gradsquared{\vz^{(i-1)}}u
    &=
      {\mW^{(i)}}^{\top}
      \left(
      \gradsquared{\vz^{(i)}}u
      \right)
      \mW^{(i)}
  \end{split}
\end{align}

\subsection{Computing the parameter derivative of the input Laplacian}
Recall, that the entries of the Fisher are composed from parameter derivatives of the input Laplacian, see~\eqref{eq:FisherInterior}.
Here, we only consider architectures where only the linear layers possess trainable parameters and therefore we compute the parameter $\partial_{\theta} \Delta u_\theta$ of the input Laplacian.

We have now identified the direct children of $\mW^{(i)}$ in the Laplacian's compute graph.
This allows us to write down its derivative $\grad{\mW^{(i)}} \Delta u$, which---by the chain rule---is simply the accumulated backpropagated error from the direct children:
\begin{align}\label{eq:laplacian-gradient}
  \begin{split}
    \grad{\mW^{(i)}} \Delta u_{\vtheta}
    &=
      \sum_{\bullet \in \left\{ \vz^{(i)}, \grad{\vz^{(i-1)}}u, \gradsquared{\vz^{(i-1)}}u \right\}}
      \left(
      \jac_{\mW^{(i)}}\bullet
      \right)^{\top}
      \grad{\bullet}\Delta u
    \\
    &=
      \left(
      \jac_{\mW^{(i)}}\vz^{(i)}
      \right)^{\top}
      \grad{\vz^{(i)}}\Delta u
      +
      \left(
      \jac_{\mW^{(i)}}\grad{\vz^{(i-1)}}u
      \right)^{\top}
      \grad{\grad{\vz^{(i-1)}}u}\Delta u
      +
      \left(
      \jac_{\mW^{(i)}}\gradsquared{\vz^{(i-1)}}u
      \right)^{\top}
      \grad{\gradsquared{\vz^{(i-1)}}u}\Delta u\,.
  \end{split}
\end{align}

% \paragraph{The Fisher}
With the Laplacian's gradient \Cref{eq:laplacian-gradient}, we can write down the Fisher block (up to summation over the data) for $\mW^{(i)}$ as
\begin{align}\label{eq:fisher}
  \begin{split}
    \mF^{(i)}
    &=
      \left(
      \grad{\mW^{(i)}} \Delta u_{\vtheta}
      \right)
      \left(
      \grad{\mW^{(i)}} \Delta u_{\vtheta}
      \right)^{\top}
    \\
    &=
      \sum_{\textcolor{blue}{\bullet} \in \left\{ \vz^{(i)}, \grad{\vz^{(i-1)}}u, \gradsquared{\vz^{(i-1)}}u \right\}}
      \sum_{\textcolor{red}{\bullet} \in \left\{ \vz^{(i)}, \grad{\vz^{(i-1)}}u, \gradsquared{\vz^{(i-1)}}u \right\}}
      \left(
      \left(
      \jac_{\mW^{(i)}}\textcolor{blue}{\bullet}
      \right)^{\top}
      \grad{\textcolor{blue}{\bullet}}\Delta u
      \right)
      \left(
      \left(
      \jac_{\mW^{(i)}}\textcolor{red}{\bullet}
      \right)^{\top}
      \grad{\textcolor{red}{\bullet}}\Delta u
      \right)^{\top}
    \\
    &=
      \sum_{\textcolor{blue}{\bullet} \in \left\{ \vz^{(i)}, \grad{\vz^{(i-1)}}u, \gradsquared{\vz^{(i-1)}}u \right\}}
      \sum_{\textcolor{red}{\bullet} \in \left\{ \vz^{(i)}, \grad{\vz^{(i-1)}}u, \gradsquared{\vz^{(i-1)}}u \right\}}
      \underbrace{
      \left(
      \jac_{\mW^{(i)}}\textcolor{blue}{\bullet}
      \right)^{\top}
      \left[
      \left(
      \grad{\textcolor{blue}{\bullet}}\Delta u
      \right)
      \left(
      \grad{\textcolor{red}{\bullet}}\Delta u
      \right)^{\top}
      \right]
      \left(
      \jac_{\mW^{(i)}}\textcolor{red}{\bullet}
      \right)}_{
      \coloneqq \mF^{(i)}_{\textcolor{blue}{\bullet}, \textcolor{red}{\bullet}}
      }\,.
  \end{split}
\end{align}

The Fisher consists of nine different terms.
Eventually, we would like to approximate it with a single Kronecker product, like KFAC~\citep{martens2015optimizing}.
\toodoo{We could either work with the diagonal terms, or if all terms admit Kronecker representations we could also approximate this sum of Kroneckers by the Kronecker of the sums...?}
We begin by studying the diagonal terms $\mF_{\textcolor{blue}{\bullet}, \textcolor{blue}{\bullet}}^{(i)}$ and call them zeroth, first and second order diagonal terms for $\textcolor{blue}{\bullet} = \vz, \nabla u, \nabla^2 u$, respectively.
\toodoo{Other question: might it be possible to write the nine terms as a Kronecker product of a sum of 3 matrices with a sum of 3 matrices?}

\paragraph{Computing $\mJ_{\mW^{(i)}}\textcolor{blue}{\bullet}$}
Let us first compute the Jacobians $\mJ_{\mW^{(i)}}\textcolor{blue}{\bullet}$ for which we envoke~\eqref{eq:spatialDerivatives}.
We already know the Jacobian from the linear layer's forward pass,
\begin{subequations}\label{eq:fisher-jacobians}
  \begin{align}
    \jac_{\mW}\left( \mW \vx \right) = \vx^{\top} \otimes \mI\,.
  \end{align}
  The Jacobian from the gradient backpropagation is
  \begin{align}
    \jac_{\mW}\left( \mW^{\top} \vx \right) = \mI \otimes \vx^{\top}\,,
  \end{align}
  and the Jacobian from the Hessian backpropagation is
  \begin{align}\label{subeq:fisher-jacobians-hbp}
    \jac_{\mW}\left( \mW^{\top} \mX \mW \right)
    =
    \mI \otimes \mW^{\top}\mX
    +
    \mK \left(
    \mI
    \otimes
    \mW^{\top}\mX^{\top}
    \right)\,,
  \end{align}
\end{subequations}
where $\mK \in \sR^{\dim(\mZ) \times \dim(\mZ)}$ (denoting $\mZ := \mW^{\top}\mX \mW$) is a permutation matrix that, when multiplied onto a vector whose basis corresponds to that of the flattened output $\mZ$, modifies the order from first-varies-fastest into last-varies-fastest, i.e.
\begin{equation*}
  \mK \flatten(\mZ) = \flatten(\mZ^{\top})\,.
\end{equation*}
Re-introducing the layer indices (\eqref{eq:spatialDerivatives}), this yields
\begin{align}
  \begin{split}
    \jac_{\mW^{(i)}}\vz^{(i)}
    &=
      {\vz^{(i-1)}}^\top\otimes \mI
    \\
    \jac_{\mW^{(i)}}\grad{\vz^{(i-1)}}u
    &=
      \mI\otimes
      \grad{\vz^{(i)}}u
    \\
    \jac_{\mW^{(i)}}\gradsquared{\vz^{(i-1)}}u
    &=
      \mI \otimes
      \left[
      {\mW^{(i)}}^{\top}
      \left(
      \gradsquared{\vz^{(i)}}u
      \right)
      \right]
      +
      \mK
      \left(
      \mI \otimes
      \left[
      {\mW^{(i)}}^{\top}
      \left(
      \gradsquared{\vz^{(i)}}u
      \right)^{\top}
      \right]
      \right)
  \end{split}
\end{align}
We will now use symmetries in the objects used during Hessian backpropagation to simplify those equations further.
At a first glance, it looks like the Fisher consists of 16 terms, as there are 4 terms from the Jacobians in \Cref{eq:fisher-jacobians}.
However, we can simplify into 9 terms:

First, $\gradsquared{\vz^{(i)}}u$ is symmetric, that is
\begin{align*}
  \jac_{\mW^{(i)}}\left( {\mW^{(i)}}^{\top} \left( \gradsquared{\vz^{(i)}}u  \right)\mW^{(i)} \right)
  &=
    \mI \otimes
    \left[
    {\mW^{(i)}}^{\top} \left( \gradsquared{\vz^{(i)}}u  \right)
    \right]
    +
    \mK
    \left(
    \mI \otimes
    \left[
    {\mW^{(i)}}^{\top}
    \left(
    \gradsquared{\vz^{(i)}}u
    \right)
    \right]
    \right)
    \shortintertext{and the transposed Jacobian is}
  &\mI \otimes
    \left[
    \left( \gradsquared{\vz^{(i)}}u  \right) \mW^{(i)}
    \right]
    +
    \left(
    \mI \otimes
    \left[
    \left(
    \gradsquared{\vz^{(i)}}u
    \right)
    \mW^{(i)}
    \right]
    \right)
    \mK^{\top}\,.
\end{align*}
Second, we multiply the transpose Jacobian onto $\grad{\gradsquared{\vz^{(i-1)}}u}\Delta u$, which inherits symmetry from the Hessian, $[\grad{\gradsquared{\vz^{(i-1)}}u}\Delta u]_{j,k} = [\grad{\gradsquared{\vz^{(i-1)}}u}\Delta u]_{k,j}$.
Due to this symmetry, the action of $\mK$ (or $\mK^{\top}$) does not alter it,
\begin{align*}
  \mK^{\top}\left( \grad{\gradsquared{\vz^{(i-1)}}u}\Delta u \right) = \grad{\gradsquared{\vz^{(i-1)}}u}\Delta u\,.
\end{align*}
In other words, it does not matter how we flatten (first- or last-varies-fastest).
This simplifies the VJP to
\begin{align*}
  \left(
  \mI \otimes
  \left[
  \left( \gradsquared{\vz^{(i)}}u  \right) \mW^{(i)}
  \right]
  \right)
  \grad{\gradsquared{\vz^{(i-1)}}u}\Delta u
  +
  \left(
  \mI \otimes
  \left[
  \left(
  \gradsquared{\vz^{(i)}}u
  \right)
  \mW^{(i)}
  \right]
  \right)
  \mK^{\top}
  \grad{\gradsquared{\vz^{(i-1)}}u}\Delta u
  =
  2 \left(
  \mI \otimes
  \left[
  \left( \gradsquared{\vz^{(i)}}u  \right) \mW^{(i)}
  \right]
  \right)
  \grad{\gradsquared{\vz^{(i-1)}}u}\Delta u
  \,.
\end{align*}
We can now write down the gradient from \Cref{eq:laplacian-gradient}, whose
self-outer product forms the Gramian for a linear layer, as
\begin{align*}
  \begin{split}
    \grad{\mW^{(i)}} \Delta u_{\vtheta}
    &=
      \underbrace{
      \left(
      {\vz^{(i-1)}}^\top\otimes \mI
      \right)^{\top}
      \grad{\vz^{(i)}}\Delta u
      }_{(1)}
      +
      \underbrace{
      \left(
      \mI \otimes \grad{\vz^{(i)}}u
      \right)^{\top}
      \grad{\grad{\vz^{(i-1)}}u}\Delta u
      }_{(2)}
      +
      \underbrace{
      2
      \left(
      \mI \otimes
      \left[
      \left( \gradsquared{\vz^{(i)}}u \right) \mW^{(i)}
      \right]
      \right)
      \grad{\gradsquared{\vz^{(i-1)}}u}\Delta u
      }_{(3)}
      \,,
  \end{split}
\end{align*}
where (1) is the contribution from the forward pass, (2) is the contribution from the gradient backpropagation, and (3) is the contribution from the Hessian backpropagation.

\begin{figure}[t]
  \centering
  Full interior Gramian\\
  \includegraphics[width=0.43\linewidth]{../kfac_pinns_exp/exp04_gramian_contributions/fig/gram_full.png}

  \begin{tabular}{ccc}
    (\textcolor{blue}{forward}, \textcolor{red}{forward})
    &
      (\textcolor{blue}{forward}, \textcolor{red}{gradient})
    &
      (\textcolor{blue}{forward}, \textcolor{red}{Hessian})
    \\
    \includegraphics[width=0.22\linewidth]{../kfac_pinns_exp/exp04_gramian_contributions/fig/gram_output_output.png}
    &
      \includegraphics[width=0.22\linewidth]{../kfac_pinns_exp/exp04_gramian_contributions/fig/gram_output_grad_input.png}
    &
      \includegraphics[width=0.22\linewidth]{../kfac_pinns_exp/exp04_gramian_contributions/fig/gram_output_hess_input.png}
    \\
    (\textcolor{blue}{gradient}, \textcolor{red}{forward})
    &
      (\textcolor{blue}{gradient}, \textcolor{red}{gradient})
    &
      (\textcolor{blue}{gradient}, \textcolor{red}{Hessian})
    \\
    \includegraphics[width=0.22\linewidth]{../kfac_pinns_exp/exp04_gramian_contributions/fig/gram_grad_input_output.png}
    &
      \includegraphics[width=0.22\linewidth]{../kfac_pinns_exp/exp04_gramian_contributions/fig/gram_grad_input_grad_input.png}
    &
      \includegraphics[width=0.22\linewidth]{../kfac_pinns_exp/exp04_gramian_contributions/fig/gram_grad_input_hess_input.png}
    \\
    (\textcolor{blue}{Hessian}, \textcolor{red}{forward})
    &
      (\textcolor{blue}{Hessian}, \textcolor{red}{gradient})
    &
      (\textcolor{blue}{Hessian}, \textcolor{red}{Hessian})
    \\
    \includegraphics[width=0.22\linewidth]{../kfac_pinns_exp/exp04_gramian_contributions/fig/gram_hess_input_output.png}
    &
      \includegraphics[width=0.22\linewidth]{../kfac_pinns_exp/exp04_gramian_contributions/fig/gram_hess_input_grad_input.png}
    &
      \includegraphics[width=0.22\linewidth]{../kfac_pinns_exp/exp04_gramian_contributions/fig/gram_hess_input_hess_input.png}
  \end{tabular}
  \caption{Contributions $\mG_{\Omega,\textcolor{blue}{\bullet}, \textcolor{red}{\bullet}}$ to the Laplacian's Gramian $\mG_{\Omega}$ from different children in the computation graph on a synthetic toy problem.
    We use a $4 \to 3 \to 2 \to 1$ sigmoid-activated MLP and 10 randomly generated inputs. The contributions are highlighted as in \Cref{eq:fisher}.}\label{fig:gramian-contribution-children}
\end{figure}
%%% Local Variables:
%%% mode: latex
%%% TeX-master: "../main"
%%% End:


The Jacobians from \Cref{eq:fisher-jacobians} allow to express the Fisher in terms of Kronecker-structured expressions consisting of 9 terms in total.
For the off-diagonal terms, which stem from two different children, we combine the two terms that involve the same children into a symmetric term.
\Cref{fig:gramian-contribution-children} shows the resulting 6 terms.

\input{figures/gramian_sum_children.tex}

\Cref{fig:gramian-contribution-summed-children} shows an alternative comparison where we group the terms stemming from two identical and two different nodes (that is the diagonals and off-diagonals of \Cref{eq:fisher-jacobians}) into one matrix.
For the Gramian's block diagonal (leftmost panel), we can see that the contribution from identical nodes (center panel) is larger than that of different nodes (rightmost panel).
This is a first promising observation: Since we tackle the approximation of the Gramian's block diagonal, we can focus on the contributions from identical nodes.
These are also simpler to compute because they can be computed at one stage of backpropagation.

\toodoo{F.D.\,Write down the full expression for the Fisher, grouping the terms with identical and non-identical children.}

\toodoo{Propose ways to Kronecker-approximate the terms from identical children in to a single Kronecker product.}

\paragraph{Computing $\grad{\textcolor{blue}{\bullet}}\Delta u$}
The terms $\grad{\textcolor{blue}{\bullet}}\Delta u$ are automatically computed when computing the gradient of the loss via backdrop.

\paragraph{Zeroth order diagonal term}
Note that one of the nine terms is the term similar to the original KFAC paper, namely
when $\textcolor{blue}{\bullet} = \textcolor{red}{\bullet} = \vz^{(i)}$
(remember that $\jac_{\mW^{(i)}} \vz^{(i)} = \vz^{(i-1)} \otimes \mI$):
\begin{align}\label{eq:original-kfac}
  \begin{split}
    \mF^{(i)}_{\textcolor{blue}{\vz^{(i)}}, \textcolor{red}{\vz^{(i)}}}
    &=
      \left(
      \textcolor{blue}{\vz^{(i-1)} \otimes \mI}
      \right)
      \left[
      \left(
      \textcolor{blue}{\grad{\vz^{(i-1)}}\Delta u}
      \right)
      \left(
      \textcolor{red}{\grad{\vz^{(i-1)}}\Delta u}
      \right)^{\top}
      \right]
      \left(
      \textcolor{red}{{\vz^{(i-1)}}^{\top} \otimes \mI}
      \right)
    \\
    &=
      \textcolor{blue}{\vz^{(i-1)}}
      \textcolor{red}{{\vz^{(i-1)}}^{\top}}
      \otimes
      \left(
      \textcolor{blue}{\grad{\vz^{(i-1)}}\Delta u}
      \right)
      \left(
      \textcolor{red}{\grad{\vz^{(i-1)}}\Delta u}
      \right)^{\top}
    \\
    &\coloneqq \mA_{\text{KFAC}}^{(i)} \otimes \mB_{\text{KFAC}}^{(i)}\,.
  \end{split}
\end{align}
\toodoo{I think the subscript should be $\nabla_{\vz^{(i)}} \Delta u$ instead of $\nabla_{\vz^{(i-1)}} \Delta u$ or am I missing something? }
Note however, that in comparison to traditional KFAC, there are Laplacian terms.
\Cref{eq:original-kfac} illustrates that the Kronecker structure emerges from
the Jacobians. So we need to investigate these objects closer for the remaining
terms of \Cref{eq:fisher}.


\paragraph{First order diagonal term}

\paragraph{Second order diagonal term}


%%% Local Variables:
%%% mode: latex
%%% TeX-master: "../main"
%%% End:


\subsection{Kronecker Structure of the Gramian}\label{sec:kronecker-structure-gramian}

This subsection describes the Jacobians of the operations with the weight matrix in
a linear layer during the computation of the Laplacian and ends by providing an
equation for the Gramian block.

\subsection{Kronecker-factored Approximate Gramian (KFAG)}
This subsection describes the Kronecker approximation we propose for the per-layer Gramian and justify that the approximation is similar to the concepts used in other KFAC works.
We will rely on a similar intuition than \citet{eschenhagen2023kroneckerfactored} in the presence of weight sharing.
However, we will use a slightly different motivation to obtain the Kronecker structure in the presence of weight sharing.

\begin{table}
  \centering
  \caption{Overview of Kronecker approximations for matrix multiplication with a weight $\mW \in \sR^{D_1 \times D_2}$ in the presence and absence of weight sharing and transposition.
    Transposition changes to which Kronecker factor the input/grad output contributes.
    Weight sharing requires an additional approximation to obtain a single Kronecker factor.
  }\label{tab:kronecker-approximations-basic-operations}
  \resizebox{\linewidth}{!}{%
    \begin{tabular}{cccccc}
      \toprule
      (share,
      &
        Input
      &
        Output
      &
        Grad output
      &
        Jacobian
      &
        Gramian proxy
      \\
      \quad transp.)
      &
        $\tX$
      &
        $\tZ$
      &
        $\grad{\tZ}\ell$ % \coloneqq \grad{\flatten(\tZ)}\ell$
      &
        $\jac_{\mW}\tZ$ % \coloneqq \jac_{\flatten(\mW)}\flatten(\tZ)$
      &
        $(\jac_{\mW}\tZ)^{\top} \grad{\tZ}\ell (\grad{\tZ}\ell)^{\top} \jac_{\mW}\tZ
        \approx \mA \otimes \mB$
      \\
      \midrule
      (\xmark, \xmark)
      &
        $\vx \in \sR^{D_2}$
      &
        $\vz = \mW \vx \in \sR^{D_1}$
      &
        $\grad{\vz}\ell \in \sR^{D_1}$
      &
        $\vx^{\top} \otimes \mI_{D_1}$
      &
        $\vx \vx^{\top} \otimes \grad{\vz}\ell (\grad{\vz}\ell)^{\top}$
      \\
      (\xmark, \cmark)
      &
        $\vx \in \sR^{D_1}$
      &
        $\vz = \mW^{\top} \vx \in \sR^{D_2}$
      &
        $\grad{\vz}\ell \in \sR^{D_2}$
      &
        $\mI_{D_2} \otimes \vx^{\top}$
      &
        $\grad{\vz}\ell (\grad{\vz}\ell)^{\top} \otimes \vx \vx^{\top}$
      \\
      (\cmark, \xmark)
      &
        $\mX \in \sR^{D_2 \times S}$
      &
        $\mZ = \mW \mX \in \sR^{D_1 \times S}$
      &
        $\grad{\mZ}\ell \in \sR^{S D_1}$
      &
        $\mX^{\top} \otimes \mI_{D_1}$
      &
        $
        \mX \mX^{\top}
        \otimes
        \frac{1}{S}
        \flatten^{-1}(\grad{\mZ}\ell) {\flatten^{-1}(\grad{\mZ}\ell) }^{\top}
        $
      \\
      (\cmark, \cmark)
      &
        $\mX \in \sR^{D_1 \times S}$
      &
        $\mZ = \mW^{\top} \mX \in \sR^{D_2 \times S}$
      &
        $\grad{\mZ}\ell \in \sR^{S D_2}$
      &
        $\mK \left( \mI_{D_2} \otimes \mX^{\top} \right)$
      &
        $
        \frac{1}{S}
        \flatten^{-1}(\grad{\mZ}\ell) {\flatten^{-1}(\grad{\mZ}\ell)}^{\top}
        \otimes
        \mX \mX^{\top}
        $
      \\
      New
      &
        $\mX \in \mathrm{Sym}(D_1)$
      &
        $\mZ = \mW^{\top} \mX \mW \in \mathrm{Sym}(D_2)$
      &
        $\grad{\mZ}\ell \in \sR^{D_2^2}$
      &
        $\mI \otimes \mW^{\top} \mX + \mK \left( \mI \otimes \mW^{\top} \mX^{\top} \right)$
        &
          $\frac{4}{D_2} \flatten^{-1}(\grad{\mZ}\ell) {\flatten^{-1}(\grad{\mZ}\ell)}^{\top} \otimes \mX \mW \mW^{\top} \mX$
      \\
      \bottomrule
    \end{tabular}
  }
\end{table}


\paragraph{Some intuition:} Consider a weight matrix $\mW \in \sR^{D_1 \times
  D_2}$ inside a neural network that produces a loss $\ell$. We want to compute
the Gramian $\mF(\mW) = \grad{\mW}\ell (\grad{\mW}\ell)^{\top}$. Consider the following scenarios:
\begin{enumerate}
\item \textbf{(No sharing, no transpose)} The matrix is used to form the matrix vector product $\vz = \mW \vx \in \sR^{D_2}$ with an input vector $\vx \in \sR^{D_1}$.
  The Gramian is then
  \begin{align*}
    \mF(\mW) = \vx \vx^{\top} \otimes \grad{\vz}\ell (\grad{\vz}\ell)^{\top}\,.
  \end{align*}
  The input to the matrix multiply forms the first Kronecker factor, the gradient w.r.t.\,the matrix multiply's output forms the second Kronecker factor.

\item \textbf{(No sharing, transpose)} The matrix is used to form the transpose matrix vector product $\vz = \mW^{\top} \vx \in \sR^{D_1}$.
  The Gramian is then
  \begin{align*}
    \mF(\mW) = \grad{\vz}\ell (\grad{\vz}\ell)^{\top} \otimes \vx \vx^{\top}\,.
  \end{align*}
  The gradient w.r.t.\,the transpose matrix multiply's output forms the first Kronecker factor, the input to the matrix multiply forms the second Kronecker factor.
  I.e.
  the Gramian w.r.t.\,the transpose matrix is just the Gramian of the non-transposed matrix, but with swapped Kronecker factors.

\item \textbf{(Sharing, no transpose)} The matrix is used to form the matrix matrix product $\mZ = \mW \mX \in \sR^{D_2 \times S}$ with an input matrix $\mX \in \sR^{D_1 \times S}$ consisting of $S$ vector-valued columns which share the same weights.
  The Gramian is then
  \begin{align*}
    \mF(\mW) =
    \left(
    \mX
    \otimes
    \mI_{D_2}
    \right)
    \grad{\mZ}\ell (\grad{\mZ}\ell)^{\top}
    \left(
    \mX^{\top}
    \otimes
    \mI_{D_2}
    \right)
  \end{align*}
  Note that this does not simplify into a single Kronecker product.
  We need to apply another approximation.
  One way to obtain a single Kronecker factor is to seek an approximation for the central term such that $\grad{\mZ}\ell (\grad{\mZ}\ell)^{\top} \approx \mI \otimes \mU$ with $\mU \in \sR^{D_1 \times D_1}$.
  Let's look for the matrix that minimizes the reconstruction error
  $\left\lVert \grad{\mZ}\ell (\grad{\mZ}\ell)^{\top} - \mI \otimes \mU
  \right\rVert_{\text{F}}^2$. The solution to this is given by the average
  diagonal blocks of $\grad{\mZ}\ell (\grad{\mZ}\ell)^{\top}$. We can write it
  as
  \begin{align}
    \mU
    =
    \frac{1}{S}
    \flatten^{-1}
    \left(
    \grad{\mZ}\ell
    \right)
    \left[
    \flatten^{-1}
    \left(
    \grad{\mZ}\ell
    \right)
    \right]^{\top}
  \end{align}
  where $\flatten^{-1} \left(\grad{\mZ}\ell \right) \in \sR^{D_1 \times S}$.
  With this approximation, we can express the Gramian as a single Kronecker product:
  \begin{align*}
    \begin{split}
      \mF(\mW)
      &\approx
        \left(
        \mX
        \otimes
        \mI_{D_2}
        \right)
        \left(
        \mI
        \otimes
        \mU
        \right)
        \left(
        \mX^{\top}
        \otimes
        \mI_{D_2}
        \right)
      \\
      &=
        \mX \mX^{\top}
        \otimes
        \mU
      \\
      &=
        \mX \mX^{\top}
        \otimes
        \frac{1}{S}
        \flatten^{-1}
        \left(
        \grad{\mZ}\ell
        \right)
        \left[
        \flatten^{-1}
        \left(
        \grad{\mZ}\ell
        \right)
        \right]^{\top}
    \end{split}
  \end{align*}
  Again, note that the input to the matrix multiply forms the first Kronecker factor, the gradient w.r.t.\,the matrix multiply's output forms the second Kronecker factor.
  However, we needed additional approximations to obtain a single Kronecker product.

\item \textbf{(Sharing, transpose)} The matrix is used to form the transpose matrix matrix product $\mZ = \mW^{\top} \mX \in \sR^{D_1 \times S}$ with $\mX \in \sR^{D_2 \times S}$.
  The Gramian is then
  \begin{align*}
    \begin{split}
      \mF(\mW)
      &=
        \left(
        \mI_{D_1}
        \otimes
        \mX
        \right)
        \mK^{\top}
        \grad{\mZ}\ell (\grad{\mZ}\ell)^{\top}
        \mK
        \left(
        \mI_{D_1}
        \otimes
        \mX^{\top}
        \right)\,.
    \end{split}
  \end{align*}
  Again, this does not simplify into a single Kronecker product.
  So we are forced to make another approximation.
  Let's first take a closer look at $\mK^{\top} \grad{\mZ}\ell$.
  The application of $\mK^{\top}$ simply changes the flattening scheme of the vector.
  With the definition $\grad{\mZ^{\top}}\ell := \grad{\flatten(\mZ^{\top})} \ell$, we have that $\mK^{\top} \grad{\mZ}\ell = \grad{\mZ^{\top}}\ell$.
  Just like in the (sharing, no transpose) case from above, we will now look for an approximation of $\grad{\mZ^{\top}}\ell (\grad{\mZ^{\top}}\ell)^{\top} \approx \mU \otimes \mI$ with $\mU \in \sR^{D_1\times D_1}$.
  Again, we choose $\mU$ to minimize the reconstruction $\left\lVert \grad{\mZ^{\top}}\ell (\grad{\mZ^{\top}}\ell)^{\top} - \mU \otimes \mI \right\rVert_{\text{F}}^2$.
  We can transform this by applying $\mK$ from the left and from the right before evaluating the squared Frobenius norm (this only rearranges the elements and the Frobenius norm does not depend on the order).
  So we have $\left\lVert \mK \left( \grad{\mZ^{\top}}\ell (\grad{\mZ^{\top}}\ell)^{\top} - \mU \otimes \mI \right) \mK \right\rVert_{\text{F}}^2 = \left\lVert \grad{\mZ}\ell (\grad{\mZ}\ell)^{\top} - \mI \otimes \mU \right\rVert_{\text{F}}^2 $, which is the same objective from the (sharing, no transpose) case.
  Hence, its solution is $\mU = \nicefrac{1}{S} \flatten^{-1}
  \left(\grad{\mZ}\ell \right) \left[\flatten^{-1} \left(\grad{\mZ}\ell \right)
  \right]^{\top}$. With that, the single Kronecker product approximation of the
  Fisher is
  \begin{align*}
    \begin{split}
      \mF^{(i)}
      &\approx
        \left(
        \mI_{D_1}
        \otimes
        \mX
        \right)
        \left(
        \mU
        \otimes
        \mI_{S}
        \right)
        \left(
        \mI_{D_1}
        \otimes
        \mX^{\top}
        \right)
      \\
      &=
        \mU
        \otimes
        \mX \mX^{\top}
      \\
      &=
        \frac{1}{S}
        \flatten^{-1}
        \left(
        \grad{\mZ}\ell
        \right)
        \left[
        \flatten^{-1}
        \left(
        \grad{\mZ}\ell
        \right)
        \right]^{\top}
        \otimes
        \mX \mX^{\top}
    \end{split}
  \end{align*}
  The input to the transpose matrix multiply forms the first Kronecker factor, the gradient w.r.t.\,the transpose matrix multiply's output forms the second Kronecker factor.

\end{enumerate}

\paragraph{Single datum case} As a first step, let's write down only the diagonal terms of the Gramian, i.e.\,the terms caused by the same children and try to simplify each term into a Kronecker product of same dimension:
\begin{align}
  \begin{split}
    \mF^{(i)}
    &\approx
      \underbrace{
      \left(
      {\vz^{(i-1)}}^\top\otimes \mI
      \right)^{\top}
      \grad{\vz^{(i)}}\Delta u
      \left(
      \left(
      {\vz^{(i-1)}}^\top\otimes \mI
      \right)^{\top}
      \grad{\vz^{(i)}}\Delta u
      \right)^{\top}
      }_{(1, 1)}
    \\
    &\phantom{=}+
      \underbrace{
      \left(
      \mI \otimes \grad{\vz^{(i)}}u
      \right)^{\top}
      \grad{\grad{\vz^{(i-1)}}u}\Delta u
      \left(
      \left(
      \mI \otimes \grad{\vz^{(i)}}u
      \right)^{\top}
      \grad{\grad{\vz^{(i-1)}}u}\Delta u
      \right)^{\top}
      }_{(2, 2)}
    \\
    &\phantom{=}+
      \underbrace{
      2
      \left(
      \mI \otimes
      \left[
      \left( \gradsquared{\vz^{(i)}}u \right) \mW^{(i)}
      \right]
      \right)
      \grad{\gradsquared{\vz^{(i-1)}}u}\Delta u
      \left(
      2
      \left(
      \mI \otimes
      \left[
      \left( \gradsquared{\vz^{(i)}}u \right) \mW^{(i)}
      \right]
      \right)
      \grad{\gradsquared{\vz^{(i-1)}}u}\Delta u
      \right)^{\top}
      }_{(3, 3)}
      \,.
      \shortintertext{Without approximations, we can re-write this as}
    &=
      \vz^{(i-1)} {\vz^{(i-1)}}^\top
      \otimes
      \left(
      \grad{\vz^{(i)}}\Delta u
      \right)
      \left(
      \grad{\vz^{(i)}}\Delta u
      \right)^{\top}
    \\
    &\phantom{=}+
      \left(
      \grad{\vz^{(i)}}u
      \right)
      \left(
      \grad{\vz^{(i)}}u
      \right)^{\top}
      \otimes
      \left(
      \grad{\grad{\vz^{(i-1)}}u}\Delta u
      \right)
      \left(
      \grad{\grad{\vz^{(i-1)}}u}\Delta u
      \right)^{\top}
    \\
    &\phantom{=}+
      4
      \left(
      \mI \otimes
      \left[
      \left( \gradsquared{\vz^{(i)}}u \right) \mW^{(i)}
      \right]
      \right)
      \left[
      \left(
      \grad{\gradsquared{\vz^{(i-1)}}u}\Delta u
      \right)
      \left(
      \grad{\gradsquared{\vz^{(i-1)}}u}\Delta u
      \right)^{\top}
      \right]
      \left(
      \mI \otimes
      \left[
      {\mW^{(i)}}^{\top}
      \left( \gradsquared{\vz^{(i)}}u \right)
      \right]
      \right)
      \,.
      \intertext{The first two terms have the same Kronecker structure, $h^{(i-1)} \times h^{(i-1)} \otimes h^{(i)} \times h^{(i)}$.
      The third term does not simplify further, because the non-identity term in the Jacobians (outer terms) is a matrix, not a vector.
      One way to obtain the same Kronecker structure is to approximate the central term $ \left(\grad{\gradsquared{\vz^{(i-1)}}u}\Delta u \right) \left(\grad{\gradsquared{\vz^{(i-1)}}u}\Delta u \right)^{\top} \approx \mU \otimes \mI$ where $\mU \in \sR^{h^{(i-1)}\times h^{(i-1)}}$ (remember that $\grad{\gradsquared{\vz^{(i-1)}}}u \in \sR^{{h^{(i-1)}}^2}$.
      In the following, let $g := \grad{\gradsquared{\vz^{(i-1)}}}u$ for brevity.
      The `optimal' $\mU$ that minimizes the Frobenius norm $\left\lVert\mU \otimes \mI - \vg \vg^{\top} \right\rVert_{\text{F}}^2$ is the averaged block diagonal $\mU = \nicefrac{1}{h^{(i-1)}} \mG \mG^{\top}$ where $\mG = \flatten^{-1}(\vg) \in \sR^{h^{(i-1)} \times h^{(i-1)}}$.
      With this additional approximation, we can also express the third term as a Kronecker product:}
    &\approx
      \vz^{(i-1)} {\vz^{(i-1)}}^\top
      \otimes
      \left(
      \grad{\vz^{(i)}}\Delta u
      \right)
      \left(
      \grad{\vz^{(i)}}\Delta u
      \right)^{\top}
    \\
    &\phantom{=}+
      \left(
      \grad{\vz^{(i)}}u
      \right)
      \left(
      \grad{\vz^{(i)}}u
      \right)^{\top}
      \otimes
      \left(
      \grad{\grad{\vz^{(i-1)}}u}\Delta u
      \right)
      \left(
      \grad{\grad{\vz^{(i-1)}}u}\Delta u
      \right)^{\top}
    \\
    &\phantom{=}+
      \frac{4}{h^{(i-1)}}
      \left(
      \flatten^{-1}
      \left(
      \grad{\gradsquared{\vz^{(i-1)}}u}\Delta u
      \right)
      \right)
      \left(
      \flatten^{-1}
      \left(
      \grad{\gradsquared{\vz^{(i-1)}}u}\Delta u
      \right)
      \right)^{\top}
      \otimes
      \left( \gradsquared{\vz^{(i)}}u \right)
      \mW^{(i)}
      {\mW^{(i)}}^{\top}
      \left( \gradsquared{\vz^{(i)}}u \right)
      \,.
      \shortintertext{Let's make the final approximation from three Kronecker products into a single Kronecker product.
      In KFAC-style, we use a Kronecker-of-sums to approximate a sum-of-Kroneckers:}
    &\approx
      \mA^{(i)} \otimes \mB^{(i)}
  \end{split}
\end{align}
with
\begin{subequations}
  \label{eq:gram-kronecker-approximations-unbatched}
  \begin{align}
    \begin{split}
      \mA^{(i)}
      &=
        \vz^{(i-1)} {\vz^{(i-1)}}^\top
        +
        \left(
        \grad{\vz^{(i)}}u
        \right)
        \left(
        \grad{\vz^{(i)}}u
        \right)^{\top}
      \\
      &\phantom{=}+
        \frac{4}{h^{(i-1)}}
        \left(
        \flatten^{-1}
        \left(
        \grad{\gradsquared{\vz^{(i-1)}}u}\Delta u
        \right)
        \right)
        \left(
        \flatten^{-1}
        \left(
        \grad{\gradsquared{\vz^{(i-1)}}u}\Delta u
        \right)
        \right)^{\top}\,,
    \end{split}
    \\
    \begin{split}
      \mB^{(i)}
      &=
        \left(
        \grad{\vz^{(i)}}\Delta u
        \right)
        \left(
        \grad{\vz^{(i)}}\Delta u
        \right)^{\top}
        +
        \left(
        \grad{\grad{\vz^{(i-1)}}u}\Delta u
        \right)
        \left(
        \grad{\grad{\vz^{(i-1)}}u}\Delta u
        \right)^{\top}
      \\
      &\phantom{=}+
        \left( \gradsquared{\vz^{(i)}}u \right)
        \mW^{(i)}
        {\mW^{(i)}}^{\top}
        \left( \gradsquared{\vz^{(i)}}u \right)\,.
    \end{split}
  \end{align}
\end{subequations}

\paragraph{With batching} In the presence of multiple data points, we need to approximate the Gramian $\mF^{(i)} \approx \nicefrac{1}{N} \sum_{n=1}^N \mA_n^{(i)} \otimes \mB_n^{(i)}$ further (the subscripts $_n$ denote the computation on datum $n$).
We can just do that in the same way as the original KFAC paper, which gives us
\begin{align}\label{eq:gram-kronecker-approximations-batched}
  \mF^{(i)}
  \approx
  \left(
  \frac{1}{N}\sum_{n=1}^N \mA_n^{(i)}
  \right)
  \otimes
  \left(
  \sum_{n=1}^N \mB_n^{(i)}
  \right)
\end{align}
\Cref{eq:gram-kronecker-approximations-unbatched,eq:gram-kronecker-approximations-batched} are our proposed approximation for the Gramian.
%%% Local Variables:
%%% mode: latex
%%% TeX-master: "../main"
%%% End:


\section{Generalization to other PDEs}

\input{sections/generalization.tex}

\section{Experiments}

\subsection{Setup}
\begin{itemize}
    \item damping: try both constant and adaptive damping 
    \item try without line search with learning rate schedule 
    \item use weights and biases for large scale experiments 
    \item next steps: modular code 
\end{itemize}

\subsection{A two-dimensional Poisson equation}



\subsection{asdf}

We want to show the following things:
\begin{itemize}
\item We can safely discard the Gramian's off-diagonal blocks without harming
  training performance. This reduces the Gramian's size, but still imposes
  strong constraints on scalability.

\item Our proposed Kronecker approximation works roughly as well as the
  full/block diagonal Gramian, while being much cheaper to compute, store, and
  invert.

\item Thanks to the Kronecker approximation of the Gramian, we can scale to larger neural networks where the other methods either do not work (storing the Gramian is prohibitibely expensive) or become quite slow (matrix-free linear system solve via Gramian-vector products).
\end{itemize}

Todos:
\begin{itemize}
\item concrete example ground truth: 2d Poisson on unit square with sine target
\end{itemize}

Ideas:
\begin{itemize}
\item try out different approximations
  \begin{itemize}
  \item Ground truth
  \item Block diagonal exact
  \item Diagonal
  \item Block diagonal with different approximations
  \end{itemize}
\end{itemize}

\section{Conclusion}

\paragraph{Limitations:}
\begin{itemize}
\item Only works for sequential networks
\item Hessian backpropagation requires memory quadratic in the intermediate features size and therefore becomes impractical for large intermediates like in CNNs.
\item Our implementation is likely manual and relatively hard to automate with the current graph inspection tools offered by ML frameworks.
\end{itemize}

\begin{ack} % automatically suppressed in anonymized submission
  The authors thank Runa Eschenhagen for insightful discussions on the relation to KFAC for weight sharing layers.
\end{ack}

\bibliography{references}
\bibliographystyle{icml2024.bst}

\clearpage
\appendix

% Use different numbering in appendix to make it clear that a
% section/table/algorithm is not in the main text
\renewcommand\thefigure{\thesection\arabic{figure}}
\renewcommand\thetable{\thesection\arabic{table}}
\renewcommand{\theequation}{\thesection\arabic{equation}}

% modified title header from main page, code extracted from NeurIPS template
\makeatletter
\vbox{%
  \hsize\textwidth
  \linewidth\hsize
  \vskip 0.1in
  \@toptitlebar
  \centering
  {\LARGE\bf \@title (Supplementary Material)\par}
  \@bottomtitlebar
  \vskip 0.3in \@minus 0.1in
}
\makeatother

% APPENDIX TOC
\startcontents[sections]
\printcontents[sections]{l}{1}{\setcounter{tocdepth}{2}}
\vspace{2em}
%%% Local Variables:
%%% mode: latex
%%% TeX-master: "../main"
%%% End:

\input{sections/laplacian_shallow_net.tex}

\section{Inverting the Sum of Two Kronecker Matrices}
PINNs use two losses: a boundary and an interior loss.
When designing a Kronecker-factored curvature approximation, we have two choices:
\begin{enumerate}
\item We can approximate each loss individually with a Kronecker product, i.e.
  \begin{align*}
    \mK \coloneqq \mA_1 \otimes \mA_2 + \mB_1 \otimes \mB_2
  \end{align*}
  where $\mA_{1,2}, \mB_{1,2}$ are invertible and positive definite.

\item We can further approximate the above through a single Kronecker product, e.g.\,by summing,
  \begin{align*}
    \mK' \coloneqq (\mA_1 + \mB_1) \otimes (\mA_2 + \mB_2) \approx \mK\,.
  \end{align*}
\end{enumerate}
Eventually, we want to use their inverses.
For $\mK'$, this is straightforward to do.
For $\mK$, we need to invert the sum of two Kronecker products, which is more challenging.
We can proceed as follows to invert $\mK$:
\begin{enumerate}
\item Simultaneously diagonalize $(\mA_i, \mB_i)$ by solving the generalized eigenvalue problem
  \begin{align*}
    \mA_i \mV_i = \mB \mV_i \mLambda_i
  \end{align*}
  where $\mV_i$'s columns are the (generalized) eigenvectors and $\mLambda_i$ is a diagonal matrix containing the (generalized) eigenvalues.

\item Consider the expression
  \begin{align*}
    &\left(
      \mA_1 \otimes \mA_2
      +
      \mB_1 \otimes \mB_2
      \right)
      \left(
      \mV_1 \otimes \mV_2
      \right)
    \\
    &=
      \mB_1 \mV_1 \mLambda_1 \otimes \mB_2 \mV_2 \mLambda_2
      +
      \left( \mB_1 \otimes \mB_2 \right)
      \left( \mV_1 \otimes \mV_2 \right)
    \\
    &=
      \left( \mB_1 \otimes \mB_2 \right)
      \left( \mV_1 \otimes \mV_2 \right)
      \left(
      \mLambda_1 \otimes \mLambda_2 + \mI \otimes \mI
      \right)\,.
  \end{align*}
  Rearrange into
  \begin{align*}
    \left(
    \mA_1 \otimes \mA_2
    +
    \mB_1 \otimes \mB_2
    \right)
    =
    \left( \mB_1 \otimes \mB_2 \right)
    \left( \mV_1 \otimes \mV_2 \right)
    \left(
    \mLambda_1 \otimes \mLambda_2 + \mI \otimes \mI
    \right)
    \left( \mV_1^{-1} \otimes \mV_2^{-1} \right)
    \,.
  \end{align*}

  \item Take the inverse to obtain
    \begin{align*}
      \mK^{-1}
      =
      \left( \mV_1 \otimes \mV_2 \right)
      \left(
      \mLambda_1 \otimes \mLambda_2 + \mI \otimes \mI
      \right)^{-1}
      \left( \mV_1^{-1}\mB_1^{-1} \otimes \mV_2^{-1}\mB_2^{-1} \right)
      \,.
    \end{align*}
    Notice that $\mLambda_1 \otimes \mLambda_2 + \mI \otimes \mI$ is diagonal, and therefore easy to invert.
\end{enumerate}

%%% Local Variables:
%%% mode: latex
%%% TeX-master: "../main"
%%% End:


\end{document}

%%% Local Variables:
%%% mode: latex
%%% TeX-master: t
%%% End:

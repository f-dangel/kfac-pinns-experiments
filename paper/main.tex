\documentclass{article}

% if you need to pass options to natbib, use, e.g.:
% \PassOptionsToPackage{numbers, compress}{natbib}
% before loading neurips_2023

\usepackage{neurips_2024}
\usepackage{comment}
\usepackage[most]{tcolorbox}

\definecolor{kfac}{rgb}{0.55294118, 0.17647059, 0.22352941}  %

% to compile a preprint version, e.g., for submission to arXiv, add add the
% [preprint] option:
% \usepackage[preprint]{neurips_2023}

% to compile a camera-ready version, add the [final] option, e.g.:
% \usepackage[final]{neurips_2023}

% to avoid loading the natbib package, add option nonatbib:
% \usepackage[nonatbib]{neurips_2023}

\input{preamble/custom_early.tex}
\input{preamble/neurips_2023.tex}
\input{preamble/goodfellow.tex} % follow DL notation from the Goodfellow book
% ===================================================================
% MATH
% ===================================================================
\usepackage{nicefrac} % fractions that fit into inline text

% ===================================================================
% REFERENCES
% ===================================================================
\usepackage{cleveref} % automatically adds type of reference, MUST BE LOADED AFTER AMSMATH

%%% Local Variables:
%%% mode: latex
%%% TeX-master: "../main"
%%% End:

\newcommand{\papertitle}{%
  Kronecker-Factored Approximate Curvature for Physics-Informed Neural Networks
}
\title{\papertitle}

% The \author macro works with any number of authors. There are two commands
% used to separate the names and addresses of multiple authors: \And and \AND.
%
% Using \And between authors leaves it to LaTeX to determine where to break the
% lines. Using \AND forces a line break at that point. So, if LaTeX puts 3 of 4
% authors names on the first line, and the last on the second line, try using
% \AND instead of \And before the third author name.

\author{%
  Felix Dangel\thanks{Equal contribution}\\
  Vector Institute \\
  Toronto \\ Canada \\
  \texttt{fdangel@vectorinstitute.ai} \\
  \And
  Johannes M\"uller$^*$\\
  Chair of Mathematics of Information Processing \\
  RWTH Aachen University \\
  Aachen, Germany \\
  \texttt{mueller@mathc.rwth-aachen.de} \\
  \And
  Marius Zeinhofer$^*$\\
  Seminar for Applied Mathematics, ETH Z\"urich, \\
  Department of Nuclear Medicine, University Hospital Freiburg\\
  \texttt{marius.zeinhofer@uniklinik-freiburg.de}
}
%%% Local Variables:
%%% mode: latex
%%% TeX-master: "../main"
%%% End:


\newcommand{\jm}[1]{\textcolor{blue}{#1}}

\begin{document}

\maketitle

\begin{abstract}
Physics-Informed Neural Networks (PINNs) are infamous for being hard to train.
Recently, second-order methods based on natural gradient and Gauss-Newton methods have shown promising performance, improving the accuracy achieved by first-order methods by several orders of magnitude. 
While promising, the proposed methods only scale to networks with a few thousand parameters due to the high computational cost to evaluate, store, and invert the curvature matrix.
We propose Kronecker-factored approximate curvature (KFAC) for PINN losses that greatly reduces the computational cost and allows scaling to much larger networks.
Our approach goes beyond the popular KFAC for traditional deep learning problems as it captures contributions from a PDE's differential operator that are crucial for optimization. 
To establish KFAC for such losses, 
we use Taylor-mode automatic differentiation to describe the differential operator's computation graph as a forward network with shared weights which allows us to 
%we provide new insights into the differential operator's computation graph using Taylor-mode automatic differentiation and 
%combine them with a recently proposed generalization of KFAC to linear layers with weight sharing.
apply a variant of KFAC for networks with weight-sharing. 
%Specifically, we show that linear layers inside a neural net process multiple vectors containing partial derivatives, hence act like linear layers with weight sharing which allows for a straightforward KFAC approximation.
Empirically, we find that our KFAC-based optimizers are competitive with expensive second-order methods on small problems, scale more favorably to higher-dimensional neural networks and PDEs, and consistently outperform first-order methods.
\end{abstract}

\section{Introduction}
PINNs are difficult to optimize.
\begin{itemize}
    \item PINNs receive ever growing amount of attention
    \item their failure to produce high accuracy solution when trained with variants of GD like Adam is well documented
    \item Quasi-Newton methods like L-BFGS yield improved but still not very high accuracy
    \item Other suggestions: reweighting of the loss, specialized sampling strategies, greedy training, reformulation as saddle point problem
    \item recently, a variant of NG based on the geometry of the specific energy / PDE was proposed; yields greatly improved accuracy over direct gradient-based optimizers and enjoys the nice property that it can be shown to mimic Newton's method in function space; for PINNs it can be seen as Gau\ss-Newton method in the space of residuals, for other problems as a generalized GN?
    \item whereas, this method was shown to be able to produce highly accurate approximations of the solution of the PDE it comes with a considerable iteration cost as it involves the solution of a linear system of the size of the number of parameters. Hence, this is only feasible for networks of small to moderate size when done naively.
    \item we build on the idea of Kronecker-factored approximations known as KFAC proposed in the context of supervised learning to provide an efficient implementation of energy natural gradients; 
    however, PDE terms appear in the Gramian, so existing implementations can not be used off the shelve 
\end{itemize}

\paragraph{Contribution:} \toodoo{Formulate our goal.}

\begin{itemize}
    \item we develop a Kronecker-factored approximation of the Gramian matrix appearing as a preconditioner in the energy natural gradient method; we call it KFAG 
    \item we provide an efficient implementation of KFAG; we experimentally show that it provides a good approximation of the true Gramian 
    \item We demonstrate that KFAG can be used to efficiently train networks of considerable size in a PINN style setting 
\end{itemize}

\paragraph{Related work:}
\begin{itemize}
\item OG KFAC papers: \cite{martens2015optimizing}, \cite{martens2018kroneckerfactored}, double check similarities to RNNs
\item KFAC for Rayleigh quotients:
\item PINNs: recent preconditioning papers
\end{itemize}

%%% Local Variables:
%%% mode: latex
%%% TeX-master: "../main"
%%% End:


\section{Background}\label{sec:background}
For simplicity, we present our approach for multi-layer perceptrons (MLPs) consisting of fully-connected and element-wise activation layers.
However, the generality of Taylor-mode automatic differentiation and KFAC for linear layers with weight sharing allow our KFAC to be applied to such layers (e.g.\,fully-connected, convolution, attention) in arbitrary neural network architectures.

\paragraph{Flattening \& Derivatives}
We vectorize matrices using the \emph{first-index-varies-fastest} convention, i.e.\,column-stacking (row index varies first, column index varies second) and denote the corresponding flattening operation by $\flatten$.
This allows to reduce derivatives of matrix- or tensor-valued objects back to the vector case by flattening a functions input and output domain before differentiation.
The Jacobian of a vector-to-vector function $\va \mapsto \vb(\va)$ has entries $[\jac_{\va}\vb]_{i,j} = \nicefrac{\partial \evb_i}{\partial \eva_j}$.
For a matrix-to-matrix function $\mA \mapsto \mB(\mA)$, the Jacobian is $\jac_{\mA} \mB = \jac_{\flatten \mA }\flatten\mB$.
A useful property of $\flatten$ is $\flatten(\mA\mX\mB) = (\mB^\top\otimes \mA)\flatten{\mX}$ for matrices $\mA, \mX, \mB$ which implies $\jac_\mX(\mA\mX\mB) = \mB^\top\otimes \mA$.

\paragraph{Sequential neural nets} Consider a \emph{sequential neural network} $u_{\vtheta} = f_{\vtheta^{(L)}} \circ f_{\vtheta^{(L-1)}} \circ \ldots \circ f_{\vtheta^{(1)}} $ of depth $L\in\mathbb N$. It consists of layers $f_{\vtheta^{(l)}}\colon \sR^{h^{(l-1)}}\to\sR^{h^{(l)}}$, $\vz^{(l-1)}\mapsto \vz^{(l)} = f_{\vtheta^{(l)}}(\vz^{(l-1)})$ with trainable parameters $\vtheta^{(l)} \in \sR^{p^{(l)}}$ that transform an input $\vz^{(0)} \coloneqq \vx \in \mathbb R^{d \coloneqq h^{(0)}}$ into a prediction $u_\vtheta(\vx) = \vz^{(L)} \in \sR^{h^{(L)}}$ via intermediate representations $\vz^{(l)} \in \sR^{h^{(l)}}$.
In the context of PINNs, we use networks with scalar outputs ($h^{(L)}=1$) and denote the concatenation of all parameters by $\vtheta = (\vtheta^{(1)\top}, \dots, \vtheta^{(L)\top})^{\top} \in \sR^P$.
A common choice is to alternate fully-connected and activation layers.
Linear layers map $\vz^{(l-1)} \mapsto \vz^{(l)} = \mW^{(l)} \vz^{(l-1)}$ using a weight matrix $\mW^{(l)} = \flatten^{-1}\vtheta^{(l)}  \in \sR^{h^{(l)} \times h^{(l-1)}}$ (bias terms can be added as an additional column and by appending a $1$ to the layer input).
Activation layers map $\vz^{(l-1)}\mapsto \vz^{(l)} \sigma(\vz^{(l-1)})$ element-wise for a (typically smooth) $\sigma\colon\mathbb R\to\mathbb R$.

\subsection{Energy Natural Gradients for Physics-Informed Neural Networks}
Let us consider a domain $\Omega\subseteq\mathbb R^d$ and the partial differential equation
\begin{align*}\tag{PE}\label{eq:PE}
  -\mathcal{L} u & = f \quad \text{in }\Omega \\
  u & = g \quad \text{on }\partial\Omega
\end{align*}
with square-integrable right hand side $f\in L^2(\Omega)$ and twice continuously differentiable boundary condition $g\in C^2(\Omega)\cap C(\overline{\Omega})$.
$\mathcal{L}$ denotes a differential operator, e.g.\,the Laplacian $\mathcal{L} u = \Delta u = \sum_{i=1}^d \partial_{\evx_i}^2 u$.
We parametrize $u$ using a neural network and train its parameters $\vtheta$ to minimize the loss
\begin{align}\label{eq:pinn-loss}
  L(\vtheta)
  &=
    \underbrace{\frac{1}{2N_\Omega} \sum_{n=1}^{N_\Omega} (\mathcal{L} u_\vtheta(\vx_i) + f(\vx_n))^2}_{\eqqcolon L_\Omega(\vtheta)} + \underbrace{\frac{1}{2N_{\partial\Omega}}\sum_{n=1}^{N_{\partial\Omega}} ( u_\vtheta(\vx^\text{b}_n) - g(\vx^\text{b}_n))^2}_{\eqqcolon L_{\partial\Omega}(\vtheta)},
\end{align}
with points $\{\vx_n \in \Omega \}_{n=1}^{N_\Omega}$ from the domain's interior, and points $\{\vx^\text{b}_n \in \partial\Omega \}_{n=1}^{N_{\partial\Omega}}$ on its boundary.\footnote{The second regression loss can also include other constraints like measurement data.}

%%% Local Variables:
%%% mode: latex
%%% TeX-master: "../main"
%%% End:

\subsection{Energy natural gradients}

%\begin{itemize}
%    \item recently, energy NGs have been proposed
%    \item one can show that they mimic Newtons method in function space
%    \item yield very good accuracy
%    \item
%\end{itemize}

Natural gradients have been introduced by~\citet{amari1998natural} and have shown great success in reinforcement learning, and other problems...
The general idea is to replace the vanilla GD update rule by a preconditioned version
    \[ \theta_{k+1} = \theta_k - \eta_k G(\theta_k)^{-1} \nabla L(\theta_k), \]
where $G(\theta)\in\mathbb R^{p\times p}$, $G(\theta)_{ij} \coloneqq g_{u_\theta}(\partial_{\theta_i} u_\theta, \partial_{\theta_j} u_\theta)$ is a matrix capturing the function space geometry of the problem and its parametrization.
Classically, $G$ is chosen as the Fisher information matrix,
%\begin{equation}
%    F_I(\theta)_{ij} = \sum_{x} \frac{\partial_{\theta_i}p_\theta(x)\partial_{\theta_j}p_\theta(x)}{p_\theta(x)} = \sum_{x} \partial_{\theta_i} \log p_\theta(x) \partial_{\theta_j} \log p_\theta(x),
%\end{equation}
in which case the Riemannian metric $g$ is given by the Fisher-Rao metric~\cite[text]{keylist}.
For a supervised learning problem with training data $(x_1, y_1), \dots, (x_N, y_N)$ the (empirical) Fisher-information matrix commonly used, has entries the entries 
\begin{equation}
  F(\theta)_{ij} = \sum_{n=1}^N \partial_{\theta_i} u_\theta(x_n)\partial_{\theta_j} u_\theta(x_n),
\end{equation}
see~\cite{amari2000natural,martens2020new}. 

In the PINN setting however, the models $u_\theta$ are functions rather than probability measures %$p_\theta$
and the loss involves PDE terms.
%In order to adjust the definition of
To capture the geometric properties of this specific problem we consider the following Fisher / Gramian matrix %to this problemThe energy natural gradient is for this example to use the Fischer/Gramian of the form
\begin{equation}
  F(\theta) = F_\Omega(\theta) + F_{\partial\Omega}(\theta) = \frac1{{N_\Omega}} \sum_{k=1}^{N_\Omega} \partial_{\theta_i} \Delta u_\theta(x_k) \partial_{\theta_j} \Delta u_\theta(x_k) + \frac1{{N_{\partial\Omega}}} \sum_{k=1}^{N_{\partial\Omega}} \partial_{\theta_i} u_\theta(x_k^b) \partial_{\theta_j} u_\theta (x_k^b).
\end{equation}
%where
%\begin{equation}\label{eq:FisherInterior}
%  F_\Omega(\theta)_{ij} = \frac1{{N_\Omega}} \sum_{k=1}^{N_\Omega} \partial_{\theta_i} \Delta u_\theta(x_k) \partial_{\theta_j} \Delta u_\theta(x_k)
  % = \frac1{{N_\Omega}} \sum_{i=1}^{N_\Omega} (\partial_{\theta_i} f_\theta) (\partial_{\theta_j} f_\theta ),
%\end{equation}
% where $f_\theta = \Delta u_\theta$.
%and
%\begin{equation}
%  F_{\partial\Omega}(\theta)_{ij} = \frac1{{N_{\partial\Omega}}} \sum_{k=1}^{N_{\partial\Omega}} \partial_{\theta_i} u_\theta(x_k^b) \partial_{\theta_j} u_\theta (x_k^b).
  % = \frac1{{N_\Omega}} \sum_{i=1}^{N_\Omega} (\partial_{\theta_i} f_\theta) (\partial_{\theta_j} f_\theta ),
%\end{equation}
It can be shown that the energy natural gradient method mimics Newton's method up to a projection onto the tangent space of the model and a discretization error that vanishes quadratically in the step size~\cite{muller2023achieving, }
Further, the Gramian matrix admits an interpretation as the Gauß-Newton matrix of the residual function, see~\cite{} and Appendix....

%%% Local Variables:
%%% mode: latex
%%% TeX-master: "../main"
%%% End:

\subsection{Kronecker-factored Approximate Curvature (KFAC)}

%\toodoo{F.D.  Get to the point more quickly. Make notation  consistent.}

We review the idea of Kronecker-factored Approximate Curvature (KFAC) which was introduced by~\citet{?, martens2015optimizing, ?} as an approximation of the per-layer Fisher information by a Kronecker product to speed up the computation of the natural gradient direction. 
%We review the general principle here. 
%This approach was introduced in the context of maximum likelihood estimation with a neural network and we reivew its basic principles here. 
%Later, we will expand this to Gramian matrices with PDE terms. 
\toodoo{We could also follow \cite{eschenhagen2023kroneckerfactored} for the notation} 


\paragraph{Block-diagonal approximation}
In the first step the Fisher-information matrix $\mF$ can be approximated by a block diagonal matrix with blocks corresponding to the Fisher-information matrices of the individual layers of the network, i.e., $\mF(\vtheta) \approx \operatorname{diag}(\mF(\vtheta^{(1)}), \dots, \mF(\vtheta^{(L)}))$. 
Note that a block diagonal linear system can be solved by solving the subsystems corresponding to the individual blocks, hence reducing the computational complexity. 

\paragraph{Kronecker-factored approximation of the blocks}
We now consider the individual blocks $\mF(\vtheta^{(l)})$, for which we examine $\jac_{\vtheta^{(l)}} u_{\vtheta}(x_n)$ for a fixed data point. 
The parameters $\vtheta^{(l)} = \mW^{(l)}$ of the $l$-th layer appears in the computational graph by $\vz^{(l)} = \mW^{(l)}\vz^{(l-1)}$ and note that by the vec-trick, we have $\jac_{\vtheta^{(l)}} \vz^{(l)} = \vz^{(l-1)} \otimes I$. 
By the chain rule, we have
\begin{align}
    \jac_{\vtheta^{(l)}} u_{\vtheta}(\vx_n) & = \jac_{\vtheta^{(l)}} \vz_n^{(l)} \jac_{\vz^{(l)}}  \vz^{(L)}_n = %(\vz^{(l-1)}_n\otimes I) \nabla_{\vz^{(l)}}  \vz^{(L)}_n = 
    \vz^{(l-1)}_n\otimes  \jac_{\vz^{(l)}}  \vz^{(L)}_n. 
\end{align}
%and hence 
%\begin{align}
%    \nabla_{\vtheta^{(l)}} u_{\vtheta}(x_n)\nabla_{\vtheta^{(l)}} u_{\vtheta}(x_n)^\top = (\vz^{(l-1)}_n\otimes \vz^{(l-1)}_n) \nabla_{\vz^{(l)}}  \vz^{(L)}_n\nabla_{\vz^{(l)}}  \vz^{(L)}_n^\top.  
%\end{align}
Summing over the data points and using  $\sum_n \mA_n \otimes \mB_n \approx (\sum_n \mA_n) \otimes (\sum_n \mB_n)$ we obtain 
\begin{equation}
    \mF(\vtheta^{(l)}) %= \sum_{n=1}^N \nabla_{\vtheta^{(l)}} u_{\vtheta}(x_n)\nabla_{\vtheta^{(l)}} u_{\vtheta}(x_n)^\top 
    \approx \left(\sum_{n=1}^N \vz^{(l-1)}_n {\vz^{(l-1)}_n}^\top \right)\otimes \left(\sum_{n=1}^N\nabla_{\vz^{(l)}}  \vu_n\nabla_{\vz^{(l)}}  \vu_n^\top\right),
\end{equation}
see~\citep{eschenhagen2023kroneckerfactored}. 
Solving a Kronecker-factored linear system is only as expensive as solving two systems of the sizes of the two factors therefore greatly reducing the computational cost. \todo{correct?} 
%the inverse of a Kronecker-factored matrix is given by the Kronecker product of the individual inverses hence reducing the computational complexity of the corresponding linear system. 

\paragraph{Weight sharing}
\todo{add}

\clearpage

Assume we have drawn a data set $\smash{\sD = \left\{ (\vx_n, \vy_n) \right\}_{n=1}^N}$ with $\smash{(\vx_n, \vy_n) \stackrel{\text{i.i.d}}{\sim} p_{\text{data}}(\vx, \vy)}$.
We want to approximate the data-generating process through $p_{\vtheta}(\vx, \vy)$ by modelling a likelihood $p_{\vtheta}(\vy \mid \vx)$ for the labels with a neural network, that is we use $p_{\vtheta}(\vx, \vy) = p_{\text{data}}(\vx) p_{\vtheta}(\vy | \vx)$ and maximize $KL(p_{\text{data}} || p_{\vtheta})$.
Since $p_{\text{data}}$ is not accessible, one replaces $p_{\text{data}}(\vx)$ and $p_{\text{data}}(\vy \mid \vx)$ with their empirical distributions implied by $\sD$.
This yields the objective \cite[see][Section 4]{martens2020new}
\begin{align*}
  \frac{1}{N} \sum_{n=1}^N -\log p_{\vtheta}(\vy_n \mid \vx_n)
\end{align*}
which corresponds to the empirical risk $\frac{1}{N} \sum_{n=1}^N \ell(\vx_n, \vy_n, \vtheta)$ with a negative log-likelihood loss function, such as square or softmax cross-entropy loss.
The likelihood modelled by the neural network is of the form $p_{\vtheta}(\vy_n
\mid \vx_n) = r(\vy_n \mid f_{\vtheta}(\vx))$. The Fisher of our modelled
probability $p_{\vtheta}(\vx, \vy)$ is
\begin{align*}
  \mF(\vtheta)
  &=
    \E_{(\vx, \vy) \sim p_{\vtheta}(\vx,\vy)}
    \left[
    \grad{\vtheta} \log p_{\vtheta}(\vx, \vy)
    (\grad{\vtheta} \log p_{\vtheta}(\vx, \vy))^{\top}
    \right]
  \\
  &=
    \E_{p_{\text{data}}(\vx)}
    \E_{p_{\vtheta}(\vy \mid \vx)}
    \underbrace{
    \left[
    \grad{\vtheta} \log p_{\vtheta}(\vy \mid \vx)
    (\grad{\vtheta} \log p_{\vtheta}(\vy \mid \vx))^{\top}
    \right]
    }_{\coloneqq \mF_{\vy \mid \vx}(\vtheta)}\,.
  \\
  &\approx
    \frac{1}{N} \sum_{n=1}^N
    \E_{p_{\vtheta}(\vy \mid \vx_n)}
    \left[
    \grad{\vtheta} \log p_{\vtheta}(\vy \mid \vx_n)
    (\grad{\vtheta} \log p_{\vtheta}(\vy \mid \vx_n))^{\top}
    \right]
\end{align*}

\paragraph{One datum, no weight sharing:}
Let's start with maximum likelihood estimation with a single data point $(\vx, \vy)$.
Consider a linear layer inside a neural network which maps some vector-valued hidden feature of $\vx$, $\va \in \sR^{D_{\text{in}}}$ to a vector-valued output $\vz \in \sR^{D_{\text{out}}}$ via $\vz = \mW \va$.
$\vz$ is then further processed and used to compute the negative log-likelihood loss $\ell(\vx, \vy, \mW) = - \log p(\vy \mid \vx, \mW)$.
For this single-usage layer, the weigh matrix's Fisher is exactly Kroneckerfactored, $\mF(\mW) = \va \va^{\top} \otimes \E_{\hat{\vy} \sim p(\vy \mid \vx, \mW)}\left[ \vg \vg^{\top} \right]$ where $\vg = \grad{\vz} \ell(\vx, \hat{\vy}, \mW)$.
By applying the chain rule at the layer's output, the Kronecker structure emerges from the output-parameter Jacobian $\jac_{\mW}\vz = \va^{\top} \otimes \mI$.
In practise, we will use one sample from the model's likelihood to estimate the expectation, $\mF(\mW) \approx \vz \vz^{\top} \otimes \vg \vg^{\top}$.

% explain how batch axes are treated
\paragraph{Multiple data, no weight sharing} In the presence of multiple data points, the sum over per-datum Kronecker products is further approximated as a Kronecker product of sums over data points:
\begin{align*}
  \mF(\mW)
  &=
    \frac{1}{N}
    \sum_{n=1}^N
    \va_n \va_n^{\top} \otimes \E_{\hat{\vy}_n \sim p(\vy_n \mid \vx_n, \mW)}\left[ \vg_n \vg_n^{\top} \right]
  \\
  &\approx
    \left(
    \frac{1}{N}
    \sum_{n=1}^N
    \va_n \va_n^{\top}
    \right)
    \otimes
    \left(
    \sum_{n=1}^N
    \E_{\hat{\vy}_n \sim p(\vy_n \mid \vx_n, \mW)}\left[ \vg_n \vg_n^{\top} \right]
    \right)
  \\
  &\approx
    \left(
    \frac{1}{N}
    \sum_{n=1}^N
    \va_n \va_n^{\top}
    \right)
    \otimes
    \left(
    \sum_{n=1}^N
    \vg_n \vg_n^{\top}
    \right)
\end{align*}

% expand approximation treats the shared axis like a batch axis
\paragraph{One datum, weight sharing} Now consider a layer whose weight is applied onto \emph{multiple} vectors.
This concept is known as weight sharing.
This could be a linear layer with matrix-valued inputs like in attention, a convolution layer whose kernel is shared between patches of the input, or weights that are used multiple times throughout the computation graph (e.g.\, weight tying).
This means the layer will not process a single vector $\va$, but a sequence of vectors $\left\{ \va_1, \dots, \va_S \right\}$ where $S$ denotes weight sharing number.
We can column-stack these vectors into a matrix $\mA \in \sR^{D_{\text{in}}\times S}$, likewise for the linear layer's outputs $\vz = \mW \mA \in \sR^{D_{\text{out}}\times S}$ and activation gradients $\mG \in \sR^{D_{\text{out}} \times S}$.
The output-weight Jacobian of a weight-sharing layer is $\jac_{\mW} \mZ = \mA^{\top} \otimes \mI$ \cite[see e.g.][]{dangel2020modular} and the Fisher does not simplify into a Kronecker product without further approximations.
As described in \citet{eschenhagen2023kroneckerfactored}, there are two possible Kronecker approximations for this setup.
We will focus on the \emph{expand} approximation, which yields the Kronecker approximation for convolutional layers proposed by~\citet{grosse2016kroneckerfactored}.
It treats the shared axis like a batch axis,
$\mF(\mW) \approx \nicefrac{1}{S} \sum_{s=1}^S \va_s \va_s^{\top} \otimes \sum_{s=1}^S \vg_s \vg_s^{\top}$ where $\vg_s = \grad{\vz_s} \ell(\vx, \hat{\vy}, \mW)$.
We can express this in matrix notation as $\mF(\mW) \approx \nicefrac{1}{S} \mA \mA^{\top} \otimes \mG \mG^{\top}$.


%%% Local Variables:
%%% mode: latex
%%% TeX-master: "../main"
%%% End:


\section{Kronecker-Factored Approximate Curvature for
PINNs}\label{sec:kfac_pinns}



ENGD's Gramian is a sum of PDE and boundary Gramians, $\mG(\vtheta)= \mG_\Omega(\vtheta) + \mG_{\partial\Omega}(\vtheta)$.
We will approximate each Gramian separately with a block diagonal matrix with Kronecker-factored blocks, $\mG_{\bullet}(\vtheta) \approx \diag(\mG^{(1)}_{\bullet}(\vtheta), \dots, \mG^{(L)}_{\bullet}(\vtheta))$ for $\bullet \in \{\Omega, \partial\Omega\}$ with $\mG^{(l)}_{\bullet}(\vtheta) \approx \mA^{(l)}_{\bullet} \otimes \mB^{(l)}_{\bullet}$.
For the boundary Gramian $\mG_{\partial\Omega}(\vtheta)$, we can re-use the established KFAC from~\Cref{eq:kfac-linear} as its loss corresponds to regression over the network's output.
The interior Gramian $\mG_\Omega(\vtheta)$, however, involves PDE terms in the form of network derivatives and therefore \emph{cannot} be approximated with the existing KFAC.
It requires a new approximation that we develop here for the running example of the Poisson equation and more general PDEs (\Cref{eq:KFAC-PINN,eq:KFAC-PINNs-general}).
To do so, we need to make the dependency between the weights and the differential operator $\mathcal{L}u$ explicit.
We use Taylor-mode automatic differentiation to express this computation of higher-order derivatives as forward passes of a larger net with shared weights, for which we then propose a Kronecker-factored approximation, building on KFAC's recently-proposed generalization to linear layers with weight sharing~\cite{eschenhagen2023kroneckerfactored}.

\subsection{Higher-order Forward Mode Automatic Differentiation as Weight Sharing}
\label{sec:taylor-mode-AD}

Here, we review higher-order forward mode, also known as \emph{Taylor-mode}, automatic differentiation~\citep{griewank1996algorithm, griewank2008evaluating, bettencourt2019taylor}.
Many PDEs only incorporate first- and second-order partial derivatives and we focus our discussion on second-order Taylor mode for MLPs to keep the presentation light.
However, one can treat higher-order PDEs and arbitrary network architectures completely analogously.

Taylor-mode propagates directional (higher-order) derivatives.
We now recap the forward propagation rules for MLPs consisting of fully-connected and element-wise activation layers.
Our goal is to evaluate first-and second-order partial derivatives of the form $\partial_{\evx_i}u, \partial^2_{\evx_i, \evx_j}u$ for $i,j = 1, \dots, d$.
At the first layer, set $\vz^{(0)} = \vx\in\mathbb R^d, \partial_{x_i}\vz^{(0)} = \ve_i\in\mathbb R^d$, i.e., the $i$-th basis vector and $\partial^2_{x_i,x_j}\vz^{(0)} = \vzero \in\mathbb R^d$.

For a linear layer $f_{\vtheta^{(l)}}(\vz^{(l-1)}) = \mW^{(l)} \vz^{(l-1)}$, applying the chain rule yields the propagation rule
\begin{subequations}\label{eq:forward_pass}
  \begin{align}
    \vz^{(l)}
    &=
      \mW^{(l)} \vz^{(l-1)} \quad \in \sR^{h^{(l)}}\,,
    \\
    \partial_{x_i} \vz^{(l)}
    &=
      \mW^{(l)} \partial_{x_i} \vz^{(l-1)}  \quad \in \sR^{h^{(l)}}\,,
    \\
    \label{subeq:secondOrderForward-LinearLayer}
    \partial^2_{x_i,x_j} \vz^{(l)}
    &=
      \mW^{(l)} \partial^2_{x_i,x_j} \vz^{(l-1)}  \quad \in \sR^{h^{(l)}}\,.
  \end{align}
\end{subequations}
The propagation rule through a nonlinear element-wise activation layer $\vz^{(l-1)}\mapsto \sigma(\vz^{(l-1)})$ is
\begin{subequations}\label{eq:taylor-forward-activation}
  \begin{align}
    \vz^{(l)}
    &=
      \sigma(\vz^{(l-1)})\quad \in \sR^{h^{(l)}}\,,
    \\
    \partial_{x_i} \vz^{(l)}
    &=
      \sigma'(\vz^{(l-1)}) \odot \partial_{x_i} \vz^{(l-1)}\quad \in \sR^{h^{(l)}}\,,
    \\
    \label{subeq:secondOrderForward-nonlinearLayer}
    \partial^2_{x_i,x_j} \vz^{(l)}
    &=
      \partial_{x_i} \vz^{(l-1)} \odot \sigma''(\vz^{(l-1)}) \odot \partial_{x_j} \vz^{(l-1)}
      +
      \sigma'(\vz^{(l-1)}) \odot \partial^2_{x_i,x_j} \vz^{(l-1)}\quad \in \sR^{h^{(l)}}\,.
  \end{align}
\end{subequations}

\paragraph{Forward Laplacian} For differential operators of special structure, we can fuse the Taylor-mode forward propagation of individual directional derivatives in \Cref{eq:forward_pass,eq:taylor-forward-activation} and obtain a more efficient computation.
E.g., to compute not the full Hessian but only the Laplacian, we can simplify the forward pass, which yields the \emph{forward Laplacian} framework of~\citet{li2023forward}.
To the best of our knowledge, this connection has not been pointed out in the literature.
Concretely, by summing~\eqref{subeq:secondOrderForward-LinearLayer} and~\eqref{subeq:secondOrderForward-nonlinearLayer} over $i=j$, we obtain the Laplacian forward pass for linear and activation layers
\begin{subequations}\label{eq:forward-laplacian-main}
  \begin{align}
    \label{eq:forward_Laplacian_linear}
    \Delta_\vx\vz^{(l)}
    &=
      \mW^{(l)}\Delta_\vx\vz^{(l-1)}
      \quad \in \sR^{h^{(l)}}\,,
    \\
    \label{eq:forward_Laplacian_nonlinear}
    \Delta_\vx\vz^{(l)}
    &=
      \sigma'(\vz^{(l-1)})\odot\Delta_\vx\vz^{(l-1)}
      +
      \sum_{i=1}^d \sigma''(\vz^{(l-1)})\odot (\partial_{x_i}{\vz^{(l-1)}})^{\odot 2}
      \quad \in \sR^{h^{(l)}}\,.
  \end{align}
\end{subequations}
This reduces computational cost, but is restricted to PDEs that involve second-order derivatives only via the Laplacian, or a partial Laplacian over a sub-set of input coordinates (e.g.\,heat equation, \Cref{sec:experiments}).
For a more general second-order linear PDE operator $\sum_{i,j=1}^d c_{ij} \partial^2_{\evx_i,\evx_j}$, the forward pass for a linear layer is $\mathcal{L} \vz^{(l)} = \mW^{(l)}\mathcal{L} \vz^{(l-1)}$, generalizing~\eqref{eq:forward_Laplacian_linear}, and similarly for~\Cref{eq:forward_Laplacian_nonlinear}.

Importantly, the computation of higher-order derivatives for linear layers boils down to a forward pass through the layer with weight sharing over the different partial derivatives (\Cref{eq:forward_pass}), and weight sharing can potentially be reduced depending on the differential operator's structure (\Cref{eq:forward_Laplacian_linear}).
Therefore, we can use the concept of KFAC in the presence of weight sharing to derive a principled Kronecker approximation.

\subsection{A Kronecker-factored approximation for ENGD with the Laplace operator}\label{sec:KFAC-Laplace}
Just like in the case of the classic KFAC algorithm, we only consider the diagonal blocks of the Gauss-Newton matrix, which for the Poisson equation, are given by
\begin{align}
  \mG_\Omega(\mW^{(l)}) = \frac1{{N_\Omega}} \sum_{n=1}^{N_\Omega} \jac_{\mW^{(l)}} \Delta_\vx \vu_n^\top \jac_{\mW^{(l)}} \Delta_\vx \vu_n \in\mathbb R^{{h^{(l-1)}}^2\times {h^{(l-1)}}^2}
\end{align}
Just like in the case of the block Fisher matrices $\mF(\vtheta^{(l)})$ we use the chain rule and compute
\begin{align*}
  \jac_{\mW^{(l)}} \Delta_\vx \vu_n & = \jac_{\mZ^{(l)}}\Delta_\vx \vu_n \jac_{\mW^{(l)}} \mZ^{(l)} %=
  % \jac_{\mW^{(l)}} \mZ^{(l)} \jac_{\mZ^{(l)}}\mZ^{(L)}\jac_{\mZ^{(L)}}\Delta_\vx \vu_n.
  % \\ & =
  % \operatorname{diag}\left({\vz^{(l-1)}}^\top \otimes \mI, \dots, {\Delta_\vx\vz^{(l-1)}}^\top \otimes \mI\right) \jac_{\mZ^{(l)}}\Delta_\vx \vu_n
  % \begin{pmatrix}
      %       \nabla_{\vz^{(l)}} \Delta_{\vx}\vu_n \\
  %   \nabla_{\partial_{\vx_1}\vz^{(l)}}\Delta_{\vx}\vu_n \\
  %   \vdots \\
  %   \nabla_{\partial_{\vx_d}\vz^{(l)}} \Delta_{\vx}\vu_n  \\
  %   \nabla_{\Delta_{\vx}\vz^{(l)}} \Delta_{\vx}\vu_n \\
  % \end{pmatrix}^\top
  % \begin{pmatrix}
  %   {\vz^{(l-1)}_n}^\top \otimes \mI \\
  %   {\partial_{\vx_1}\vz^{(l-1)}_n}^\top \otimes \mI \\
  %   \vdots \\
  %   {\partial_{\vx_d}\vz^{(l-1)}_n}^\top \otimes \mI \\
  %   {\Delta_{\vx}\vz^{(l-1)}_n}^\top \otimes \mI \\
  % \end{pmatrix}
  % \\ &
  % =
  % \sum_{s=1}^S (\nabla_{\mZ^{(l)}_s} \Delta_{\vx}\vu_n)^\top  ({\mZ^{(l-1)}_{n,s}}^\top \otimes \mI)?
  % \\ &
    = \left(\sum_{s=1}^S \mZ^{(l-1)}_{n,s} \otimes \nabla_{\mZ^{(l)}_s} \Delta_{\vx}\vu_n\right)^\top,
    %{\vz^{(l-1)}_n}^\top \otimes \jac_{\vz^{(l)}} \Delta_{\vx}\vu_n
    %+
    %\sum_{i=1}^d {\partial_{x_i}\vz^{(l-1)}_n}^\top \otimes \jac_{\partial_{x_i}\vz^{(l)}}\Delta_{\vx}\vu_n
    %\\ & \quad
    %+
    %{\Delta_{\vx}\vz^{(l-1)}_n}^\top \otimes \jac_{\Delta_{\vx}\vz^{(l)}} \Delta_{\vx}\vu_n.
    %\operatorname{diag}\left({\vz^{(l-1)}}^\top \otimes \jac_{\vz^{(l-1)}} \Delta_{\vx} \vu_n, \dots, {\Delta_\vx\vz^{(l-1)}}^\top \otimes \jac_{\Delta_\vx\vz^{(l-1)}} \Delta_{\vx} \vu_n\right)? not quite
\end{align*}
where $\mZ_{n, 1}^{(l)} = \vz_n^{(l)}, \mZ_{n, 2}^{(l)} = \partial_{x_1}\vz_n^{(l)}, \dots, \mZ_{n, 1+d}^{(l)} = \partial_{x_d}\vz_n^{(l)}\in\mathbb R^{h^{(l)}}$ and $\mZ_{n, 2+d}^{(l)} = \Delta_x\vz_n^{(l)}\in\mathbb R^{h^{(l)}}$ and $S=d+2$.
Using the notation
$\vg_{n, s}^{(l)} = \nabla_{\mZ_{s}^{(l)}} \Delta_\vx\vu_n\in\mathbb R^{h^{(l)}}$ we obtain
\begin{equation}\label{eq:laplace_gramian_block_exact}
    \mG_\Omega(\mW^{(l)})
    =
    \frac1N\sum_{n=1}^N
    \left[\sum_{s=1}^S \left( \mZ^{(l-1)}_{n,s}\otimes \vg_{n,s}^{(l)} \right)
    \cdot
    \sum_{s=1}^S \left( \mZ
    ^{(l-1)}_{n,s}\otimes \vg_{n,s}^{(l)} \right)^\top\right]
\end{equation}
Now we can use any Kronecker-factored approximation for feedforward networks with weight sharing, where we choose to work with the following.

%
%\paragraph{Kronecker approximation of the PDE Gramian}


%\paragraph{The expand approximation}
Approximating the inner matrix product of sums in equation~\eqref{eq:laplace_gramian_block_exact} by the sum of matrix products, simplifying, and approximating the sum of Kroneckers by the Kronecker of the sum we obtain
\begin{tcolorbox}[colframe=kfac, title={KFAC for ENGD with the Laplace operator},bottom=0mm,top=0mm,middle=0mm]
\begin{align}\label{eq:KFAC-PINN}
    \hat{\mG}_\Omega%^{\textup{exp}}
    (\mW^{(l)})
    \coloneqq \frac{1}{N^2 S}
    \left[\sum_{n=1}^N \sum_{s=1}^S \mZ^{(l-1)}_{n,s}{\mZ^{(l-1)}_{n,s}}^\top \right]
    \otimes
    \left[\sum_{n=1}^N\sum_{s=1}^S \vg^{(l)}_{n,s}{\vg^{(l)}_{n,s}}^\top   \right].
\end{align}
\end{tcolorbox}

%\todo{@Felix. Here it would be natural to comment on forward and empirical, right? To my understanding we do not really use the reduce approximation, correct?}

%\paragraph{The reduce approximation}
%Alternatively, approximating the sum of Kronecker products (running over $s$) by the Kronecker product of the sum, simplifying and again approximating for the outer sum (running over $n$) we obtain
%\begin{align}
%    \hat{\mG}_\Omega^{\textup{red}}(\mW^{(l)})
%    \coloneqq
%    \frac{1}{N^2 S^2} \left[\sum_{n=1}^N \left( \sum_{s=1}^S \mZ^{(l-1)}_{n,s} \right) \left(\sum_{s=1}^S {\mZ^{(l-1)}_{n,s}}\right)^\top\right]
%    \otimes
%    \left[\sum_{n=1}^N\left(\sum_{s=1}^S \vg^{(l)}_{n,s} \right) \left( \sum_{s=1}^S {\vg^{(l)}_{n,s}} \right)^\top \right]
%\end{align}
%This approximation coincide with the \emph{reduce} setting described in \cite{eschenhagen2023kroneckerfactored}.




%\paragraph{Empirical and non empirical Gramians}

%One way is to keep both Kronecker approximations separate, but then we need to invert a sum of two Kronecker products.
%This can be done, see \Cref{sec:inverse_kronecker_sum}, but adds twice the memory overhead compared to having a single Kronecker approximation.



% expand approximation treats the shared axis like a batch axis
\begin{comment}
\paragraph{Kronecker approximation under weight sharing}
Now consider a layer whose weight is applied onto \emph{multiple} vectors.
This concept is known as weight sharing.
This could be a linear layer with matrix-valued inputs like in attention, a convolution layer whose kernel is shared between patches of the input, or weights that are used multiple times throughout the computation graph (e.g.\, weight tying).
This means the layer will not process a single vector $\vz^{(l)}$, but a sequence of vectors $\{ \vz^{(l)}_1, \dots, \vz^{(l)}_S \}$ where $S$ denotes weight sharing number.
We can column-stack these vectors
and obtain
\begin{align}
  \begin{pmatrix}
    \vz^{(l)}_1
    &
    \cdots
    &
    \vz^{(l)}_S
  \end{pmatrix}
  &=
    %\begin{pmatrix}
      \mW^{(l)} %& \vb
    %\end{pmatrix}
    \begin{pmatrix}
      \vz^{(l-1)}_1
      &
      \cdots
      &
        \vz^{(l-1)}_S
      %\\
      %1 & 0 & 0
    \end{pmatrix}.
    \label{eq:forward-laplacian-linear-layer-compact}
\end{align}
The parameter Jacobian is consequently given by the
\begin{align}
    \jac_{\mW^{(l)}} \mZ^{(l)}
  &=
    %\begin{pmatrix}
      %\mW^{(l)} %& \vb
    %\end{pmatrix}
    \begin{pmatrix}
      \vz^{(l-1)}_1 \otimes \mI
      &
      \cdots
      &
        \vz^{(l-1)}_S\otimes \mI
      %\\
      %1 & 0 & 0
    \end{pmatrix}.
    \label{eq:forward-laplacian-linear-layer-compact}
\end{align}
Using the notation $\vg_s\coloneqq \nabla_{\vz^{(l)}_s} \vz^{L}_s$ the Gauss-Newton matrix is given by
\begin{align}
    \mF(\mW^{(l)}) = {\jac_{\mW^{(l)}} \mZ^{(L)}}^\top \jac_{\mW^{(l)}} \mZ^{(L)}
\end{align}

into a matrix $\mZ^{(l)} \in \sR^{D_{\text{in}}\times S}$, likewise for the linear layer's outputs $\mZ^{(l+1)} = \mW^{(l)} \mZ^{(l)} \in \sR^{D_{\text{out}}\times S}$ and activation gradients $\mG^{(l)} \in \sR^{D_{\text{out}} \times S}$.\todo{introduce the gradient notation}
The output-weight Jacobian of a weight-sharing layer is $\jac_{\mW^{(l)}} \mZ^{(l+1)} = {\mZ^{(l)}}^{\top} \otimes \mI$ \cite[see e.g.][]{dangel2020modular} and the Fisher does not simplify into a Kronecker product without further approximations.
%As described by \citet{eschenhagen2023kroneckerfactored}, there are two possible Kronecker approximations for this setup.
We will focus on the \emph{expand} approximation, which yields the Kronecker approximation for convolutional layers proposed in~\citet{grosse2016kroneckerfactored}.
It treats the shared axis like a batch axis,
\begin{align}
    \mF(\mW^{(l)}) = \frac1S \sum_{s=1}^S (\vz_s^{(l)} \otimes {\vg_s^{(l)}}^\top)({\vz_s^{(l)}}^{\top}\otimes \vg_s^{(l)}) \approx \frac{1}{S} \sum_{s=1}^S \vz_s \vz_s^{\top} \otimes \sum_{s=1}^S \vg_s \vg_s^{\top}. %, %\quad \text{where } \vg_s = ...%\grad{\vz_s} \ell(\vx, \hat{\vy}, \mW).
\end{align}
We can express this in matrix notation as $\mF(\mW^{(l)}) \approx \nicefrac{1}{S} \mZ^{(l)} {\mZ^{(l)}}^{\top} \otimes \mG^{(l)} {\mG^{(l)}}^{\top}$.
\end{comment}
%\subsection{Discussion and generalizations}

\subsection{KFAC for Gauss-Newton matrices involving general PDE terms} \label{sec:KFAC-general}
%\paragraph{Setting}
We consider the general %second-order
PDE
\begin{equation}
    %\Psi(u, \nabla u, \nabla^{2} u) = 0,
    \Psi(u, \partial u, \dots, \partial^{(k)} u) = 0,
\end{equation}
where $\Psi\colon \mathbb R^{K}\to\mathbb R^M,K=\binom{d+k}{d}$ is smooth.
%We write $\mathcal D u \coloneqq (u, \partial u, \dots, \partial^{(k)} u)$
%Set $v_\vtheta(x)\coloneqq (u_\vtheta(\vx), \nabla u_\vtheta(\vx), \nabla^{2} u_\vtheta(\vx))$
%and
By $r_\vtheta(\vx)\coloneqq \Psi(u(\vx), \partial u(\vx), \dots, \partial^{(k)} u(\vx))%v_\vtheta(x))
$ %we denote the residual and for suitable integration points $\vx_n$ we
we denote the residual and consider the PINN loss
\begin{equation}
    %\frac1N\sum_{n=1}^N \ell(\Psi(\mathcal Du_\vtheta(\vx_n)),
    L(\vtheta)\coloneqq \frac1N\sum_{n=1}^N \ell(r_{\vtheta}(\vx_n)),
\end{equation}
where $\ell\colon\mathbb R^K\to\mathbb R$ is a smooth convex function with definite Hessian $\nabla^2\ell\succ0$ that typically has a unique minimizer at $0$.
We consider the Gauss-Newton matrix
\begin{align}
    \mG(\vtheta) & \coloneqq \frac1N\sum_{n=1}^N \jac_\vtheta r_\vtheta(\vx_n)^\top \mLambda(r_\vtheta(\vx_n)) \jac_\vtheta r_\vtheta(\vx_n),
    %\\ & = \frac1N\sum_{n=1}^N \jac_\vtheta v_\vtheta(\vx_n)^\top \underbrace{\jac \Psi(v_\vtheta(\vx_n))^\top \mLambda(r_\vtheta(\vx_n)) \jac \Psi(v_\vtheta(\vx_n))}_{\eqqcolon \mA(\vtheta, \vx)}  \jac_\vtheta v_\vtheta(\vx_n),
\end{align}
where $\mLambda(r)\coloneqq \nabla^2 \ell(r)$ denotes the Hessian of the per-sample loss. %~\citep{eschenhagen2023kroneckerfrwtactored} yxxx
%and $v_\vtheta$ is a forward network with shared weights.

Typically, in PINNs, one chooses $\ell = \frac12\lVert \cdot \rVert_2^2$ to be the squared Euclidean distance, such that $\mLambda = \mI$.
Note, however, that in contrast to a regression problem, the residual $r_\vtheta$ involves PDE terms and not only function evaluations.
If we have a forward iteration for $r_\vtheta$, we can apply existing KFAC approximations.
We obtain this via higher-order forward mode automatic differentiation.

Writing $v_\vtheta(\vx)\coloneqq (%u_\vtheta(\vx), \nabla u_\vtheta(\vx), \nabla^{2} u_\vtheta(\vx)
u(\vx), \partial u(\vx), \dots, \partial^{(k)} u(\vx))$
we can express the block corresponding to the $l$-th layer of the Gauss-Newton matrix as
\begin{align}
    \mG(\mW^{(l)}) & = \frac1N\sum_{n=1}^N \jac_{\mW^{(l)}} v_\vtheta(\vx_n)^\top \underbrace{\jac \Psi(v_\vtheta(\vx_n))^\top \mLambda(r_\vtheta(\vx_n)) \jac \Psi(v_\vtheta(\vx_n))}_{\eqqcolon \mLambda_n}  \jac_{\mW^{(l)}} v_\vtheta(\vx_n),
\end{align}
and $v_\vtheta$ is a forward network with shared weights.
%The reduce approximation is then given by
%\begin{align}
%    \hat{\mG}(\mW^{(l)}) \coloneqq \left(\sum_{n=1}^N \left( \sum_{s=1}^S \vz^{(l)}_{n,s} \right) \left(\sum_{s=1}^S {\vz^{(l)}_{n,s}}^\top\right)\right)\otimes\left(\sum_{n=1}^N\left(\sum_{s=1}^S \vg^{(l)}_{n,s} \right) \mH(\vtheta, \vx_n) \left( \sum_{s=1}^S {\vg^{(l)}_{n,s}}^\top \right) \right)?
%\end{align}

Using common multi-index notation, the forward pass in~\eqref{eq:forward_pass} extends to higher-order partial derivatives
\begin{align}
    \partial_\vx^\alpha \vz^{(l)} = \mW^{(l)}\partial_\vx^\alpha \vz^{(l-1)}
\end{align}
and hence, we can perceive the computational graph of $v_\vtheta$ as a feedforward network with weight sharing over the multi-indices $\alpha\in\mathbb N$ with $\lvert \alpha \rvert \le k$.
The generalization of%the approximation
~\eqref{eq:KFAC-PINN} is given by
\begin{tcolorbox}[colframe=kfac, title={KFAC with general PDE terms},bottom=0mm,top=0mm,middle=0mm]
\begin{align}\label{eq:KFAC-PINNs-general}
    \hat{\mG}%^{\textup{exp}}
    (\mW^{(l)})
    \coloneqq \frac{1}{N^2\binom{d+k}{d}}
    \left[\sum_{n=1}^N \sum_{\lvert \alpha \rvert \le k} %\mZ^{(l-1)}_{n,\alpha}{\mZ^{(l-1)}_{n,\alpha}}^\top
    \partial_\vx^\alpha \vz_n^{(l-1)} {\partial_\vx^\alpha \vz_n^{(l-1)}}^\top\right]
    \otimes
    \left[\sum_{n=1}^N\sum_{\lvert \alpha \rvert \le k} {\mJ^{(l)}_{n,\alpha}}^\top \mLambda_n \mJ^{(l)}_{n,\alpha} \right],
\end{align}
\end{tcolorbox}
%where $\mZ^{(l)}_{n,\alpha} = \partial_\vx^\alpha \vz_n^{(l)}$ and
where $\mJ^{(l)}_{n,\alpha} = %\nabla
\jac_{\partial_\vx^\alpha \vz_n^{(l)}} \vv_n$ and $\vv_n = v_\vtheta(\vx_n) %= (\partial^\beta_\vx \vz_{n}^{L})_{\lvert \beta \rvert \le k}
$.
A detailed derivation can be found in \Cref{app:derivations}.
%
This approximation can be stated for a large class of weight-sharing architectures in which case it generalizes the \emph{expand approximation} in~\citet{eschenhagen2023kroneckerfactored}.
%\jm{Note that other Kronecker-factored approximations for weight-sharing architectures have been proposed, but due to the superior performance in our experiments, we focus on the expand approximation?}

%\todo[inline]{striking remark how that this can be applied for any Gramian involving PDE terms}
%There are some challenges we need to overcome to define a Kronecker-factored approximation of the Gramian from \Cref{eq:gramian}:
%\begin{itemize}
%\item The Gramian of the interior loss involves the parameter gradient of the Laplacian. To establish a Kronecker approximation, we need to know how the weight matrix of a layer enters the computation of the Laplacian. We will show that, when using the forward Laplacian framework from \cite{li2023forward}, the weight matrix enters the computation by multiplication onto another matrix. This is great, because we know how to define KFAC in such cases, thanks to the KFAC-for-weight-sharing framework developed by \citet{eschenhagen2023kroneckerfactored}.

%\item %The PINN loss consists of two terms, the interior and boundary loss.
  %We can develop a Kronecker approximation for both individually, which leaves us with the problem how to work with both terms to pre-condition a gradient.
  %One way is to keep the both Kronecker approximations separate, but then we need to invert a sum of two Kronecker products.
  %This can be done, see \Cref{sec:inverse_kronecker_sum}, but adds twice the memory overhead compared to having a single Kronecker approximation.
  %Also, the Kronecker sum's inversion requires solving a generalized eigenvalue problem, for which there is currently no API in PyTorch.
  %Hence we need to fall back to SciPy, which costs communication overhead because everything needs to be off-loaded to CPU.
  %Alternatively, we could summarize the two Kronecker approximations into a single one at the risk of losing downstream performance.
%\end{itemize}

\paragraph{KFAC for variational problems}
Our proposed KFAC approximation is not limited to PINNs and can be used for variational problems of the form
\begin{align}
    \min_u \int_\Omega \ell(u, \partial u, \dots, \partial^{(k)} u) \mathrm{d}x,
\end{align}
where $\ell\colon\mathbb R^K\to\mathbb R$ is a convex function.
%The discretization of this is given by
%\begin{align}
%    L(\vtheta) = \frac1N \sum_{n=1}^N \ell(u(\vx_n), \partial u(\vx_n), \dots, \partial^{(k)}u(\vx_n)),
%\end{align}
%which covers general PINNs but also variational problems.
%Then the Gauss-Newton matrix is given by
We can perceive this problem as a special case of the setting described above with $\Psi = \operatorname{id}$ and hence the KFAC approximation~\eqref{eq:KFAC-PINNs-general} remains meaningful %approximation of the discretized Gauss-Newton matrix
if $\mLambda_n$ is replaced by $\mLambda_n = \nabla^2\ell(u(\vx_n), \partial u(\vx_n), \dots, \partial^{(k)}u(\vx_n))$.
In particular, our approximation can be used for the \emph{deep Ritz method} and other variational approaches to solve PDEs~\citep{yu2018deep}.

\begin{comment}
    \subsection{Computational complexity}
\todo[inline]{what can we say? probably better to move to the respective approximations}
\begin{itemize}
    \item GN: $O(p^3+Np^2)$
    \item BFGS: $O(p^2+?)$
    \item L-BFGS: $O(mp+?)$?
    \item KFAC: $O(?)$
    \item
\end{itemize}
\end{comment}

\subsection{Algorithmic Details}

We provide the additional components that form our KFAC-based optimization algorithms, which are influenced by ENGD and the original KFAC algorithm.
See \Cref{app:pseudo} for pseudo-code.

\paragraph{Exponential moving averages and damping} %Given a combination of interior and condition loss, $\gL_{\Omega},\gL_{\partial\Omega}$ we
First, we approximate the interior and boundary Gramians at iteration $t$  according to~\eqref{eq:KFAC-PINN} and~\eqref{eq:KFAC-PINNs-general} and the classic KFAC approximation, respectively, giving
$\mG_{\Omega,t}^{(l)} \approx \mA_{\Omega,t}^{(l)}\otimes \mB_{\Omega,t}^{(l)}$ and $\mG_{\partial \Omega, t}^{(l)}\approx \mA_{\Omega,t}^{(l)}\otimes \mB_{\Omega,t}^{(l)}$ .
Here, we replace $\mA^{(l)}_{\bullet,t}$ with an exponential moving average $\hat{\mA}^{(l)}_{\bullet,t} = \beta \hat{\mA}^{(l)}_{\bullet,t-1} + (1 - \beta) \mA^{(l)}_{\bullet,t}$
and identically for $\hat{\mB}^{(l)}_{\bullet, t}$ for some $\beta\in[0,1)$.
Further, we add constant damping of strength $\lambda>0$ to all Kronecker factors, obtaining $\tilde{\mA}_{\bullet,t} = \hat{\mA}^{(l)}_{\bullet,t} + \lambda \mI$ and $\tilde{\mB}_{\bullet,t} = \hat{\mB}^{(l)}_{\bullet,t} + \lambda \mI$
leading to the
 block-diagonal Kronecker-factored approximation
\begin{align*}
  \mG_{\bullet, t}
  &\approx
    \blockdiag
    \left(
    \tilde{\mA}^{(1)}_{\bullet,t} \otimes \tilde{\mB}_{\bullet,t}^{(1)},
    \dots,
    \tilde{\mA}^{(L)}_{\bullet,t} \otimes \tilde{\mB}_{\bullet,t}^{(L)}
    \right)
    \qquad \bullet \in \{ \Omega, \partial\Omega\}\,.
\end{align*}
% over all previous Kronecker factors, i.e.\, as defined in ?? and identically for $\hat{\mB}^{(l)}_{\bullet, t}$.

\paragraph{Gradient pre-conditioning% and damping
}
Given the current mini-batch gradient $\vg^{(l)}$ for layer $l$, we obtain the update direction $\vDelta_t^{(l)} = -(\tilde{\mA}_{\Omega} \otimes \tilde{\mB}_{\Omega} + \tilde{\mA}_{\partial\Omega} \otimes \tilde{\mB}_{\partial\Omega})^{-1} \vg^{(l)}$ by
pre-condition with our Kronecker-factored approximate Gramian.
%First, we add constant damping of strength $\lambda>0$ to all Kronecker factors, obtaining $\tilde{\mA}_{\bullet,t} = \hat{\mA}^{(l)}_{\bullet,t} + \lambda \mI$ and $\tilde{\mB}_{\bullet,t} = \hat{\mB}^{(l)}_{\bullet,t} + \lambda \mI$.
%Then, we multiply the inverse of the sum of Kronecker products onto the gradient to obtain the update direction
This can be done without building up the full matrix utilizing a generalized eigendecomposition of the individual Kronecker factors, where we use the procedure described in \cite[Appendix I]{martens2015optimizing}.

%\paragraph{Combining PDE and boundary approximations}
%The boundary Gramian matrix $\mG_{\partial\Omega}(\mW^{(l)})$ is a classic Gauss-Newton matrix and thus we can use a classic KFAC approximation $\hat{\mG}_{\partial\Omega}(\mW^{(l)})$.
%To invert the sum $\hat{\mG}_\Omega(\mW^{(l)}) + \hat{\mG}_{\partial\Omega}(\mW^{(l)})$ of the two Kronecker approximations we use the procedure described in \cite[Appendix I]{martens2015optimizing}.

\paragraph{Learning rate and momentum}
We consider two different updates $ \vtheta_{t+1} = \vtheta_t + \vdelta_{t}$ at iteration $t$ from the pre-conditioned gradient $\vDelta_t$, which we call as `KFAC' and `KFAC*'.
KFAC uses momentum %buffer of all previous updates,
$\hat{\vdelta}_t = \mu \vdelta_{t-1} + \vDelta_t$ %(similar to momentum in SGD).
%The
where the parameter $\mu$ can be chosen by the practitioner.
Like in ENGD, we use a logarithmic grid line search along
%$\hat{\vdelta}_t$, i.e.\,
%choosing
leading to the update
$\vdelta_t = \alpha_{\star} \hat{\vdelta}_t$ where
$\alpha_{\star} = \argmin_{\alpha} \gL(\vtheta_t + \alpha \hat{\vdelta}_t)$ where $\alpha \in \{2^{-30}, \dots, 2^0\}$.
KFAC* uses the learning rate and momentum heuristic proposed for the original KFAC optimizer~\citep{martens2015optimizing}.
It parameterizes the iteration's update as $\vdelta_{t+1}(\alpha, \mu) = \alpha \vDelta_t + \mu \vdelta_t$, then obtains the optimal parameters by minimizing the quadratic model $m(\vdelta_{t+1}) = \gL(\vtheta_t) + \vdelta_{t+1}^{\top} \nabla_{\vtheta_t}\gL(\vtheta_t) + \nicefrac{1}{2}\vdelta_{t+1}^{\top} (\mG(\vtheta_t) + \lambda \mI) \vdelta_{t+1}$ with the true damped Gramian.
The optimal learning rate and momentum $\argmin_{\alpha, \mu} m(\vdelta_t)$ are given by~(see \citep[][Section 7]{martens2015optimizing} for details)
\begin{align}
  \begin{pmatrix}
    \alpha^{\star} \\ \mu^{\star}
  \end{pmatrix}
  =
  -
  \begin{pmatrix}
    \vDelta_t^{\top} \mG \vDelta_t + \lambda \left\lVert \vDelta_t \right\rVert^2
    & \vDelta_t \mG \vdelta_t + \lambda \vDelta^{\top}_t \vdelta_t
    \\
    \vDelta_t \mG \vdelta_t + \lambda \vDelta^{\top}_t \vdelta_t
    &
      \vdelta_t^{\top} \mG \vDelta_t + \lambda \left\lVert \vdelta_t \right\rVert^2
  \end{pmatrix}^{-1}
  \begin{pmatrix}
    \vDelta_t \nabla_{\vtheta_t} \gL
    \\
    \vdelta_t \nabla_{\vtheta_t} \gL
  \end{pmatrix}\,.
\end{align}
The computational cost is dominated by the two Gramian-vector products $\mG \vDelta_t$ and $\mG \vdelta_t$ which can be performed with automatic differentiation~\citep{pearlmutter1994fast,schraudolph2002fast}.

%%% Local Variables:
%%% mode: latex
%%% TeX-master: "../main"
%%% End:


\section{Experiments}\label{sec:experiments}

\begin{figure}[tb]
  \centering
  \includegraphics{../kfac_pinns_exp/exp17_groupplot_poisson2d/loss.pdf}
  \caption{Scaling behaviour of different optimizers for learning the solution of a 2d-Poisson equation w.r.t.
    neural network size under a given time budget of $10^3\,\text{s}$ on an RTX 6000 GPU.}
  \label{fig:pedagogical-example}
\end{figure}

\subsection{Setup}

Small-scale example (2D Poisson?)
\begin{itemize}
    \item Tested optimizers: Adam, GD with LS?, Newton's method / (L-)BFGS, Full ENGD, (Block)-Diagonal approximation for ENGD?, KFAG, other KFACs: for the l2-GN, empirical Fisher, Quantum KFAC
    \item damping: try both constant and adaptive damping
    \item try without line search with learning rate schedule
\end{itemize}

\begin{itemize}
    \item use weights and biases for large scale experiments
    \item next steps: modular code
\end{itemize}

\subsection{The Poisson equation}

\subsection{asdf}

We want to show the following things:
\begin{itemize}
\item We can safely discard the Gramian's off-diagonal blocks without harming
  training performance. This reduces the Gramian's size, but still imposes
  strong constraints on scalability.

\item Our proposed Kronecker approximation works roughly as well as the
  full/block diagonal Gramian, while being much cheaper to compute, store, and
  invert.

\item Thanks to the Kronecker approximation of the Gramian, we can scale to larger neural networks where the other methods either do not work (storing the Gramian is prohibitively expensive) or become quite slow (matrix-free linear system solve via Gramian-vector products).
\end{itemize}

Todos:
\begin{itemize}
\item concrete example ground truth: 2d Poisson on unit square with sine target
\end{itemize}

Ideas:
\begin{itemize}
\item try out different approximations
  \begin{itemize}
  \item Ground truth
  \item Block diagonal exact
  \item Diagonal
  \item Block diagonal with different approximations
  \end{itemize}
\end{itemize}

%%% Local Variables:
%%% mode: latex
%%% TeX-master: "../main"
%%% End:


\section{Discussion and Conclusion}\label{sec:conclusion}
\paragraph{Limitations:}
\begin{itemize}
\item Only works for sequential networks
%\item Hessian backpropagation requires memory quadratic in the intermediate features size and therefore becomes impractical for large intermediates like in CNNs.
\item Our implementation is likely manual and relatively hard to automate with the current graph inspection tools offered by ML frameworks.
\end{itemize}



%%% Local Variables:
%%% mode: latex
%%% TeX-master: "../main"
%%% End:


\begin{ack} % automatically suppressed in anonymized submission
  The authors thank Runa Eschenhagen for insightful discussions on the relation to KFAC for weight sharing layers.
\end{ack}
%%% Local Variables:
%%% mode: latex
%%% TeX-master: "../main"
%%% End:


\bibliography{references}
\bibliographystyle{icml2024.bst}

\clearpage
\appendix

% Use different numbering in appendix to make it clear that a
% section/table/algorithm is not in the main text
\renewcommand\thefigure{\thesection\arabic{figure}}
\renewcommand\thetable{\thesection\arabic{table}}
\renewcommand{\theequation}{\thesection\arabic{equation}}

% modified title header from main page, code extracted from NeurIPS template
\makeatletter
\vbox{%
  \hsize\textwidth
  \linewidth\hsize
  \vskip 0.1in
  \@toptitlebar
  \centering
  {\LARGE\bf \@title (Supplementary Material)\par}
  \@bottomtitlebar
  \vskip 0.3in \@minus 0.1in
}
\makeatother

% APPENDIX TOC
\startcontents[sections]
\printcontents[sections]{l}{1}{\setcounter{tocdepth}{2}}
\vspace{2em}
%%% Local Variables:
%%% mode: latex
%%% TeX-master: "../main"
%%% End:


\input{sections/checklist.tex}

\end{document}
%%% Local Variables:
%%% mode: latex
%%% TeX-master: t
%%% End:

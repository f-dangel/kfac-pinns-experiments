We work with the following standard feedforward networks.

\paragraph{Sequential neural networks} Consider a \emph{sequential neural network} $%u_{\vtheta} = f^{(L)}_{\vtheta^{(L)}} \circ f^{(L-1)}_{\vtheta^{(L-1)}} \circ \ldots \circ f^{(1)}_{\vtheta^{(1)}}
u_{\vtheta} = f_{\vtheta^{(L)}} \circ f_{\vtheta^{(L-1)}} \circ \ldots \circ f_{\vtheta^{(1)}}
$ with depth $L\in\mathbb N$ that consists of layers %$f^{(l)}_{\vtheta^{(l)}}$
\begin{align}
  \begin{split}
    %u_{\vtheta}
    %&=
    %f^{(L)}_{\vtheta^{(L)}} \circ f^{(L-1)}_{\vtheta^{(L-1)}} \circ \ldots \circ f^{(1)}_{\vtheta^{(1)}}
    %\\
    f%^{(l)}
    _{\vtheta^{(l)}}\colon \sR^{h^{(l-1)}}
    &\to
      \sR^{h^{(l)}}\,,
    %\\
    \quad \vz^{(l-1)}
    %&
    \mapsto
      \vz^{(l)} = f%^{(l)}
      _{\vtheta^{(l)}}(\vz^{(l-1)})
  \end{split}
\end{align}
with trainable parameters $\vtheta^{(l)} \in \sR^{p^{(l)}}%, l=1,\dots, L,
$ that transform an input $\vz_0=\vx \in \mathbb R^{d_{\operatorname{in}}}$ into a prediction $\vu = \vz^{(L)}\in \sR^{d_{\operatorname{out}}}$ via intermediate representations $\vz^{(l)} \in \sR^{h^{(l)}}%, l= 0, \dots, L
$.
%where $\vz^{(0)} \coloneqq \vx$, $\vz^{(L)} \coloneqq \vu$, and
By $\vtheta = ({\vtheta^{(1)}}^{\top}, \dots, {\vtheta^{(L)}}^{\top})^{\top}$ we denote the concatenation of the parameters of the individual layers.
% A parameter might be empty, e.g.\,if the layer is an activation layer.
Here, we consider networks with alternating  linear and nonlinear layers.
Here, the weights of the linear layers are trainable and we denote them by $\vtheta^{(l)} = \mW^{(l)}$, where the layer is given by $\vz^{(l)} = \mW^{(l)} \vz^{(l-1)}$.
The nonlinear layers are given by $\vz\mapsto \sigma(\vz)$ componentwise for a (typically smooth) activation function $\sigma\colon\mathbb R\to\mathbb R$.
%We denote the parameters of a linear layer by $\vtheta^{(l)} = \mW^{(l)}$ where the layer is given by $\vz^{(l)} = \mW^{(l)} \vz^{(l-1)}$.

\paragraph{Flattening, Jacobians, and Hessians}
%Above, we assumed all quantities ($\vz^{(l)}, \vtheta^{(l)}$) to be vectors.
%In case of tensor-valued quantities, we can first flatten them into vectors to reduce to the vector case.
%We consider the following vectorization of the matrices appearing as the trainable parameters of the network.
To vectorize the weight matrices of the linear layers we follow the % index convention to vectorize will be
\emph{first-varies-fastest} convention, which means column-stacking (row index varies first, column index varies second).
We denote the corresponding flattening operation by $\flatten(\cdot)$.
Very useful is the so-called \emph{vec-trick} stating that $\flatten(\mA\mX\mB) = (\mB^\top\otimes \mA)\flatten{\mX}$
%\begin{equation}\label{eq:vecTrick}
%  \flatten(\mA\mX\mB) = (\mB^\top\otimes \mA)\flatten{\mX}
%\end{equation}
for matrices $\mA, \mX, \mB$.
In particular, this shows that $\mB^\top\otimes \mA$ is the matrix representing the linear mapping $\mX\mapsto \mA\mX\mB$ and hence $\jac_\mX(\mA\mX\mB) = \mB^\top\otimes \mA$.
The flattening notation allows to reduce derivatives of matrix/tensor-valued objects back to the matrix case.
Consider the Jacobian $\jac_{\va}\vb$ of a vector $\vb$ w.r.t.\,a vector $\va$.
It collects all partial derivatives as $[\mJ_{\va}\vb]_{i,j} = \nicefrac{\partial [\vb]_i}{\partial [\va]_j}$.
For the Jacobian $\jac_{\mA}\mB$ of a matrix $\mB$ w.r.t.\,a matrix $\mA$, we simply have $\jac_{\mA} \mB = \jac_{\flatten( \mA )}\flatten(\mB)$.
Likewise, the Hessian $\gradsquared{\va}b$ of a scalar $b$ w.r.t.\,a vector $\va$ collects the second-order partial derivatives according to $[\gradsquared{\va}b]_{i,j} = \nicefrac{\partial^2 b}{\partial [\va]_i \partial [\va]_j}$.
For the Hessian $\gradsquared{\mA} b$ of a scalar $b$ w.r.t.\,a matrix $\mA$, we simply have $\gradsquared{\mA} b = \gradsquared{\flatten(\mA)}b$.
We also have $\grad{\mA} b = \grad{\flatten(\mA)} b$ for the gradient of a scalar w.r.t.\,a matrix.





\subsection{Physics-informed neural networks (PINNs)}
%\begin{itemize}
%    \item PINNs are one of the most promising approaches to neural network based PDE solvers
%    \item
%\end{itemize}
Let us consider a domain $\Omega\subseteq\mathbb R^d$ and the Poisson equation %and how to write the Gramian/Fisher as the product of a Jacobian and its transpose, see equation \eqref{eq:Jacobian_Fischer}.
%Is this what you were after Felix?
%I think it differs from the standard setting in the sense that suddenly the Laplacian shows up.
%It can also be used to derive a matrix-free version of the energy natural gradient descent, like done in Hessian-free optimization.
%$\bullet$ We consider the Poisson equation
\begin{align*}\tag{PE}\label{eq:PE}
  -\mathcal{L} u & = f \quad \text{in }\Omega \\
  u & = g \quad \text{on }\partial\Omega
\end{align*}
where $f\in L^2(\Omega)$ and $g\in H^{3/2}(\partial\Omega)$ are some given right hand side and boundary values.
Here, $mathcal{L}$ denotes a differential operator, e.g., the \emph{Laplacian} $\mathcal{L} u = \Delta u = \sum_i \partial_{\vx_i}^2 u$.
%Let us consider the function space energy
%  \[ E(u) = \frac{1}{2} \int_\Omega (\Delta u + f)^2 \mathrm dx + \frac12 \int_{\partial\Omega} (u-g)^2 \mathrm ds, \]
%which is designed such that $u$ is a solution of~\eqref{eq:PE} if and only if $E(u)=0$.
%Hence, we discretize the integral and use the following loss function t
In \emph{Physics-informed neural networks (PINNs)}, we parametrize $u$ using a neural network and train its parameters $\vtheta$ using the loss
  %\[ L(\theta) \coloneqq E(u_\theta) = \frac{1}{2} \int_\Omega (\Delta u_\theta + f)^2 \mathrm dx + \frac12 \int_{\partial\Omega} (u_\theta-g)^2 \mathrm ds. \]
%Indeed, one can show that $L$ controls the error, i.e., $\lVert u_\theta - u^\star \rVert_{H^{1/2}(\Omega)} \le c_{\operatorname{reg}} \sqrt{L(\theta)}$, where $c_{\operatorname{reg}}$ denotes the regularity constant of the PDE, see~\cite{?}.
%In practice, we have to discretize the integrals in the PINN loss, and obtain the empirical loss function %with discretized integrals, is
\begin{align}
  L(\vtheta)
  &=
    %\frac{1}{2N_\Omega}
    %L_\Omega(\vtheta) + %\frac{1}{2N_{\partial\Omega}}
    %L_{\partial\Omega}(\vtheta)
  %\\&
  %=
    \underbrace{\frac{1}{2N_\Omega} \sum_{i=1}^{N_\Omega} (\mathcal{L} u_\vtheta(\vx_i) + f(\vx_i))^2}_{\eqqcolon L_\Omega(\vtheta)} + \underbrace{\frac{1}{2N_{\partial\Omega}}\sum_{i=1}^{N_{\partial\Omega}} ( u_\vtheta(\vx^b_i) - g(\vx^b_i))^2}_{\eqqcolon L_{\partial\Omega}(\vtheta)},
\end{align}
where $(x_n)_{n=1}^{N_\Omega}$ and  $(x^b_n)_{n=1}^{N_{\partial\Omega}}$ are points in the interior and on the boundary of $\Omega$ respectively.
%Whereas we choose the Poisson equation for the sake of simplicity, this approach can easily be generalized to other PDEs. 
%The Kronecker-factored approximations derived in~\Cref{?} are formulated for general PDEs. 
%The number and choice of the points can be controlled by the practitioner, where different sampling strategies have been developed~\cite[text]{keylist}.
%This approach is simple and intuitive and is usually associated with the possibility to work well for high-dimensional PDEs.
%However, it is well documented in the literature that first-order optimizers fail to produce satisfactory accuracy even for low-dimensional problems~\cite{}.

%\begin{itemize}
%    \item If you run GD / Adam you get no good accuracy~\cite{?}
%    \item L-BFGS improves things, but is not super satisfactory
%    \item recently, function space inspired methods and preconditioners have been used with promising success~\cite[text]{CPINNs, ENG, preconditioning?, FS paper, Navier-Stokes}; we describe the approach we use here in more detail
%\end{itemize}

%%% Local Variables:
%%% mode: latex
%%% TeX-master: "../main"
%%% End:

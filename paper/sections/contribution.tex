
The overall Gauß-Newton matrix is given by the sum $\mG(\vtheta)= \mG_\Omega(\vtheta) + \mG_{\partial\Omega}(\vtheta)$ of two Gramian matrices stemming from the PDE and the boundary term, respectively.
A Kronecker factored approximation of the boundary Gramian $\mG_{\partial\Omega}(\vtheta)$ can be obtained just like the case of the  Fisher $\mF(\vtheta)$.
The interior Gramian $\mG_\Omega(\vtheta)$, which involves PDE terms, requires a different approximation that we provide in this section.
For this, we use forward mode automatic differentiation to express the computation of higher-order derivatives as forward passes of a larger network with shared weights, for which Kronecker approximations have been developed by~\citet{eschenhagen2023kroneckerfactored}.
The two individual Kronecker-factored approximations can either be combined into a single Kronecker-factored approximation or inverted using an explicit formula based on the individual inversions.

\subsection{Higher-order forward mode automatic differentiation}
\label{sec:taylor-mode-AD}

Here, we review higher-order forward mode differentiation, also known as \emph{Taylor-mode automatic differentiation}~\citep{griewank1996algorithm, griewank2008evaluating, bettencourt2019taylor}.
Many PDEs only incorporate first and second partial derivatives and we focus our discussion here on second-order automatic differentiation for the sake of simplicity of the presentation.
However, a completely analogous approach can be taken for higher-order PDEs.

How to compute the function values through a forward pass is well known.
The computation of the gradient through forward mode differentiation uses the chain rule $\jac (f\circ g) = \jac f (g) \jac g$.
If we want to compute the Hessian in a forward-mode style, we use the \emph{Hessian chain rule}, which is given by
\begin{align}
       %\nabla^2 (f\circ g) & = (\jac g)^\top\cdot \nabla^2 f(g) \cdot \jac g + \sum_{k} \partial_{k} f(g) \cdot \nabla^2 g_k \\
       \partial_{\vx_i} \partial_{\vx_j} (f \circ g)(\vx) & = \partial_{\vx_i} g(\vx)^\top \nabla^2f(g(\vx))\partial_{\vx_j} g(\vx) + \jac f(g(\vx)) \partial_{\vx_i} \partial_{\vx_j} g(\vx).
\end{align}
If $f_{\vtheta^{(l)}}(\vz) = \mW^{(l)} \vz %+ \vb^{(l)}
$ is a linear layer, then setting $z^{(0)} = x\in\mathbb R^d, \partial_{x_i}z^{(0)} = \textrm{e}_i\in\mathbb R^d$, i.e., the $i$-th basis vector and $\partial_{x_i}\partial_{x_j}z^{(0)} = 0\in\mathbb R^d$, the second-order forward pass is given by
\begin{subequations}
    \begin{align}
    \vz^{(l)} & = \mW^{(l)} \vz^{(l-1)} %+ \vb^{(l)}
    \\
    \partial_{\vx_i} \vz^{(l)} & = \mW^{(l)} \partial_{\vx_i} \vz^{(l-1)} \\
    \partial_{\vx_i}\partial_{\vx_j} \vz^{(l)} & = \mW^{(l)} \partial_{\vx_i}\partial_{\vx_j} \vz^{(l-1)} \label{subeq:secondOrderForward-LinearLayer}.
\end{align}
\end{subequations}
The second-order forward pass through a nonlinear layer $\vz\mapsto \sigma(\vz)$ is given by
\begin{subequations}
    \begin{align}
    \vz^{(l)} & = \sigma(\vz^{(l-1)}) \\
    \partial_{\vx_i} \vz^{(l)} & = \sigma'(\vz^{(l-1)}) \odot \partial_{\vx_i} \vz^{(l-1)} \\
    \partial_{\vx_i}\partial_{\vx_j} \vz^{(l)} & = \partial_{\vx_i} \vz^{(l-1)} \odot \sigma''(\vz^{(l-1)}) \odot \partial_{\vx_j} \vz^{(l-1)} + \sigma'(\vz^{(l-1)}) \odot \partial_{\vx_i} \partial_{\vx_j} \vz^{(l-1)} \label{subeq:secondOrderForward-nonlinearLayer}.
\end{align}
\end{subequations}



\paragraph{Forward Laplacian}
To compute not the full Hessian but only the Laplacian, we can simplify the forward pass to only propagate the Laplacian which is known as \emph{forward Laplacian}~\citep{li2023forward}. Introducing the vector $\mZ^{(l)}\coloneqq  \begin{pmatrix}
    \vz^{(l)}
    &
    \partial_{\vx_1} \vz^{(l)}
    &
    \dots
    &
    \partial_{\vx_d} \vz^{(l)}
    &
    %\partial_i\partial_j
    \Delta_\vx\vz^{(l)}
\end{pmatrix}\in \mathbb R^{(d+2)h^{(l)}}$ with $\mZ^{(0)}=  \begin{pmatrix}
    \vx
    &
    \ve_1
    &
    \dots
    &
    \ve_d
    &
    0
\end{pmatrix}\in \mathbb R^{(d+2)d}$ and summing~\eqref{subeq:secondOrderForward-LinearLayer} and~\eqref{subeq:secondOrderForward-nonlinearLayer} over $i=j$ we obtain the Laplacian forward pass for linear layers by

%We obtain the Laplacian forward pass by summing~\eqref{subeq:secondOrderForward-LinearLayer} and~\eqref{subeq:secondOrderForward-nonlinearLayer} over $i=j$ which yields
%\begin{subequations}
%    \begin{align}
%        \Delta \vz^{(l)} & = \mW^{(l)} \Delta \vz^{(l-1)} \quad \text{and } \\
%        \Delta \vz^{(l)} & = \sum_i \sigma''(\vz^{(l-1)}) \odot (\partial_{\vx_i} \vz^{(l-1)})^{\odot 2} + \sigma'(\vz^{(l-1)}) \odot \Delta \vz^{(l-1)}
%    \end{align}
%\end{subequations}
\begin{align}
  %\nonumber
  \mZ^{(l)}\coloneqq %\underbrace{
  \begin{pmatrix}
    \vz^{(l)}
    \\
    \partial_{\vx_1} \vz^{(l)}
    \\
    \vdots
    \\
    \partial_{\vx_d} \vz^{(l)}
    \\
    %\partial_i\partial_j
    \Delta_\vx\vz^{(l)}
  \end{pmatrix}
  %}_{\eqqcolon %\mT^f
  %\in \sR^{D_{\text{out}} \times (D+2)}
  %\mZ^{(l)}}
  & =
    %\begin{pmatrix}
      %\mW^{(l)} %& \vb^{(l)}
    %\end{pmatrix}
    %\underbrace{
    %\begin{pmatrix}
    %  \vz^{(l-1)}
    %  &
    %  \partial_{i} \vz^{(l-1)}
    %  &
    %    \partial_i\partial_j \vz^{(l-1)}
      %\\1 & 0 & 0
    %\end{pmatrix}
    %}_{\eqqcolon \mT^h \in \sR^{(D_{\text{in}} +1) \times (D+2)}}\,,
    %\shortintertext{or, in compact form,}
    %\mT^f
  %&=
    %\tilde{\mW}
    %\mT^h\,.
    \begin{pmatrix}
    \mW^{(l)}\vz^{(l-1)}
    \\
    \mW^{(l)}\partial_{\vx_1} \vz^{(l-1)}
    \\
    \vdots
    \\
    \mW^{(l)}\partial_{\vx_d} \vz^{(l-1)}
    \\
    %\partial_i\partial_j
    \mW^{(l)}\Delta_\vx\vz^{(l-1)}
  \end{pmatrix}
     %= \tilde{\mW}^{(l)}\mZ^{(l-1)},
     %\quad \mZ^{(l)} = F(\mZ^{(l-1)})
    \label{eq:forward-laplacian-linear-layer-compact}
\end{align}
and for nonlinear layers by
\begin{align}
    \mZ^{(l)}
    =
     \begin{pmatrix}
         \sigma(\vz^{(l-1)}) \\
         \sigma'(\vz^{(l-1)})\odot \partial_{x_1}\vz^{(l-1)} \\
         \vdots \\
         \sigma'(\vz^{(l-1)})\odot \partial_{x_d}\vz^{(l-1)} \\
         \sigma'(\vz^{(l-1)})\odot\Delta_x\vz^{(l-1)} + \sum_{i=1}^d \sigma''(\vz^{(l-1)})\odot \partial_{x_i}{\vz^{(l-1)}}^{\odot 2}
     \end{pmatrix}
     \label{eq:forward-laplacian-nonlinear-layer-compact}
\end{align}
This greatly reduces the computational complexity of the forward pass, but is restricted to PDEs that involve second-order derivatives only via the Laplacian. %, i.e., for PDEs of the form $\Psi(u, \partial u, \Delta u)=0$.
%In matrix form, higher-order forward mode automatic differentiation is given by
%or $\mT^f=\tilde{\mW}\mT^h$. \todo{do we need this notation?}
%for linear and nonlinear layers, respectively, where $\tilde{\mW}^{(l)} = \operatorname{diag}(\mW^{(l)}, \dots, \mW^{(l)})$ is a block diagonal matrix.
We can interpret this formula as a forward pass through a larger neural network with weight sharing in the spirit of \cite{eschenhagen2023kroneckerfactored}  by rearranging $\mZ^{(l)}$ as a matrix of shape $(d+2)\times h^{(l)}$ and noting that $\mZ^{(l)} = \mZ^{(l-1)}{\mW^{(l)}}^\top$.

\subsection{Kronecker-factored approximations for the Poisson equation}
For the Poisson equation, the individual block matrices of the PDE Gramian are given by
\begin{align}
    \mG_\Omega(\mW^{(l)}) = \frac1{{N_\Omega}} \sum_{n=1}^{N_\Omega} \jac_{\mW^{(l)}} \Delta_\vx \vu_n^\top \jac_{\mW^{(l)}} \Delta_\vx \vu_n
\end{align}
Just like in the case of the block Fisher matrices $\mF(\vtheta^{(l)})$ we use the chain rule and compute
\begin{align*}
    \jac_{\mW^{(l)}} \Delta_\vx \vu_n & = \jac_{\mZ^{(l)}}\Delta_\vx \vu_n \jac_{\mW^{(l)}} \mZ^{(l)} %=
    %\jac_{\mW^{(l)}} \mZ^{(l)} \jac_{\mZ^{(l)}}\mZ^{(L)}\jac_{\mZ^{(L)}}\Delta_\vx \vu_n.
    %\\ & =
    %\operatorname{diag}\left({\vz^{(l-1)}}^\top \otimes \mI, \dots, {\Delta_\vx\vz^{(l-1)}}^\top \otimes \mI\right) \jac_{\mZ^{(l)}}\Delta_\vx \vu_n
    %\begin{pmatrix}
    %    \nabla_{\vz^{(l)}} \Delta_{\vx}\vu_n \\
    %    \nabla_{\partial_{\vx_1}\vz^{(l)}}\Delta_{\vx}\vu_n \\
    %    \vdots \\
    %    \nabla_{\partial_{\vx_d}\vz^{(l)}} \Delta_{\vx}\vu_n  \\
    %    \nabla_{\Delta_{\vx}\vz^{(l)}} \Delta_{\vx}\vu_n \\
    %\end{pmatrix}^\top
    %\begin{pmatrix}
    %    {\vz^{(l-1)}_n}^\top \otimes \mI \\
    %    {\partial_{\vx_1}\vz^{(l-1)}_n}^\top \otimes \mI \\
    %    \vdots \\
    %    {\partial_{\vx_d}\vz^{(l-1)}_n}^\top \otimes \mI \\
    %    {\Delta_{\vx}\vz^{(l-1)}_n}^\top \otimes \mI \\
    %\end{pmatrix}
    \\ & = \sum_{s=1}^S (\nabla_{\mZ^{(l)}_s} \Delta_{\vx}\vu_n)^\top  ({\mZ^{(l-1)}_{n,s}}^\top \otimes \mI)?
    \\ & = \left(\sum_{s=1}^S \mZ^{(l-1)}_{n,s} \otimes \nabla_{\mZ^{(l)}_s} \Delta_{\vx}\vu_n\right)^\top,
    %{\vz^{(l-1)}_n}^\top \otimes \jac_{\vz^{(l)}} \Delta_{\vx}\vu_n
    %+
    %\sum_{i=1}^d {\partial_{\vx_i}\vz^{(l-1)}_n}^\top \otimes \jac_{\partial_{\vx_i}\vz^{(l)}}\Delta_{\vx}\vu_n
    %\\ & \quad
    %+
    %{\Delta_{\vx}\vz^{(l-1)}_n}^\top \otimes \jac_{\Delta_{\vx}\vz^{(l)}} \Delta_{\vx}\vu_n.
    %\operatorname{diag}\left({\vz^{(l-1)}}^\top \otimes \jac_{\vz^{(l-1)}} \Delta_{\vx} \vu_n, \dots, {\Delta_\vx\vz^{(l-1)}}^\top \otimes \jac_{\Delta_\vx\vz^{(l-1)}} \Delta_{\vx} \vu_n\right)? not quite
\end{align*}
where $\mZ_{n, 1}^{(l)} = \vz_n^{(l)}, \mZ_{n, 2}^{(l)} = \partial_{x_1}\vz_n^{(l)}, \dots, \mZ_{n, 1+d}^{(l)} = \partial_{x_d}\vz_n^{(l)}$ and $\mZ_{n, 2+d}^{(l)} = \Delta_x\vz_n^{(l)}$.
Using the notation
$\vg_{n, s}^{(l)} = \nabla_{\mZ_{s}^{(l)}} \Delta_\vx\vu_n$ we obtain
\begin{equation}\label{eq:laplace_gramian_block_exact}
    \mG_\Omega(\mW^{(l)})
    =
    \frac1N\sum_{n=1}^N
    \left[\sum_{s=1}^S \left( \mZ^{(l-1)}_{n,s}\otimes \vg_{n,s}^{(l)} \right)
    \cdot
    \sum_{s=1}^S \left( \mZ
    ^{(l-1)}_{n,s}\otimes \vg_{n,s}^{(l)} \right)^\top\right]
\end{equation}
Now we can use any Kronecker-factored approximation for feedforward networks with weight sharing, where we choose to work with the following. 

%
%\paragraph{Kronecker approximation of the PDE Gramian}


\paragraph{The expand approximation}
Approximating the inner matrix product of sums in equation~\eqref{eq:laplace_gramian_block_exact} by the sum of matrix products, simplifying, and approximating the sum of Kroneckers by the Kronecker of the sum we obtain
\begin{align}
    \hat{\mG}_\Omega^{\textup{exp}}(\mW^{(l)})
    \coloneqq \frac{1}{N^2 S}
    \left[\sum_{n=1}^N \sum_{s=1}^S \mZ^{(l-1)}_{n,s}{\mZ^{(l-1)}_{n,s}}^\top \right]
    \otimes
    \left[\sum_{n=1}^N\sum_{s=1}^S \vg^{(l)}_{n,s}{\vg^{(l)}_{n,s}}^\top   \right].
\end{align}
This approximation corresponds to the \emph{expand} setting described by~\citet{eschenhagen2023kroneckerfactored}. 
%\todo{@Felix. Here it would be natural to comment on forward and empirical, right? To my understanding we do not really use the reduce approximation, correct?} 

\paragraph{Forward KFAC}
One downside of our approach is that we need to store the full computation graph of the forward Laplacian framework to compute the grad-output-based Kronecker factors. 
To speed up computation, we can neglect the quantities in the approximation $\hat{\mG}_\Omega^{\textup{exp}}$ that require a backward pass. 
We call the corresponding approximation \emph{forward KFAC}, which is given by 
\begin{align}
    \hat{\mG}_\Omega^{\textup{for}}(\mW^{(l)})
    \coloneqq \frac{1}{N^2 S}
    \left[\sum_{n=1}^N \sum_{s=1}^S \mZ^{(l-1)}_{n,s}{\mZ^{(l-1)}_{n,s}}^\top \right]
    \otimes
    \mI.
\end{align}
Some works find that if setting this factor to $\mI$, the resulting input-based KFAC still yields good performance~\cite{benzing2022gradient,petersen2023isaac}. 
%\paragraph{Input-based Kronecker approximation:} 

\paragraph{Empirical KFAC}
... how does this look? ... 
probably move to appendix or simply say it can be done in the future... 


%\paragraph{The reduce approximation}
%Alternatively, approximating the sum of Kronecker products (running over $s$) by the Kronecker product of the sum, simplifying and again approximating for the outer sum (running over $n$) we obtain
%\begin{align}
%    \hat{\mG}_\Omega^{\textup{red}}(\mW^{(l)})
%    \coloneqq
%    \frac{1}{N^2 S^2} \left[\sum_{n=1}^N \left( \sum_{s=1}^S \mZ^{(l-1)}_{n,s} \right) \left(\sum_{s=1}^S {\mZ^{(l-1)}_{n,s}}\right)^\top\right]
%    \otimes
%    \left[\sum_{n=1}^N\left(\sum_{s=1}^S \vg^{(l)}_{n,s} \right) \left( \sum_{s=1}^S {\vg^{(l)}_{n,s}} \right)^\top \right]
%\end{align}
%This approximation coincide with the \emph{reduce} setting described in \cite{eschenhagen2023kroneckerfactored}.




%\paragraph{Empirical and non empirical Gramians}

\paragraph{Combining PDE and boundary Gramian}
One way is to keep both Kronecker approximations separate, but then we need to invert a sum of two Kronecker products. 
This can be done, see \Cref{sec:inverse_kronecker_sum}, but adds twice the memory overhead compared to having a single Kronecker approximation.



% expand approximation treats the shared axis like a batch axis
\begin{comment}
\paragraph{Kronecker approximation under weight sharing}
Now consider a layer whose weight is applied onto \emph{multiple} vectors.
This concept is known as weight sharing.
This could be a linear layer with matrix-valued inputs like in attention, a convolution layer whose kernel is shared between patches of the input, or weights that are used multiple times throughout the computation graph (e.g.\, weight tying).
This means the layer will not process a single vector $\vz^{(l)}$, but a sequence of vectors $\{ \vz^{(l)}_1, \dots, \vz^{(l)}_S \}$ where $S$ denotes weight sharing number.
We can column-stack these vectors
and obtain
\begin{align}
  \begin{pmatrix}
    \vz^{(l)}_1
    &
    \cdots
    &
    \vz^{(l)}_S
  \end{pmatrix}
  &=
    %\begin{pmatrix}
      \mW^{(l)} %& \vb
    %\end{pmatrix}
    \begin{pmatrix}
      \vz^{(l-1)}_1
      &
      \cdots
      &
        \vz^{(l-1)}_S
      %\\
      %1 & 0 & 0
    \end{pmatrix}.
    \label{eq:forward-laplacian-linear-layer-compact}
\end{align}
The parameter Jacobian is consequently given by the
\begin{align}
    \jac_{\mW^{(l)}} \mZ^{(l)}
  &=
    %\begin{pmatrix}
      %\mW^{(l)} %& \vb
    %\end{pmatrix}
    \begin{pmatrix}
      \vz^{(l-1)}_1 \otimes \mI
      &
      \cdots
      &
        \vz^{(l-1)}_S\otimes \mI
      %\\
      %1 & 0 & 0
    \end{pmatrix}.
    \label{eq:forward-laplacian-linear-layer-compact}
\end{align}
Using the notation $\vg_s\coloneqq \nabla_{\vz^{(l)}_s} \vz^{L}_s$ the Gauß-Newton matrix is given by
\begin{align}
    \mF(\mW^{(l)}) = {\jac_{\mW^{(l)}} \mZ^{(L)}}^\top \jac_{\mW^{(l)}} \mZ^{(L)}
\end{align}

into a matrix $\mZ^{(l)} \in \sR^{D_{\text{in}}\times S}$, likewise for the linear layer's outputs $\mZ^{(l+1)} = \mW^{(l)} \mZ^{(l)} \in \sR^{D_{\text{out}}\times S}$ and activation gradients $\mG^{(l)} \in \sR^{D_{\text{out}} \times S}$.\todo{introduce the gradient notation}
The output-weight Jacobian of a weight-sharing layer is $\jac_{\mW^{(l)}} \mZ^{(l+1)} = {\mZ^{(l)}}^{\top} \otimes \mI$ \cite[see e.g.][]{dangel2020modular} and the Fisher does not simplify into a Kronecker product without further approximations.
%As described by \citet{eschenhagen2023kroneckerfactored}, there are two possible Kronecker approximations for this setup.
We will focus on the \emph{expand} approximation, which yields the Kronecker approximation for convolutional layers proposed by~\citet{grosse2016kroneckerfactored}.
It treats the shared axis like a batch axis,
\begin{align}
    \mF(\mW^{(l)}) = \frac1S \sum_{s=1}^S (\vz_s^{(l)} \otimes {\vg_s^{(l)}}^\top)({\vz_s^{(l)}}^{\top}\otimes \vg_s^{(l)}) \approx \frac{1}{S} \sum_{s=1}^S \vz_s \vz_s^{\top} \otimes \sum_{s=1}^S \vg_s \vg_s^{\top}. %, %\quad \text{where } \vg_s = ...%\grad{\vz_s} \ell(\vx, \hat{\vy}, \mW).
\end{align}
We can express this in matrix notation as $\mF(\mW^{(l)}) \approx \nicefrac{1}{S} \mZ^{(l)} {\mZ^{(l)}}^{\top} \otimes \mG^{(l)} {\mG^{(l)}}^{\top}$.
\end{comment}
%\subsection{Discussion and generalizations}

\subsection{General preconditioned PDE methods}
%\paragraph{Setting}
We consider the general %second-order 
PDE
\begin{equation}
    %\Psi(u, \nabla u, \nabla^{2} u) = 0,
    \Psi(u, \partial u, \dots, \partial^{(k)} u) = 0,
\end{equation}
where we assume $\Psi\colon \mathbb R^{K}\to\mathbb R^M$ to be smooth.
%We write $\mathcal D u \coloneqq (u, \partial u, \dots, \partial^{(k)} u)$
%Set $v_\vtheta(x)\coloneqq (u_\vtheta(\vx), \nabla u_\vtheta(\vx), \nabla^{2} u_\vtheta(\vx))$
%and
We denote the residual by $r_\vtheta(\vx)\coloneqq \Psi(u_\vtheta(\vx), \nabla u_\vtheta(\vx), \nabla^{2} u_\vtheta(\vx))%v_\vtheta(x))
$ %we denote the residual and for suitable integration points $\vx_n$ we
and consider the PINN loss
\begin{equation}
    %\frac1N\sum_{n=1}^N \ell(\Psi(\mathcal Du_\vtheta(\vx_n)),
    L(\vtheta)\coloneqq \frac1N\sum_{n=1}^N \ell(r_{\vtheta}(\vx_n)),
\end{equation}
where we assume $\ell\colon\mathbb R^M\to\mathbb R$ is a convex function with definite Hessian $\nabla^2\ell\succ0$ and unique minimizer at $0$.
We consider the Gauß-Newton matrix
\begin{align}
    \mG(\vtheta) & \coloneqq \frac1N\sum_{n=1}^N \jac_\vtheta r_\vtheta(\vx_n)^\top \mLambda(r_\vtheta(\vx_n)) \jac_\vtheta r_\vtheta(\vx_n),
    %\\ & = \frac1N\sum_{n=1}^N \jac_\vtheta v_\vtheta(\vx_n)^\top \underbrace{\jac \Psi(v_\vtheta(\vx_n))^\top \mLambda(r_\vtheta(\vx_n)) \jac \Psi(v_\vtheta(\vx_n))}_{\eqqcolon \mA(\vtheta, \vx)}  \jac_\vtheta v_\vtheta(\vx_n),
\end{align}
where $\mLambda(r)\coloneqq \nabla^2 \ell(r)$ denotes the Hessian of the per-sample loss. %~\citep{eschenhagen2023kroneckerfrwtactored} yxxx
%and $v_\vtheta$ is a forward network with shared weights.

Typically, in PINNs, one chooses $\ell = \frac12\lVert \cdot \rVert_2^2$ to be the squared Euclidean distance, such that $\mLambda = \mI$.
Note, however, that in contrast to a regression problem, the residual $r_\vtheta$ involves PDE terms and not only function evaluations.
If we have a forward iteration for $r_\vtheta$, we can apply existing KFAC approximations.
We obtain this via higher-order forward mode automatic differentiation.

Writing $v_\vtheta(\vx)\coloneqq (%u_\vtheta(\vx), \nabla u_\vtheta(\vx), \nabla^{2} u_\vtheta(\vx)
u(\vx), \partial u(\vx), \dots, \partial^{(k)} u(\vx))$
we can express the Gauß-Newton matrix as
\begin{align}
    \mG(\vtheta) & = \frac1N\sum_{n=1}^N \jac_\vtheta v_\vtheta(\vx_n)^\top \underbrace{\jac \Psi(v_\vtheta(\vx_n))^\top \mLambda(r_\vtheta(\vx_n)) \jac \Psi(v_\vtheta(\vx_n))}_{\eqqcolon \mH(\vtheta, \vx_n)}  \jac_\vtheta v_\vtheta(\vx_n), 
\end{align}
and $v_\vtheta$ is a forward network with shared weights.
The reduce approximation is then given by
\begin{align}
    \hat{\mG}(\mW^{(l)}) \coloneqq \left(\sum_{n=1}^N \left( \sum_{s=1}^S \vz^{(l)}_{n,s} \right) \left(\sum_{s=1}^S {\vz^{(l)}_{n,s}}^\top\right)\right)\otimes\left(\sum_{n=1}^N\left(\sum_{s=1}^S \vg^{(l)}_{n,s} \right) \mH(\vtheta, \vx_n) \left( \sum_{s=1}^S {\vg^{(l)}_{n,s}}^\top \right) \right)?
\end{align}

\todo[inline]{striking remark how that this can be applied for any Gramian involving PDE terms}

%There are some challenges we need to overcome to define a Kronecker-factored approximation of the Gramian from \Cref{eq:gramian}:
%\begin{itemize}
%\item The Gramian of the interior loss involves the parameter gradient of the Laplacian. To establish a Kronecker approximation, we need to know how the weight matrix of a layer enters the computation of the Laplacian. We will show that, when using the forward Laplacian framework from \cite{li2023forward}, the weight matrix enters the computation by multiplication onto another matrix. This is great, because we know how to define KFAC in such cases, thanks to the KFAC-for-weight-sharing framework developed by \citet{eschenhagen2023kroneckerfactored}.

%\item %The PINN loss consists of two terms, the interior and boundary loss.
  %We can develop a Kronecker approximation for both individually, which leaves us with the problem how to work with both terms to pre-condition a gradient.
  %One way is to keep the both Kronecker approximations separate, but then we need to invert a sum of two Kronecker products.
  %This can be done, see \Cref{sec:inverse_kronecker_sum}, but adds twice the memory overhead compared to having a single Kronecker approximation.
  %Also, the Kronecker sum's inversion requires solving a generalized eigenvalue problem, for which there is currently no API in PyTorch.
  %Hence we need to fall back to SciPy, which costs communication overhead because everything needs to be off-loaded to CPU.
  %Alternatively, we could summarize the two Kronecker approximations into a single one at the risk of losing downstream performance.
%\end{itemize}


\begin{comment}
    \subsection{Computational complexity}
\todo[inline]{what can we say? probably better to move to the respective approximations}
\begin{itemize}
    \item GN: $O(p^3+Np^2)$
    \item BFGS: $O(p^2+?)$ 
    \item L-BFGS: $O(mp+?)$?
    \item KFAC: $O(?)$ 
    \item  
\end{itemize}
\end{comment}

%%% Local Variables:
%%% mode: latex
%%% TeX-master: "../main"
%%% End:

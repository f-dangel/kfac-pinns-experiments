
\subsection{A Kronecker-factored approximation for the Poisson equation}

Just like in the case of the classic KFAC algorithm, we only consider the diagonal blocks of the Gauß-Newton matrix, which for the Poisson equation, are given by
\begin{align}
    \mG_\Omega(\mW^{(l)}) = \frac1{{N_\Omega}} \sum_{n=1}^{N_\Omega} \jac_{\mW^{(l)}} \Delta_\vx \vu_n^\top \jac_{\mW^{(l)}} \Delta_\vx \vu_n
\end{align}
Just like in the case of the block Fisher matrices $\mF(\vtheta^{(l)})$ we use the chain rule and compute
\begin{align*}
    \jac_{\mW^{(l)}} \Delta_\vx \vu_n & = \jac_{\mZ^{(l)}}\Delta_\vx \vu_n \jac_{\mW^{(l)}} \mZ^{(l)} %=
    %\jac_{\mW^{(l)}} \mZ^{(l)} \jac_{\mZ^{(l)}}\mZ^{(L)}\jac_{\mZ^{(L)}}\Delta_\vx \vu_n.
    %\\ & =
    %\operatorname{diag}\left({\vz^{(l-1)}}^\top \otimes \mI, \dots, {\Delta_\vx\vz^{(l-1)}}^\top \otimes \mI\right) \jac_{\mZ^{(l)}}\Delta_\vx \vu_n
    %\begin{pmatrix}
    %    \nabla_{\vz^{(l)}} \Delta_{\vx}\vu_n \\
    %    \nabla_{\partial_{\vx_1}\vz^{(l)}}\Delta_{\vx}\vu_n \\
    %    \vdots \\
    %    \nabla_{\partial_{\vx_d}\vz^{(l)}} \Delta_{\vx}\vu_n  \\
    %    \nabla_{\Delta_{\vx}\vz^{(l)}} \Delta_{\vx}\vu_n \\
    %\end{pmatrix}^\top
    %\begin{pmatrix}
    %    {\vz^{(l-1)}_n}^\top \otimes \mI \\
    %    {\partial_{\vx_1}\vz^{(l-1)}_n}^\top \otimes \mI \\
    %    \vdots \\
    %    {\partial_{\vx_d}\vz^{(l-1)}_n}^\top \otimes \mI \\
    %    {\Delta_{\vx}\vz^{(l-1)}_n}^\top \otimes \mI \\
    %\end{pmatrix}
    %\\ & 
    %= 
    %\sum_{s=1}^S (\nabla_{\mZ^{(l)}_s} \Delta_{\vx}\vu_n)^\top  ({\mZ^{(l-1)}_{n,s}}^\top \otimes \mI)?
    %\\ & 
    = \left(\sum_{s=1}^S \mZ^{(l-1)}_{n,s} \otimes \nabla_{\mZ^{(l)}_s} \Delta_{\vx}\vu_n\right)^\top,
    %{\vz^{(l-1)}_n}^\top \otimes \jac_{\vz^{(l)}} \Delta_{\vx}\vu_n
    %+
    %\sum_{i=1}^d {\partial_{\vx_i}\vz^{(l-1)}_n}^\top \otimes \jac_{\partial_{\vx_i}\vz^{(l)}}\Delta_{\vx}\vu_n
    %\\ & \quad
    %+
    %{\Delta_{\vx}\vz^{(l-1)}_n}^\top \otimes \jac_{\Delta_{\vx}\vz^{(l)}} \Delta_{\vx}\vu_n.
    %\operatorname{diag}\left({\vz^{(l-1)}}^\top \otimes \jac_{\vz^{(l-1)}} \Delta_{\vx} \vu_n, \dots, {\Delta_\vx\vz^{(l-1)}}^\top \otimes \jac_{\Delta_\vx\vz^{(l-1)}} \Delta_{\vx} \vu_n\right)? not quite
\end{align*}
where $\mZ_{n, 1}^{(l)} = \vz_n^{(l)}, \mZ_{n, 2}^{(l)} = \partial_{x_1}\vz_n^{(l)}, \dots, \mZ_{n, 1+d}^{(l)} = \partial_{x_d}\vz_n^{(l)}$ and $\mZ_{n, 2+d}^{(l)} = \Delta_x\vz_n^{(l)}$ and $S=d+2$.
Using the notation
$\vg_{n, s}^{(l)} = \nabla_{\mZ_{s}^{(l)}} \Delta_\vx\vu_n$ we obtain
\begin{equation}\label{eq:laplace_gramian_block_exact}
    \mG_\Omega(\mW^{(l)})
    =
    \frac1N\sum_{n=1}^N
    \left[\sum_{s=1}^S \left( \mZ^{(l-1)}_{n,s}\otimes \vg_{n,s}^{(l)} \right)
    \cdot
    \sum_{s=1}^S \left( \mZ
    ^{(l-1)}_{n,s}\otimes \vg_{n,s}^{(l)} \right)^\top\right]
\end{equation}
Now we can use any Kronecker-factored approximation for feedforward networks with weight sharing, where we choose to work with the following.

%
%\paragraph{Kronecker approximation of the PDE Gramian}

%\paragraph{Individual steps} 
Using the crude approximation $\sum_{m=1}^M \mA_m\sum_{m=1}^M \mB_m \approx M^{-1}\sum_{m=1}^M \mA_m \mB_m$ we compute 
\begin{align*}
    & \sum_{n=1}^N
    \left[\sum_{s=1}^S \left( \mZ^{(l-1)}_{n,s}\otimes \vg_{n,s}^{(l)} \right)
    \cdot
    \sum_{s=1}^S \left( \mZ
    ^{(l-1)}_{n,s}\otimes \vg_{n,s}^{(l)} \right)^\top\right]
    \\ \approx & \;
    \frac{1}{S}\sum_{n=1}^N
    \left[\sum_{s=1}^S \left( \mZ^{(l-1)}_{n,s}\otimes \vg_{n,s}^{(l)} \right)
    \cdot
    \left( \mZ
    ^{(l-1)}_{n,s}\otimes \vg_{n,s}^{(l)} \right)^\top\right]
    \\ = & \;
    \frac{1}{S}\sum_{n=1}^N
    \left[\sum_{s=1}^S \left( \mZ^{(l-1)}_{n,s}{\mZ
    ^{(l-1)}_{n,s}}^\top\otimes \vg_{n,s}^{(l)}{\vg_{n,s}^{(l)}}^\top \right)
    \right]
    \\ \approx & \;
    \frac{1}{NS}\left[\sum_{n=1}^N \sum_{s=1}^S \mZ^{(l-1)}_{n,s}{\mZ^{(l-1)}_{n,s}}^\top \right]
    \otimes
    \left[\sum_{n=1}^N\sum_{s=1}^S \vg^{(l)}_{n,s}{\vg^{(l)}_{n,s}}^\top \right],
\end{align*}
%\paragraph{The expand approximation}
%Approximating the inner matrix product of sums in equation~\eqref{eq:laplace_gramian_block_exact} by the sum of matrix products\todo{can we justify this?}, simplifying, and approximating the sum of Kroneckers by the Kronecker of the sum we obtain 
which gives the approximation 
\begin{tcolorbox}[colframe=kfac, title={KFAC for Gauß-Newton with the Laplace operator},bottom=0mm,top=0mm,middle=0mm]
\begin{align*}\label{eq:KFAC-PINN}
    \hat{\mG}_\Omega%^{\textup{exp}}
    (\mW^{(l)})
    \coloneqq \frac{1}{N^2 S}
    \left[\sum_{n=1}^N \sum_{s=1}^S \mZ^{(l-1)}_{n,s}{\mZ^{(l-1)}_{n,s}}^\top \right]
    \otimes
    \left[\sum_{n=1}^N\sum_{s=1}^S \vg^{(l)}_{n,s}{\vg^{(l)}_{n,s}}^\top \right].
\end{align*}
\end{tcolorbox}

\subsection{KFAC for Gauß-Newton matrices involving general PDE terms}

Here, we provide the derivation of the general Kronecker-factored approximation of the Gauß-Newton matrix. 
Recall that we consider the general PDE
\begin{equation}
    \Psi(u, \partial u, \dots, \partial^{(k)} u) = 0,
\end{equation}
where $\Psi\colon \mathbb R^{K}\to\mathbb R^M$ is some smooth function.
By $r_\vtheta(\vx)\coloneqq \Psi(u(\vx), \partial u(\vx), \dots, \partial^{(k)} u(\vx))$ we denote the residual and consider the PINN loss
\begin{equation}
    L(\vtheta)\coloneqq \frac1N\sum_{n=1}^N \ell(r_{\vtheta}(\vx_n)),
\end{equation}
where $\ell\colon\mathbb R^K\to\mathbb R, K=\binom{d+k}{d}$ is a smooth convex function with definite Hessian $\nabla^2\ell\succ0$ that typically has a unique minimizer at $0$. 
We consider the Gauß-Newton matrix
\begin{align}
    \mG(\vtheta) & \coloneqq \frac1N\sum_{n=1}^N \jac_\vtheta r_\vtheta(\vx_n)^\top \mLambda(r_\vtheta(\vx_n)) \jac_\vtheta r_\vtheta(\vx_n),
\end{align}
where $\mLambda(r)\coloneqq \nabla^2 \ell(r)$ denotes the Hessian of the per-sample loss. 

Typically, in PINNs, one chooses $\ell = \frac12\lVert \cdot \rVert_2^2$ to be the squared Euclidean distance, such that $\mLambda = \mI$.
Note, however, that in contrast to a regression problem, the residual $r_\vtheta$ involves PDE terms and not only function evaluations.
If we have a forward iteration for $r_\vtheta$, we can apply existing KFAC approximations.
We obtain this via higher-order forward mode automatic differentiation.

Writing $v_\vtheta(\vx)\coloneqq (%u_\vtheta(\vx), \nabla u_\vtheta(\vx), \nabla^{2} u_\vtheta(\vx)
u(\vx), \partial u(\vx), \dots, \partial^{(k)} u(\vx))$
we can express the block corresponding to the $l$-th layer of the Gauß-Newton matrix as 
\begin{align}
    \mG(\mW^{(l)}) & = \frac1N\sum_{n=1}^N \jac_{\mW^{(l)}} v_\vtheta(\vx_n)^\top \underbrace{\jac \Psi(v_\vtheta(\vx_n))^\top \mLambda(r_\vtheta(\vx_n)) \jac \Psi(v_\vtheta(\vx_n))}_{\eqqcolon \mA_n}  \jac_{\mW^{(l)}} v_\vtheta(\vx_n),
\end{align}
and $v_\vtheta$ is a forward network with shared weights.

Note that the forward pass in~\eqref{eq:forward_pass} extends to higher-order partial derivatives 
\begin{align}
    \partial_\vx^\alpha \vz^{(l)} = \mW^{(l)}\partial_\vx^\alpha \vz^{(l-1)} 
\end{align}
and hence, we can perceive the computational graph of $v_\vtheta$ as a feedforward network with weight sharing over the multi-indices $\alpha\in\mathbb N$ with $\lvert \alpha \rvert \le k$.

Just like in the case of the block Fisher matrices $\mF(\vtheta^{(l)})$ we use the chain rule and compute
\begin{align*}
    \jac_{\mW^{(l)}} \vv_n & = \sum_{\lvert \alpha \rvert\le k}\jac_{\partial_\vx^\alpha \vz^{(l-1)}}\vv_n \jac_{\mW^{(l)}} \partial_\vx^\alpha \vz^{(l)} = 
    \left(\sum_{\lvert \alpha \rvert\le k} \partial_\vx^\alpha \vz^{(l-1)}_n \otimes \nabla_{\partial_\vx^\alpha \vz^{(l-1)}} \vv_n\right)^\top,
\end{align*}
where $\vv_n = v_\vtheta(\vx_n)$. 
Using the notation $\vg^{(l)}_{n,\alpha} = \nabla_{\partial_\vx^\alpha \vz_n^{(l)}} \vv_n$ we obtain
\begin{equation}\label{eq:laplace_gramian_block_exact}
    \mG_\Omega(\mW^{(l)})
    =
    \frac1N\sum_{n=1}^N
    \left[\sum_{\lvert \alpha \rvert\le k} \left( \partial_\vx^\alpha \vz^{(l-1)}_n \otimes \vg_{n,s}^{(l)} \right)
    \mA_n 
    \sum_{\lvert \alpha \rvert\le k} \left( \partial_\vx^\alpha \vz^{(l-1)}_n \otimes \vg_{n,\alpha}^{(l)} \right)^\top\right]. 
\end{equation}
\todo[inline]{not sure, whether this works without any structure in $\mA_n$ or without another approximation}
Using the crude approximation $\sum_{m=1}^M \mA_m\sum_{m=1}^M \mB_m \approx M^{-1}\sum_{m=1}^M \mA_m \mB_m$ we compute 
\begin{align*}
    & \sum_{n=1}^N
    \left[\sum_{\lvert \alpha \rvert\le k} \left( \partial_\vx^\alpha \vz^{(l-1)}_n\otimes \vg_{n,\alpha}^{(l)} \right)
    \mA_n
    \sum_{\lvert \alpha \rvert\le k} \left( \partial_\vx^\alpha \vz^{(l-1)}_n\otimes \vg_{n,\alpha}^{(l)} \right)^\top\right]
    \\ \approx & \;
    \frac{1}{\binom{d+k}{d}} \sum_{n=1}^N
    \left[\sum_{\lvert \alpha \rvert\le k} \left( \partial_\vx^\alpha \vz^{(l-1)}_n \otimes \vg_{n,\alpha}^{(l)} \right)
    \mA_n
    \left( \partial_\vx^\alpha \vz^{(l-1)}_n \otimes \vg_{n,\alpha}^{(l)} \right)^\top\right]
    \\ = & \;
    \frac{1}{\binom{d+k}{d}}\sum_{n=1}^N
    \left[\sum_{\lvert \alpha \rvert\le k} \left( \partial_\vx^\alpha \vz^{(l-1)}_n{\partial_\vx^\alpha \vz^{(l-1)}_n}^\top\otimes \vg_{n,\alpha}^{(l)}{\vg_{n,\alpha}^{(l)}}^\top \right)
    \right]
    \\ \approx & \;
    \frac{1}{N\binom{d+k}{d}}\left[\sum_{n=1}^N \sum_{\lvert \alpha \rvert\le k} \partial_\vx^\alpha \vz^{(l-1)}_n{\partial_\vx^\alpha \vz^{(l-1)}_n}^\top \right]
    \otimes
    \left[\sum_{n=1}^N\sum_{\lvert \alpha \rvert\le k} \vg^{(l)}_{n,\alpha}{\vg^{(l)}_{n,\alpha}}^\top \right],
\end{align*}
which gives the approximation 
\begin{tcolorbox}[colframe=kfac, title={KFAC for Gauß-Newton with general PDE terms},bottom=0mm,top=0mm,middle=0mm]
\begin{align*}\label{eq:KFAC-PINNs-general}
    \hat{\mG}%^{\textup{exp}}
    (\mW^{(l)})
    \coloneqq \frac{1}{N^2 \binom{d+k}{d}}
    \left[\sum_{n=1}^N \sum_{\lvert \alpha \rvert \le k} %\mZ^{(l-1)}_{n,\alpha}{\mZ^{(l-1)}_{n,\alpha}}^\top 
    \partial_\vx^\alpha \vz_n^{(l-1)} {\partial_\vx^\alpha \vz_n^{(l-1)}}^\top\right] 
    \otimes
    \left[\sum_{n=1}^N\sum_{\lvert \alpha \rvert \le k} \vg^{(l)}_{n,\alpha} \mA_n {\vg^{(l)}_{n,\alpha}}^\top \right]. 
\end{align*}
\end{tcolorbox}


This approximation can be stated for a large class of weight-sharing architectures in which case it generalizes the \emph{expand approximation} by~\citet{eschenhagen2023kroneckerfactored}. 




\begin{equation*}
    \mG_\Omega(\mW^{(l)})
    =
    \frac1N\sum_{n=1}^N
    \left[\sum_{s=1}^S \left( \mZ^{(l-1)}_{n,s}\otimes \vg_{n,s}^{(l)} \right)
    \cdot
    \sum_{s=1}^S \left( \mZ
    ^{(l-1)}_{n,s}\otimes \vg_{n,s}^{(l)} \right)^\top\right]
\end{equation*}

%
%\paragraph{Kronecker approximation of the PDE Gramian}

\begin{comment}
    \paragraph{The reduce approximation}
Approximating the sum of Kronecker products (running over $s$) by the Kronecker product of the sum, simplifying and again approximating for the outer sum (running over $n$) we obtain
\begin{align*}
    \hat{\mG}_\Omega^{\textup{red}}(\mW^{(l)})
    \coloneqq
    \frac{1}{N^2 S^2} \left[\sum_{n=1}^N \left( \sum_{s=1}^S \mZ^{(l-1)}_{n,s} \right) \left(\sum_{s=1}^S {\mZ^{(l-1)}_{n,s}}\right)^\top\right]
    \otimes
    \left[\sum_{n=1}^N\left(\sum_{s=1}^S \vg^{(l)}_{n,s} \right) \left( \sum_{s=1}^S {\vg^{(l)}_{n,s}} \right)^\top \right]
\end{align*}
This approximation agrees with the \emph{reduce} setting described by~\citet{eschenhagen2023kroneckerfactored}.

\paragraph{Individual steps} Can be moved to the appendix or deleted:

\begin{align*}
    &\; \sum_{n=1}^N
    \left[\sum_{s=1}^S \left( \mZ^{(l-1)}_{n,s}\otimes \vg_{n,s}^{(l)} \right)
    \cdot
    \sum_{s=1}^S \left( \mZ
    ^{(l-1)}_{n,s}\otimes \vg_{n,s}^{(l)} \right)^\top\right]
    \\ \approx & \;
    \sum_{n=1}^N
    \frac{1}{S^2}\left[\left(\left( \sum_{s=1}^S\mZ^{(l-1)}_{n,s}\right)\otimes \left(\sum_{s=1}^S\vg_{n,s}^{(l)}\right)\right)
    \cdot
    \left(\left( \sum_{s=1}^S\mZ^{(l-1)}_{n,s}\right)^\top\otimes \left(\sum_{s=1}^S\vg_{n,s}^{(l)}\right)^\top\right)\right]
    \\ = & \;
    \frac{1}{S^2}\sum_{n=1}^N
    \left[\left(\left( \sum_{s=1}^S\mZ^{(l-1)}_{n,s}\right)\left( \sum_{s=1}^S\mZ^{(l-1)}_{n,s}\right)^\top \right)\otimes \left(\left(\sum_{s=1}^S\vg_{n,s}^{(l)}\right)\left(\sum_{s=1}^S\vg_{n,s}^{(l)}\right)^\top\right)
     \right]
    \\ \approx & \;
    \frac{1}{N S^2}\left[\sum_{n=1}^N \left( \sum_{s=1}^S \mZ^{(l-1)}_{n,s} \right) \left(\sum_{s=1}^S {\mZ^{(l-1)}_{n,s}}\right)^\top\right]
    \otimes
    \left[\sum_{n=1}^N\left(\sum_{s=1}^S \vg^{(l)}_{n,s} \right) \left( \sum_{s=1}^S {\vg^{(l)}_{n,s}} \right)^\top \right]
\end{align*}
\end{comment}

\begin{comment}
    %\paragraph{The expand approximation}
Alternatively, approximating the inner matrix product of sums in equation~\eqref{eq:laplace_gramian_block_exact} by the sum of matrix products, simplifying, and approximating the sum of Kroneckers by the Kronecker of the sum we obtain
\begin{align*}
    \hat{\mG}_\Omega^{\textup{exp}}(\mW^{(l)})
    \coloneqq \frac{1}{N^2 S}
    \left[\sum_{n=1}^N \sum_{s=1}^S \mZ^{(l-1)}_{n,s}{\mZ^{(l-1)}_{n,s}}^\top \right]
    \otimes
    \left[\sum_{n=1}^N\sum_{s=1}^S \vg^{(l)}_{n,s}{\vg^{(l)}_{n,s}}^\top   \right].
\end{align*}
which coincides with the \emph{expand} setting of \cite{eschenhagen2023kroneckerfactored}. 

\paragraph{Forward KFAC}
One downside of our approach is that we need to store the full computation graph of the forward Laplacian framework to compute the grad-output-based Kronecker factors.
To speed up computation, we can neglect the quantities in the approximation $\hat{\mG}_\Omega^{\textup{exp}}$ that require a backward pass.
We call the corresponding approximation \emph{forward KFAC}, which is given by
\begin{align}
    \hat{\mG}_\Omega^{\textup{for}}(\mW^{(l)})
    \coloneqq \frac{1}{N^2 S}
    \left[\sum_{n=1}^N \sum_{s=1}^S \mZ^{(l-1)}_{n,s}{\mZ^{(l-1)}_{n,s}}^\top \right]
    \otimes
    \mI.
\end{align}
Some works find that if setting this factor to $\mI$, the resulting input-based KFAC still yields good performance~\cite{benzing2022gradient,petersen2023isaac}.
%\paragraph{Input-based Kronecker approximation:}

\paragraph{Empirical KFAC}
\begin{itemize}
    \item write the Hessian as the expectation with log-likelihood and Gaussian noise 
    \item then approximate the expectation with empirical samples 
    \item this reduces computation time...? 
\end{itemize}
\end{comment}

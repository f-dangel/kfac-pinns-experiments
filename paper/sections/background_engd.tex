%\subsection{The Gauß-Newton Method for PINNs} 
%\begin{itemize}
%    \item recently, energy NGs have been proposed
%    \item one can show that they mimic Newtons method in function space
%    \item yield very good accuracy
%    \item
%\end{itemize}

\begin{comment}
    Natural gradients have been introduced by~\citet{amari1998natural} and have shown great success in reinforcement learning, and other problems...
The general idea is to replace the vanilla GD update rule by a preconditioned version
    \[ \vtheta_{k+1} = \vtheta_k - \eta_k \mG(\vtheta_k)^{-1} \nabla L(\vtheta_k), \]
where $\mG(\vtheta)\in\mathbb R^{p\times p}$, $\mG(\vtheta)_{ij} \coloneqq g_{u_\vtheta}(\partial_{\vtheta_i} u_\vtheta, \partial_{\vtheta_j} u_\vtheta)$ is a matrix capturing the function space geometry of the problem and its parametrization and $g$ denotes a suitable Riemannian metric.
In least squares regression with data $(\vx_1, \vy_1), \dots, (\vx_N, \vy_N)$ the %Gramian $\mG(\vtheta)$ is given by 
a popular choice as a Gramian is given by the Gauß-Newton matrix 
%\begin{equation}
%    F_I(\vtheta)_{ij} = \sum_{x} \frac{\partial_{\vtheta_i}p_\vtheta(x)\partial_{\vtheta_j}p_\vtheta(x)}{p_\vtheta(x)} = \sum_{x} \partial_{\vtheta_i} \log p_\vtheta(x) \partial_{\vtheta_j} \log p_\vtheta(x),
%\end{equation}
%in which case the Riemannian metric $g$ is given by the Fisher-Rao metric~\citet{}.
%For a supervised learning problem with training data $(\vx_1, \vy_1), \dots, (\vx_N, \vy_N)$ the (empirical) Fisher-information matrix commonly used, has entries the entries
\begin{equation}
  %\mF(\vtheta)_{ij} = \sum_{n=1}^N \partial_{\vtheta_i} u_\vtheta(\vx_n)\partial_{\vtheta_j} u_\vtheta(\vx_n) \quad \text{maybe use?}
  \mF(\vtheta) = \frac1N\sum_{n=1}^N \jac_{\vtheta} u_{\vtheta}(\vx_n)^\top \jac_{\vtheta} u_{\vtheta}(\vx_n) = \frac1N\sum_{n=1}^N \jac_{\vtheta} \vu_n^\top \jac_{\vtheta} \vu_n
  ,
 \end{equation}
which agrees with the (empirical) Fisher information matrix~\citep{martens2020new, eschenhagen2023kroneckerfactored}. Here, we have used the suggestive short-hand notation $\vu_n = u_\vtheta(\vx_n)$. 
\end{comment}

It is well-documented that first-order optimizers like gradient descent and Adam fail to produce satisfactory results when used to train PINNs~\citep{cuomo2022scientific}. Still, lately, function-space-inspired methods have shown more promising results~\citep{muller2024optimization}. 
Here, we focus on, \emph{energy natural gradients (ENGD)}, which in the case of PINNs agree with the Gauß-Newton method~\citep{muller2023achieving}. 
ENGD mimics Newton's method up to a projection onto the tangent space of the model and a discretization error that vanishes quadratically in the step size, thus providing locally optimal updates in the residual%~\citep{muller2023achieving}
. Alternatively, the Gauß-Newton method can be motivated from the standpoint of operator-preconditioning, where the Gauß-Newton matrix leads to optimal conditioning of the problem~\citep{de2023operator}. 

%To account for the PDE terms in the loss function in PINNs,
%the models $u_\vtheta$ are functions rather than probability measures %$p_\vtheta$
%and the loss involves PDE terms.
%In order to adjust the definition of
%To capture the geometric properties of this specific problem we consider the following Fisher / Gramian matrix %to this problemThe energy natural gradient is for this example to use the Fischer/Gramian of the form
Natural gradient methods perform parameter updates via a preconditioned gradient descent scheme $\vtheta_{t+1} = \vtheta_t - \alpha_t \mG(\vtheta_t)^+\nabla L(\vtheta_t)$, where $\mG(\vtheta)^+$ denotes the pseudo-inverse of a suitable \emph{Gramian matrix} $\mG(\vtheta)$. 
In the context of ENGD for the PINN-loss~\eqref{eq:pinn-loss}, the Gramian is given by 
\begin{equation}\label{eq:gramian}
  \mG(\vtheta) = %\mG_\Omega(\vtheta) + \mG_{\partial\Omega}(\vtheta) =
  \underbrace{\frac1{{N_\Omega}} \sum_{n=1}^{N_\Omega} \jac_{\vtheta} \mathcal{L} u_\vtheta(\vx_n)^\top \jac_{\vtheta} \mathcal{L} u_\vtheta(\vx_n)}_{\eqqcolon \mG_\Omega(\vtheta)} + \underbrace{\frac1{{N_{\partial\Omega}}} \sum_{n=1}^{N_{\partial\Omega}} \jac_{\vtheta} u_\vtheta(\vx_n^b)^\top \jac_{\vtheta} u_\vtheta (\vx_n^b)}_{\eqqcolon \mG_{\partial\Omega}(\vtheta)}.
  %\frac1{{N_\Omega}} \sum_{k=1}^{N_\Omega} \partial_{\vtheta_i} \Delta u_\vtheta(x_k) \partial_{\vtheta_j} \Delta u_\vtheta(x_k) + \frac1{{N_{\partial\Omega}}} \sum_{k=1}^{N_{\partial\Omega}} \partial_{\vtheta_i} u_\vtheta(x_k^b) \partial_{\vtheta_j} u_\vtheta (x_k^b)
\end{equation}
%where
%\begin{equation}\label{eq:FisherInterior}
%  F_\Omega(\vtheta)_{ij} = \frac1{{N_\Omega}} \sum_{k=1}^{N_\Omega} \partial_{\vtheta_i} \Delta u_\vtheta(x_k) \partial_{\vtheta_j} \Delta u_\vtheta(x_k)
  % = \frac1{{N_\Omega}} \sum_{i=1}^{N_\Omega} (\partial_{\vtheta_i} f_\vtheta) (\partial_{\vtheta_j} f_\vtheta ),
%\end{equation}
% where $f_\vtheta = \Delta u_\vtheta$.
%and
%\begin{equation}
%  F_{\partial\Omega}(\vtheta)_{ij} = \frac1{{N_{\partial\Omega}}} \sum_{k=1}^{N_{\partial\Omega}} \partial_{\vtheta_i} u_\vtheta(x_k^b) \partial_{\vtheta_j} u_\vtheta (x_k^b).
  % = \frac1{{N_\Omega}} \sum_{i=1}^{N_\Omega} (\partial_{\vtheta_i} f_\vtheta) (\partial_{\vtheta_j} f_\vtheta ),
%\end{equation}
%has been suggested under the name \emph{energy natural gradient} (ENG), see~\cite{muller2023achieving}.


%The energy natural gradient method mimics Newton's method up to a projection onto the tangent space of the model and a discretization error that vanishes quadratically in the step size
%and improves the accuracy of PINNs by several orders of magnitude when compared to GD, Adam, or BFGS~\citep{muller2023achieving, ?}.
%The Gramian matrix $\mG(\theta)$ can be interpreted as the Gauß-Newton matrix of the residual function, see~\cite{} and Appendix....

%\paragraph{Interpretation as Gau\ss-Newton method}
%Consider the combined residual
Note that $\mG(\vtheta)$ is the Gauß-Newton matrix of $r(\theta) = ((2N_\Omega)^{-1/2}r_\Omega(\vtheta), (2N_{\partial\Omega})^{-1/2} r_{\partial\Omega}(\vtheta))^\top$, 
%\begin{equation*}
%    r\colon\mathbb R^p\to\mathbb R^{N_\Omega+N_{\partial\Omega}}, \quad 
%    \vtheta 
%    \mapsto 
%    \begin{pmatrix}
%        (2N_\Omega)^{-1/2}r_\Omega(\vtheta) \\ 
%        (2N_{\partial\Omega})^{-1/2}r_{\partial\Omega}(\vtheta)
%    \end{pmatrix},
%\end{equation*}
where $r_{\Omega}(\vtheta)_n = \Delta u_\vtheta(\vx_n) + f(\vx_n)$ and $r_{\Omega}(\vtheta)_n = u_\vtheta(\vx_n^b) - g(\vx_n^b)$, respectively. 
%Then %(with the right choice of the inner product)
%$L(\vtheta) = \frac12 \lVert r(\vtheta) \rVert_2^2$ and hence the Gau\ss-Newton matrix is given by
%\begin{equation*}
%    \jac_\vtheta r(\vtheta)^\top \jac_\vtheta r(\vtheta) = \jac_\vtheta r_\Omega(\vtheta)^T \jac_\vtheta r_\Omega(\vtheta) + \jac_\vtheta r_{\partial\Omega}(\theta)^T \jac_\vtheta r_{\partial\Omega}(\vtheta) = \mG(\vtheta). 
%\end{equation*}

%%% Local Variables:
%%% mode: latex
%%% TeX-master: "../main"
%%% End:

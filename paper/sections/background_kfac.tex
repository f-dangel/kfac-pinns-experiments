\subsection{Kronecker-factored Approximate Curvature (KFAC)}

%\toodoo{F.D.  Get to the point more quickly. Make notation  consistent.}

We review the idea of Kronecker-factored Approximate Curvature (KFAC) which was introduced by~\citet{heskes2000natural, martens2015optimizing} as an approximation of the per-layer Fisher information by a Kronecker product to speed up the computation of the natural gradient direction. 
%We review the general principle here. 
%This approach was introduced in the context of maximum likelihood estimation with a neural network and we reivew its basic principles here. 
%Later, we will expand this to Gramian matrices with PDE terms. 
%\toodoo{We could also follow \cite{eschenhagen2023kroneckerfactored} for the notation} 


\paragraph{Block-diagonal approximation}
In the first step the Fisher-information matrix $\mF$ can be approximated by a block diagonal matrix with blocks corresponding to the Fisher-information matrices of the individual layers of the network, i.e., $\mF(\vtheta) \approx \operatorname{diag}(\mF(\vtheta^{(1)}), \dots, \mF(\vtheta^{(L)}))$. 
Note that a block diagonal linear system can be solved by solving the subsystems corresponding to the individual blocks, hence significantly reducing the computational complexity. 

\paragraph{Flattening, Jacobians, and Hessians}
%Above, we assumed all quantities ($\vz^{(l)}, \vtheta^{(l)}$) to be vectors.
%In case of tensor-valued quantities, we can first flatten them into vectors to reduce to the vector case.
%We consider the following vectorization of the matrices appearing as the trainable parameters of the network.
To vectorize the weight matrices of the linear layers we follow the % index convention to vectorize will be
\emph{first-varies-fastest} convention, which means column-stacking (row index varies first, column index varies second).
We denote the corresponding flattening operation by $\flatten(\cdot)$.
Very useful is the so-called \emph{vec-trick} stating that $\flatten(\mA\mX\mB) = (\mB^\top\otimes \mA)\flatten{\mX}$
%\begin{equation}\label{eq:vecTrick}
%  \flatten(\mA\mX\mB) = (\mB^\top\otimes \mA)\flatten{\mX}
%\end{equation}
for matrices $\mA, \mX, \mB$.
In particular, this shows that $\mB^\top\otimes \mA$ is the matrix representing the linear mapping $\mX\mapsto \mA\mX\mB$ and hence $\jac_\mX(\mA\mX\mB) = \mB^\top\otimes \mA$.
The flattening notation allows to reduce derivatives of matrix/tensor-valued objects back to the matrix case.
Consider the Jacobian $\jac_{\va}\vb$ of a vector $\vb$ w.r.t.\,a vector $\va$.
It collects all partial derivatives as $[\mJ_{\va}\vb]_{i,j} = \nicefrac{\partial [\vb]_i}{\partial [\va]_j}$.
For the Jacobian $\jac_{\mA}\mB$ of a matrix $\mB$ w.r.t.\,a matrix $\mA$, we simply have $\jac_{\mA} \mB = \jac_{\flatten( \mA )}\flatten(\mB)$.
Likewise, the Hessian $\gradsquared{\va}b$ of a scalar $b$ w.r.t.\,a vector $\va$ collects the second-order partial derivatives according to $[\gradsquared{\va}b]_{i,j} = \nicefrac{\partial^2 b}{\partial [\va]_i \partial [\va]_j}$.
For the Hessian $\gradsquared{\mA} b$ of a scalar $b$ w.r.t.\,a matrix $\mA$, we simply have $\gradsquared{\mA} b = \gradsquared{\flatten(\mA)}b$.
We also have $\grad{\mA} b = \grad{\flatten(\mA)} b$ for the gradient of a scalar w.r.t.\,a matrix.

\paragraph{Kronecker-factored approximation of the blocks}
We now consider the individual blocks $\mF(\vtheta^{(l)})$, for which we examine $\jac_{\vtheta^{(l)}} u_{\vtheta}(x_n)$ for a fixed data point. 
The parameters $\vtheta^{(l)} = \mW^{(l)}$ of the $l$-th layer appears in the computational graph by $\vz^{(l)} = \mW^{(l)}\vz^{(l-1)}$ and note that by the vec-trick, we have $\jac_{\mW^{(l)}} \vz^{(l)} = {\vz^{(l-1)}}^\top \otimes \mI$. 
By the chain rule, we have
\begin{align}
    \jac_{\mW^{(l)}} \vu_n%u_{\vtheta}(\vx_n) 
    & = \jac_{\vz^{(l)}}  \vu_n \jac_{\mW^{(l)}} \vz_n^{(l)}%\vz^{(L)}_n = %(\vz^{(l-1)}_n\otimes I) \nabla_{\vz^{(l)}}  \vz^{(L)}_n 
    = {\vz^{(l-1)}_n}^\top\otimes  \jac_{\vz^{(l)}}  \vu_n.%\vz^{(L)}_n. 
\end{align}
%and hence 
%\begin{align}
%    \nabla_{\vtheta^{(l)}} u_{\vtheta}(x_n)\nabla_{\vtheta^{(l)}} u_{\vtheta}(x_n)^\top = (\vz^{(l-1)}_n\otimes \vz^{(l-1)}_n) \nabla_{\vz^{(l)}}  \vz^{(L)}_n\nabla_{\vz^{(l)}}  \vz^{(L)}_n^\top.  
%\end{align}
Summing over the data points and using  $\sum_n \mA_n \otimes \mB_n \approx N^{-1}(\sum_n \mA_n) \otimes (\sum_n \mB_n)$~\citep{martens2015optimizing} we obtain 
\begin{equation}
    \mF(\mW^{(l)}) %= \sum_{n=1}^N \nabla_{\vtheta^{(l)}} u_{\vtheta}(x_n)\nabla_{\vtheta^{(l)}} u_{\vtheta}(x_n)^\top 
    \approx \frac1N\left(\sum_{n=1}^N \vz^{(l-1)}_n {\vz^{(l-1)}_n}^\top \right)\otimes \left(\sum_{n=1}^N\nabla_{\vz^{(l)}}  \vu_n\nabla_{\vz^{(l)}}  \vu_n^\top\right),
\end{equation}
see~\citep{eschenhagen2023kroneckerfactored}. 
Solving a Kronecker-factored linear system is only as expensive as solving two systems of the sizes of the two factors therefore greatly reducing the computational cost. 
%the inverse of a Kronecker-factored matrix is given by the Kronecker product of the individual inverses hence reducing the computational complexity of the corresponding linear system. 


\begin{comment}
    This means the layer will not process a single vector $\va$, but a sequence of vectors $\left\{ \va_1, \dots, \va_S \right\}$ where $S$ denotes weight sharing number.
We can column-stack these vectors into a matrix $\mA \in \sR^{D_{\text{in}}\times S}$, likewise for the linear layer's outputs $\vz = \mW \mA \in \sR^{D_{\text{out}}\times S}$ and activation gradients $\mG \in \sR^{D_{\text{out}} \times S}$.
The output-weight Jacobian of a weight-sharing layer is $\jac_{\mW} \mZ = \mA^{\top} \otimes \mI$ \cite[see e.g.][]{dangel2020modular} and the Fisher does not simplify into a Kronecker product without further approximations.
As described by \citet{eschenhagen2023kroneckerfactored}, there are two possible Kronecker approximations for this setup.
We will focus on the \emph{expand} approximation, which yields the Kronecker approximation for convolutional layers proposed by~\citet{grosse2016kroneckerfactored}.
It treats the shared axis like a batch axis,
$\mF(\mW) \approx \nicefrac{1}{S} \sum_{s=1}^S \va_s \va_s^{\top} \otimes \sum_{s=1}^S \vg_s \vg_s^{\top}$ where $\vg_s = \grad{\vz_s} \ell(\vx, \hat{\vy}, \mW)$.
We can express this in matrix notation as $\mF(\mW) \approx \nicefrac{1}{S} \mA \mA^{\top} \otimes \mG \mG^{\top}$.
\end{comment}


%%% Local Variables:
%%% mode: latex
%%% TeX-master: "../main"
%%% End:

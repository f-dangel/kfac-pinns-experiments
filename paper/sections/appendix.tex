\clearpage

% Use different numbering in appendix to make it clear that a
% section/table/algorithm is not in the main text
\renewcommand\thefigure{\thesection\arabic{figure}}
\renewcommand\thetable{\thesection\arabic{table}}
\renewcommand{\theequation}{\thesection\arabic{equation}}

% modified title header from main page, code extracted from NeurIPS template
\makeatletter
\vbox{%
  \hsize\textwidth
  \linewidth\hsize
  \vskip 0.1in
  \@toptitlebar
  \centering
  {\LARGE\bf \@title (Supplementary Material)\par}
  \@bottomtitlebar
  \vskip 0.3in \@minus 0.1in
}
\makeatother

% APPENDIX TOC
\startcontents[sections]
\printcontents[sections]{l}{1}{\setcounter{tocdepth}{2}}
\vspace{2em}

In more compact form and written as Jacobian products, this can be expressed as
\begin{equation}\label{eq:Jacobian_Fischer}
  F_\Omega(\theta) = Dr_\Omega(\theta)^T Dr_\Omega(\theta)
  \quad \text{and} \quad
  F_{\partial\Omega}(\theta) = Dr_{\partial\Omega}(\theta)^T Dr_{\partial\Omega}(\theta).
\end{equation}
Here, $Dr_\Omega(\theta)$ and $Dr_{\partial\Omega}(\theta)$  denote the \emph{Jacobians} of the maps
\begin{equation*}
  r_{\Omega}\colon \Theta \to \mathbb{R}^{N_\Omega}, \quad r_{\Omega}(\theta) = \frac{1}{\sqrt{N_{\Omega}}}(\Delta u_\theta(x_1), \dots, \Delta u_\theta(x_{N_{\Omega}}))
\end{equation*}
and
\begin{equation*}
  r_{\partial\Omega}\colon \Theta \to \mathbb{R}^{{N_\partial\Omega}}, \quad r_{\partial\Omega}(\theta) = \frac{1}{\sqrt{N_{\partial\Omega}}}(u_\theta(x^b_1), \dots, u_\theta(x^b_{N_{\partial\Omega}})).
\end{equation*}

\paragraph{Interpretation as Gau\ss-Newton in the residual}
Consider the combined residual map
\begin{equation*}
    r\colon\mathbb R^p\to\mathbb R^{N_\Omega+N_{\partial\Omega}}, \quad \theta \mapsto \begin{pmatrix}
        r_\Omega(\theta) \\ r_{\partial\Omega}(\theta)
    \end{pmatrix}.
\end{equation*}
Then (with the right choice of the inner product) $L(\theta) = \frac12 \lVert r(\theta) \rVert_2^2$ and hence the Gau\ss-Newton matrix is given by
\[ Dr(\theta)^\top Dr(\theta) = Dr_\Omega(\theta)^T Dr_\Omega(\theta) + Dr_{\partial\Omega}(\theta)^T Dr_{\partial\Omega}(\theta) = F(\theta). \]



%%% Local Variables:
%%% mode: latex
%%% TeX-master: "../main"
%%% End:

\paragraph{SDEs and the Fokker-Planck Equation}
Assume we are given a drift $\mu:[0,1]\times \mathbb R^d\to\mathbb R^d$ and diffusivity $\sigma:[0,1] \to \mathbb R^{d\times d}$, we consider the SDE
\begin{align}\label{eq:sde_abstract}
    \mathrm dX_t
    &=
    \mu(t, X_t)\mathrm d t
    +
    \sigma(t) \mathrm d W_t,
    \quad 
    X_0 \sim p_0,
\end{align}
where $W_t$ is a standard Brownian motion and $X_0$ is distributed according to $p_0$. To solve this SDE, i.e., to obtain the dynamics of the associated probability density we can solve the \emph{Fokker-Planck equation}, which is given by
\begin{align}\label{eq:fokker_planck_abstract}
    \partial_t p
    +
    \operatorname{div}(p\mu)
    -
    \frac12 \operatorname{tr}(\sigma \sigma^\top\nabla^2 p)
    =
    0,
    \quad
    p(0) = p_0.
\end{align}
For numerical stability, transferring this PDE into $\log$-space is commonly employed. The ansatz $q = \log p$ lets us solve \eqref{eq:fokker_planck_abstract} via the \emph{nonlinear} PDE (which is a special case of a Hamilton-Jacobi-Bellman equation) given by
\begin{align}
    \partial_t q
    +
    \operatorname{div}(\mu)
    +
    \nabla q \cdot \mu
    -
    \frac12 \| \sigma^\top \nabla q \|^2
    -
    \frac12 \operatorname{tr}(\sigma \sigma^\top\nabla^2 q)
    =
    0,
    \quad
    q(0) = \log p_0.
\end{align}

\paragraph{A Sanity Check Example}
Set the drift $\mu$ to $\mu(t,x) = -1/2x$ and choose $\sigma=\sqrt 2 I$. As initial condition we use $p_0 \sim \mathcal N(0,I)$. For this concrete data, the Fokker-Planck equation is
\begin{align}\label{eq:fokker_planck_sanity_example}
    \partial_t p(t, x)
    -
    \frac12 \operatorname{div}(xp(t,x))
    -
    \Delta p(t,x)
    =
    0,
    \quad 
    p(0) \sim \mathcal N(0, I).
\end{align}
The solution to this equation is given by
\begin{equation}
    p^*(t, x) \sim \mathcal N(0, \exp(-t)I + (1 - \exp(-t) 2 I)).
\end{equation}

The logarithmic Fokker-Planck equation for this data is 
\begin{equation}
    \partial_t q(t,x)
    -
    \frac d2
    -
    \frac12\nabla q(t,x)\cdot x
    -
    \| \nabla q(t,x) \|^2
    -
    \Delta q(t,x)
    =
    0,
    \quad 
    q(0) = \log(p^*(0)),
\end{equation}
with solution $q^* = \log(p^*)$.

\begin{itemize}
    \item The fact that $\sigma$ is a diagonal matrix simplifies the term with the second partial derivatives. In general, we cannot assume this, so it is interesting to think about Taylor mode differentiation in this context. I know it will work, I am wondering how much we gain over backprop here, especially in high dimensions. 

    Let's look at $\Tr(\mA \mA^\top \nabla^2 q) = \sum_i \sum_j \sum_k \emA_{i,j} \emA_{k,j} (\nabla^2_{k,i} q) \coloneqq \sum_i \sum_k \emB_{k,i} (\nabla^2_{k,i} q)$ where $\mB \coloneqq \mA \mA^\top$. This is just a weighted sum of second-order partial derivatives. As we explain at the end of \Cref{sec:taylor-mode-AD}, we can derive forward propagation rule which looks similar to the forward Laplacian (and I believe has the same computational complexity, but I will need to write down the exact expression).

    
    \item A simple trick let's us encode the initial conditions directly into the neural networks ansatz:
    \begin{equation}
        \tilde p_\theta = t p_\theta + \mathcal N(0, I).
    \end{equation}
    This guarantees that our neural network ansatz, now called $\tilde p_\theta$ exactly satisfies the initial condition. This trick typically helps and should always be employed\footnote{We did not do that for the other equations because the Poisson and the heat equation are interesting in potentially complicated domains, where such a trick for the boundary conditions would be very difficult to realize in practice. Thus it should be considered ``cheating''. The situation for the Fokker-Planck is different. Here this trick---if helpful---should always be used.}. 
    \item The Fokker-Planck equation is a PDE on an unbounded domain, typically $\mathbb R^d$, so we need to artificially restrict ourselves to some compact interval. This causes issues with numerical accuracy and the optimization.
    \item The Fokker-Planck equation is interesting for PINNs as it frequently appears for high-dimensional settings. Hence grid-based methods are intractable. This is our motivation to look at this equation.
    \item The example above will not work well for high dimensions when using the original Fokker-Planck equation. The log-space ansatz is expected to help and seems to be the standard in the community.
\end{itemize}
Consider the $(\tilde{d}+1)$-dimensional homogeneous heat equation
\begin{align*}
  \partial_{t} u(t, \tilde{\vx})
  -
  \kappa \Delta_{\tilde{\vx}} u(t, \tilde{\vx})
  =
  0
\end{align*}
with spatial coordinates $\tilde{\vx} \in \Omega \subseteq \sR^{\tilde{d}}$ and time coordinate $t \in \mathrm{T} \subseteq \sR$ where $\mathrm{T}$ is a time interval and $\kappa >0$ denotes the heat conductivity. In this case, our neural network processes a $(d = \tilde{d} +1)$-dimensional vector $\vx = ( t,  \tilde{\vx}^{\top} )^{\top} \in \sR^d$ and we can re-write the heat equation as
\begin{align*}
  \partial_{\evx_1} u(\vx)
  -
  \kappa \sum_{d=2}^{d} \Delta_{\evx_d} u(\vx)
  =
  0\,.
\end{align*}
In the following, we consider the unit time interval $\mathrm{T} = [0;1]$, the unit square $\Omega = [0;1]^{\tilde{d}}$ and set $\kappa = \nicefrac{1}{4}$.
There are two types of constraints we need to enforce on the heat equation in order to obtain unique solutions: initial conditions and boundary conditions.
As our framework for the KFAC approximation assumes only two terms in the loss function, we combine the contributions from the boundary and initial values into one term.

To make this more precise, consider the following example solution of the heat equation, which will be used later on as well.
As initial conditions, we use $u_0(\tilde{\vx}) = u(0, \tilde{\vx}) = \prod_{i=1}^{\tilde{d}} \sin(\pi \tilde{\evx}_i)$ for $\tilde{\vx} \in \Omega$.
For boundary conditions, we use $g(t, \tilde{\vx}) = 0$ for $(t, \tilde{\vx}) \in \mathrm{T} \times \partial\Omega$.
The manufactured solution is
\begin{align*}
  u_{\star}(t, \tilde{\vx})
  =
  \exp \left(-\frac{\pi^2 \tilde{d} t}{4} \right)
  \prod_{i=1}^{\tilde{d}} \sin(\pi [\tilde{\evx}_i])\,.
\end{align*}
The PINN loss for this problem consists of three terms: a PDE term, an initial value condition term, and a spatial boundary condition term,
\begin{align*}
  L(\vtheta)
  &=
    \frac{1}{N_{\Omega}}
    \sum_{n=1}^{N_{\Omega}}
    \left(
    \partial_t u_{\vtheta}(\vx_n^{\Omega})
    -
    \frac{1}{4} \Delta_{\tilde{\vx}_n} u_{\vtheta}(\vx_n^{\Omega})
    \right)^2
  \\
  &+
    \frac{1}{N_{\partial\Omega}}
    \sum_{n=1}^{N_{\partial\Omega}}
    \left(
    u_{\vtheta}(\vx_n^{\partial\Omega})
    -
    g(\vx_n^{\partial\Omega})
    \right)^2
  \\
  &+
    \frac{1}{N_0}
    \sum_{n=1}^{N_0}
    \left(
    u_{\vtheta}(0, \vx_n^0)
    -
    u_0( \vx_n^0)
    \right)^2
\end{align*}
with $\vx_n^{\Omega} \sim \mathrm{T} \times \Omega$, and $\vx_n^{\partial\Omega} \sim \mathrm{T} \times \partial\Omega$, and $\vx_n^0 \sim \{0\} \times \Omega$.
To fit this loss into our framework which assumes two loss terms, each of whose curvature is approximated with a Kronecker factor, we combine the initial value and boundary value conditions into a single term.
Assuming $N_{\partial \Omega} = N_0 = \nicefrac{N_{\text{cond}}}{2}$ without loss of generality, we write
\begin{align*}
  L(\vtheta)
  &=
    \underbrace{
    \frac{1}{N_{\Omega}}
    \sum_{n=1}^{N_{\Omega}}
    \left\lVert
    \partial_t u_{\vtheta}(\vx_n^{\Omega})
    -
    \frac{1}{4} \Delta_{\tilde{\vx}_n} u_{\vtheta}(\vx_n^{\Omega})
    - y_n^{\Omega}
    \right\rVert^2_2
    }_{L_{\Omega}(\vtheta)}
  +
    \underbrace{
    \frac{1}{N_{\text{cond}}}
    \sum_{n=1}^{N_{\text{cond}}}
    \left\lVert
    u_{\vtheta}(\vx_n^{\text{cond}})
    -
    y_n^{\text{cond}}
    \right\rVert^2_2
    }_{L_{\text{cond}}(\vtheta)}
\end{align*}
with domain inputs $\vx_n^{\Omega} \sim \mathrm{T} \times \Omega$ and targets $y_n^{\Omega} = 0$, boundary and initial condition targets $y_n^{\text{cond}} = u_\star(\vx_n^{\text{cond}})$ with initial inputs $\vx_n^{\text{cond}} \sim \{0\} \times \Omega$ for $n = 1, \dots, \nicefrac{N_{\text{cond}}}{2}$ and boundary inputs $\vx_n^{\text{cond}} \sim \mathrm{T} \times \partial\Omega$ for $n = \nicefrac{N_{\text{cond}}}{2}+1, \dots, N_{\text{cond}}$.
This loss has the same structure as the PINN loss in \Cref{eq:pinn-loss}.

%%% Local Variables:
%%% mode: latex
%%% TeX-master: "../main"
%%% End:

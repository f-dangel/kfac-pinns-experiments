PINNs use two losses: a boundary and an interior loss.
When designing a Kronecker-factored curvature approximation, we have two choices:
\begin{enumerate}
\item We can approximate each loss individually with a Kronecker product, i.e.
  \begin{align*}
    \mK \coloneqq \mA_1 \otimes \mA_2 + \mB_1 \otimes \mB_2
  \end{align*}
  where $\mA_{1,2}, \mB_{1,2}$ are invertible and positive definite.

\item We can further approximate the above through a single Kronecker product, e.g.\,by summing,
  \begin{align*}
    \mK' \coloneqq (\mA_1 + \mB_1) \otimes (\mA_2 + \mB_2) \approx \mK\,.
  \end{align*}
\end{enumerate}
Eventually, we want to use their inverses.
For $\mK'$, this is straightforward to do.
For $\mK$, we need to invert the sum of two Kronecker products, which is more challenging.
We can proceed as follows to invert $\mK$:
\begin{enumerate}
\item Simultaneously diagonalize $(\mA_i, \mB_i)$ by solving the generalized eigenvalue problem
  \begin{align*}
    \mA_i \mV_i = \mB \mV_i \mLambda_i
  \end{align*}
  where $\mV_i$'s columns are the (generalized) eigenvectors and $\mLambda_i$ is a diagonal matrix containing the (generalized) eigenvalues.

\item Consider the expression
  \begin{align*}
    &\left(
      \mA_1 \otimes \mA_2
      +
      \mB_1 \otimes \mB_2
      \right)
      \left(
      \mV_1 \otimes \mV_2
      \right)
    \\
    &=
      \mB_1 \mV_1 \mLambda_1 \otimes \mB_2 \mV_2 \mLambda_2
      +
      \left( \mB_1 \otimes \mB_2 \right)
      \left( \mV_1 \otimes \mV_2 \right)
    \\
    &=
      \left( \mB_1 \otimes \mB_2 \right)
      \left( \mV_1 \otimes \mV_2 \right)
      \left(
      \mLambda_1 \otimes \mLambda_2 + \mI \otimes \mI
      \right)\,.
  \end{align*}
  Rearrange into
  \begin{align*}
    \left(
    \mA_1 \otimes \mA_2
    +
    \mB_1 \otimes \mB_2
    \right)
    =
    \left( \mB_1 \otimes \mB_2 \right)
    \left( \mV_1 \otimes \mV_2 \right)
    \left(
    \mLambda_1 \otimes \mLambda_2 + \mI \otimes \mI
    \right)
    \left( \mV_1^{-1} \otimes \mV_2^{-1} \right)
    \,.
  \end{align*}

  \item Take the inverse to obtain
    \begin{align*}
      \mK^{-1}
      =
      \left( \mV_1 \otimes \mV_2 \right)
      \left(
      \mLambda_1 \otimes \mLambda_2 + \mI \otimes \mI
      \right)^{-1}
      \left( \mV_1^{-1}\mB_1^{-1} \otimes \mV_2^{-1}\mB_2^{-1} \right)
      \,.
    \end{align*}
    Notice that $\mLambda_1 \otimes \mLambda_2 + \mI \otimes \mI$ is diagonal, and therefore easy to invert.
\end{enumerate}

%%% Local Variables:
%%% mode: latex
%%% TeX-master: "../main"
%%% End:

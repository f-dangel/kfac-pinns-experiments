\section{Example: Laplacian for a Shallow Network}

Here, we compute the Laplacian for a less general, yet simpler, example: A shallow neural network.

\begin{comment}
  For the one-dimensional shallow case we have
  \[ u_\theta(x) = \sum_{i=1}^m a_i\sigma(b_i x + c_i) + d. \]
  Then
  \[ u_\theta'(x) = \sum_{i=1}^m a_ib_i\sigma'(b_i x + c_i) \]
  and
  \[ u_\theta''(x) = \sum_{i=1}^m a_ib_i^2\sigma''(b_i x + c_i). \]
  The parameter derivative of the Laplacian (i.e., second-order derivative)
  \begin{align*}
    \partial_{a_i}u_\theta''(x) & = b_i^2\sigma''(b_i x + c_i) \\
    \partial_{b_i}u_\theta''(x) & = 2a_ib_i\sigma''(b_i x + c_i) + a_i b_i^2x\sigma^{(3)}(b_i x + c_i) \\
    \partial_{c_i}u_\theta''(x) & = a_i b_i^2\sigma^{(3)}(b_i x + c_i) \\
    \partial_{d}u_\theta''(x) & = 0.
  \end{align*}
  The parameter derivative of the Laplacian (i.e., second-order derivative)
  \begin{align*}
    \partial_{a_i}u_\theta''(x) & = b_i^2\sigma''(b_i x + c_i) \\
    \partial_{b_i}u_\theta''(x) & = 2a_ib_i\sigma''(b_i x + c_i) + a_i b_i^2x\sigma^{(3)}(b_i x + c_i) \\
    \partial_{c_i}u_\theta''(x) & = a_i b_i^2\sigma^{(3)}(b_i x + c_i) \\
    \partial_{d}u_\theta''(x) & = 0.
  \end{align*}
\end{comment}

First, we do the Hessian calculation as explicitly as possible. For this, we consider the easy case of a shallow network with one-dimensional output
\[ u_\theta(x) = \sum_{i=1}^m a_i\sigma(b_i^\top x + c_i) + d. \]
Then
\[ \nabla_x u_\theta(x) = \sum_{i=1}^m a_ib_i\sigma'(b_i x + c_i) \]
or
\[ \partial_{x_j} u_\theta(x) = \sum_{i=1}^m a_i(b_i)_j\sigma'(b_i x + c_i). \]
Consequently, we compute
\[ \partial_{x_k}\partial_{x_j} u_\theta(x) = \sum_{i=1}^m a_i(b_i)_j(b_i)_k\sigma'(b_i x + c_i) \]
which yields the Hessian
\[ \nabla^2_x u_\theta(x) = \sum_{i=1}^m a_i\sigma''(b_i x + c_i) \cdot b_ib_i^\top = \sum_{i=1}^m a_i\sigma''(b_i x + c_i) \cdot  b_i\otimes b_i \]
and the Laplacian is given by
\[ \Delta_x u_\theta(x) = \sum_{i=1}^m a_i\sigma''(b_i x + c_i) \operatorname{tr}(b_ib_i^\top) = \sum_{i=1}^m a_i\sigma''(b_i x + c_i)  \operatorname{tr}(b_i\otimes b_i). \]
The parameter derivatives of the Hessian are given
\begin{align*}
  \partial_{a_i}\nabla_x^2 u_\theta(x) & = \sigma''(b_i x + c_i) b_ib_i^\top \\
  \nabla_{b_i}\nabla_x^2 u_\theta(x) & = a_i\sigma''(b_i x + c_i) (b_i\otimes I+I\otimes b_i) + a_i \sigma^{(3)}(b_i x + c_i) \cdot  (I\otimes x^\top)(b_i\otimes b_i) \quad ?? \\
  % + a_i x\sigma^{(3)}(b_i x + c_i) \cdot b_ib_i^\top \\
     \partial_{c_i}\nabla_x^2 u_\theta(x) & = a_i b_i^2\sigma^{(3)}(b_i x + c_i) \\
     \partial_{d}\nabla_x^2 u_\theta(x) & = 0.
\end{align*}
The parameter derivatives of the Laplacian are given
\begin{align*}
     \partial_{a_i}\Delta_x u_\theta(x) & = \sigma''(b_i x + c_i) \operatorname{tr}(b_ib_i^\top) \\
     \nabla_{b_i}\Delta_x u_\theta(x) & = 2a_i\sigma''(b_i x + c_i) + a_i \sigma^{(3)}(b_i x + c_i) \operatorname{tr}(b_ib_i^\top) \cdot x \\
     \partial_{c_i}\Delta_x u_\theta(x) & = a_i \sigma^{(3)}(b_i x + c_i)\operatorname{tr}(b_ib_i^\top) \\
     \partial_{d}\Delta_x u_\theta(x) & = 0.
\end{align*}


\paragraph{Writing as linear algebra}

Consider a shallow network
\[ u_\theta(x) = W_2\sigma(W_1 x), \]
where $x\in\mathbb R^d$, $W_1\in\mathbb R^{m\times d}, W_2\in\mathbb R^{1\times m}$.

\paragraph{Task: }
Understand the structure, in particular the computational graph of $f_\theta = \Delta u_\theta$.

\paragraph{One dimensional case: }
Assume first that $d=1$.
Then
\[ D_x u_\theta(x) = W_2 \operatorname{diag}(\sigma'(W_1 x)) W_1 \]
and
\[ D^2_x u_\theta(x) = W_2%(W_1^\top\otimes W_2)
  \operatorname{diag}(\operatorname{diag}(\sigma''(W_1 x)) W_1)W_1. % = W_2 \operatorname{diag}\sigma''(W_1 x) W_1^2 = W_2\odot W_1^2 \sigma''(W_1x),
\]
In another form
\[D^2 u_\theta(x) = W_2 \operatorname{diag}(\operatorname{diag}\sigma''(W_1 x) W_1) W_1 = W_2 \operatorname{diag}\sigma''(W_1 x) W_1^2 = W_2\odot W_1^2 \sigma''(W_1x),
\]
where $(W_2\odot W_1^2)_k = (W_2)_k \cdot (W_1)_k^2$.
% where $(W_2\odot W_1^2)_k = (W_2)_k \cdot (W_1)_k^2$.
% The Laplacian is then given as
% \[ f_\theta(x) = \Delta u_\theta(x) = \operatorname{tr}(D_x^2u_\theta(x)) =\operatorname{tr}(W_2 \operatorname{diag}(\operatorname{diag}(\sigma''(W_1 x)) W_1) W_1).  \]
% Differentiation with respect to $W_2$ yields
% \[ \frac{\partial f_\theta(x)}{\partial W_2} = \frac{\partial \Delta u_\theta(x)}{ \partial W_2} =  \]

One can also rewrite the diagonalization operator via a Hadamard product as $\operatorname{diag}(x) = (x\mathds{1}^\top)\odot I$, where $\mathds{1}$ denotes the all one vector, $I$ the identity and $\odot$ the Hadamard (i.e., entrywise) product of two matrices.
Not sure whether this helps though...

\paragraph{Multi-dimensional case}

%%% Local Variables:
%%% mode: latex
%%% TeX-master: "../main"
%%% End:
